% -*- compile-command: "pdflatex class-notes.tex" -*-

\documentclass{article}

\usepackage{hyperref}

\begin{document}

\title{CSCI 150: Foundations of Computer Science \\ Fall 2015 Lecture Notes}

\maketitle

\section{Introduction (Wednesday, 26 August)}

\subsection*{Setup}

\begin{itemize}
\item Music: Copland, Fanfare for the Common Man
\item Meet students.
\item Introduce myself.
\item Have students meet each other. (Name, where from, why taking
  this class, one of your favorite things.)  Prize for first student
  to know all names.
\end{itemize}

\subsection*{What is Computer Science?}

This class is Foundations of Computer Science.  What is Computer
Science?  Look at each component separately.
\begin{itemize}
\item \textbf{What is a computer?}  Get some responses.
\item \textbf{What do computers do?}  Lots of things.  Commonality:
  information.  Communicating, transforming, analyzing, storing.
\item \textbf{What is science?}  Get some responses.  Scientists use
  rational investigation \& analysis, mathematics, etc. to study the
  natural world.
\end{itemize}

So what do computer scientists study?  \textbf{NOT computers!}
Rather, \textbf{information} = anything that can be expressed digitally,
i.e. with numbers/symbols.  ``Computer science'' is actually a bad
name, \emph{cf.}\ ``telescope science''.  Information structure of the
universe.

Why study it?  (Maybe get some student responses)
\begin{itemize}
\item Beautiful ideas, new ways of thinking.
\item Many applications!  Can contribute directly to human flourishing.
\item Computers are everywhere.  Understanding principles of CS =
  being an informed, engaged citizen.
\end{itemize}
Could do this without a computer, but computers are excellent enabling
tools.

\href{https://www.youtube.com/watch?v=qYZF6oIZtfc}{\textbf{Watch code.org video}}.

\subsection*{Administrivia}

\begin{itemize}
\item Syllabus review online.
\item Academic integrity.
\item BYOL.
\item Moodle.
\item Office hours.
\item Remember to come to lab!
\end{itemize}

For next time:

\begin{itemize}
\item Read chapter 1 of textbook.
\item Do HW 0, ``who are you'', if not already done.
\end{itemize}

\subsection*{Scratch intro}

Basic intro to Scratch for first lab.

\newpage

\section{Algorithms and language (Friday, 28 August)}

\subsection*{Setup}

\begin{itemize}
\item Music: Attaboy from Goat Rodeo Sessions. 5:42.  Start at 8:04.
\item Bring origami paper.
\end{itemize}

\subsection*{Administrivia}

\begin{itemize}
\item I will be gone next week.  ICFP.  Functional Programming ---
  super cool.  Really get to see intersection of math \& CS. Take CSCI
  490 in the spring (with MATH 240)!
\item In my place, Connor Bell will be lecturing \& will introduce you
  to Python.  Please show him the same respect you would show me!
\item Please download Python for Monday.  (Show python website, linked
  from our webpage.  Use version 2.7.)  \textbf{Bring your laptop!}
\end{itemize}

\subsection*{IRC}

Show IRC channel. Explain what it is for. Explain ground rules. Have
everyone log in \& try it out.

\subsection*{Reading review}

\begin{itemize}
\item Review 5 aspects of algorithms (input, output, math,
  conditionals, repetition).  Where did we see them in Scratch? Didn't
  use conditionals but saw the others.
\item Review 3 kinds of errors (syntax, semantic, runtime).  Talk
  about each in context of Scratch (doesn't have syntax errors !!!; lots
  of semantic errors; runtime = running into wall)
\item Fact that Scratch doesn't have syntax errors is a Really Big
  Deal.  Imagine if when learning a foreign language, every time you
  made even a small grammatical mistake the other person just cut you off and
  said ``I don't understand.''  That's what it will feel like learning
  Python.  So don't be discouraged---remember what you could do with
  Scratch.  You'll get there with Python too.
\item Talk about formal vs natural languages, tokens, parsing
\end{itemize}

\textbf{Quiz Monday on Chapter 1}.

\subsection*{Collatz Conjecture}

\begin{itemize}
\item Write out hailstone function.
\item Note all 5 aspects of algorithm.
\item Try it on some inputs: 4, 8, 6, 11. Draw a tree etc.  Try 27,
  realize we need a computer.
\item Simple algorithms can have very surprising results!  Visit
  Wikipedia page.
\end{itemize}

\subsection*{Origami}

\begin{itemize}
\item Explain what we are going to do: make a dinosaur.
  \begin{itemize}
  \item Everyone will be able to do it by the end of class.
  \item Then write instructions IN ENGLISH and find a confederate to
    follow them.  Can't show them image or video, or your dinosaur.
    Can't help interpret your instructions.  Just tell them to keep
    going as well as they can and get to the end.
  \item Monday: turn in both dinosaurs in class.  Turn in your
    instructions \& writeup on Moodle.
  \end{itemize}

\item Hand out origami paper.
\item Show video \& instruction image on the screen at the same time.
\item Wander around and make sure everyone can do it.
\end{itemize}

\section{XXX}

\section{XXX}

\section{XXX}

\newpage
\section{Conditionals (Wednesday, 9 September)}

\subsection*{Setup}

\begin{itemize}
\item Music: Whitacre ``i thank You God''. 6:56.  Start at 8:03.
\item Talk about Fabienne Serriere.  Show website.
\item Collect puzzle HW.
\end{itemize}

\subsection*{Quiz}
\textbf{Quiz 2}

We saw Boolean values {\tt True} and {\tt False}, in the puzzles, with
operators {\tt and}, {\tt or}, {\tt not} last time.  How else can we get
Booleans?

\textbf{Comparison operators}: \verb|>|, \verb|<|, \verb|>=|,
\verb|<=|, \verb|!=|, \verb|==|.  Demonstrate on numbers, strings, all
generate Booleans.

Now, the computer has a basis to make decisions.  Sets up branches in
the code, execute this or that.

\begin{verbatim}
passwd = input("What is the password? ")
if (passwd == "lemur"):
    print "Here be secrets."
\end{verbatim}

Tabbing is important!  Colon is important!

Introduce random module. ({\tt from random import *})
\begin{itemize}
\item {\tt random()} is number in $[0,1)$. Uniform.
\item {\tt randint(n)} is random integer between $0$ and $n$
  (inclusive).  (Aside: how can we write this in terms of {\tt
    random()}?)
\end{itemize}
Do coin flip.  Conditionally do... something.  Needs to include
\begin{itemize}
\item {\tt else}
\item {\tt elif}
\item nested {\tt if}  (flip two coins in a row)
\end{itemize}

\newpage
\section{Information encoding I (Friday, 11 September)}

\subsection*{Setup}

\begin{itemize}
\item Music: 
\end{itemize}

\newpage
\section{Information encoding II (Monday, 14 September)}



\end{document}
