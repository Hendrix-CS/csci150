% -*- compile-command: "pdflatex class-notes.tex" -*-

\documentclass{article}

\usepackage{hyperref, amsmath, xspace}

\newsavebox{\reflectbox}

\newcommand{\etc}{\emph{etc.}\xspace}

\newenvironment{reflect}[1]
{
  \begin{lrbox}{\reflectbox}
    \begin{minipage}[t]{\textwidth}
      \textbf{#1}:
}{
    \end{minipage}
  \end{lrbox}
  \fbox{\usebox{\reflectbox}}
}

\begin{document}

\title{CSCI 150: Foundations of Computer Science \\ Spring 2016 Lecture Notes}

\maketitle

\section{Introduction}

\subsection*{Setup}

\begin{itemize}
\item {}[In lab: put website up on projector so everyone knows they are
  in the right place.]
\item {}[Meet students before class.]
\item Have students meet each other. Use binary cards to put them in
  groups of 4.  Then switch groups once or twice.  Introduce people
  you have already met.
\end{itemize}

\subsection*{What is Computer Science?}

This class is Foundations of Computer Science.  What is Computer
Science?  (Terrible name: first, because why do you need to have
``science'' in the title?  e.g. ``food product''.)  Look at each
component separately.
\begin{itemize}
\item \textbf{What is a computer?}  Get some responses.  (Can insert
  brief history lesson: Antikythera mechamism: 150-100 BC, analog
  computer.  Curta mechanical calculator --- watch
  movie?. ``Computers'' referred to people.  First electronic
  computers: WWII, early 1940s.)
\item \textbf{What do computers do?}  Lots of things.  Commonality:
  information.  Communicating, transforming, analyzing, storing.
\item \textbf{What is science?}  Get some responses.  Scientists use
  rational investigation \& analysis, mathematics, \etc to study the
  natural world.
\end{itemize}

So what do computer scientists study?  \textbf{NOT computers!}
Rather, \textbf{information} = anything that can be expressed digitally,
i.e. with numbers/symbols.  ``Computer science'' is actually a bad
name, \emph{cf.}\ ``telescope science''.  Information structure of the
universe.

Why study it?  (Maybe get some student responses)
\begin{itemize}
\item Beautiful ideas, new ways of thinking.
\item Many applications!  Can contribute directly to human flourishing.
\item Computers are everywhere.  Understanding principles of CS =
  being an informed, engaged citizen.
\end{itemize}
Could do this without a computer, but computers are excellent enabling
tools.

\href{https://www.youtube.com/watch?v=qYZF6oIZtfc}{\textbf{Watch code.org video}}.

\subsection*{Administrivia}

\begin{itemize}
\item Syllabus review online.
\item Academic integrity.
\item Moodle.
\item Office hours.
\end{itemize}

For next time:

\begin{itemize}
\item Read chapter 1 of textbook.
\item Do HW 0, ``who are you'', if not already done.
\end{itemize}

\emph{Note: in Spring 2016, lab was on Tuesday, hence the first
  meeting was a lab meeting.  Did Minecraft Hour of Code.}

\subsection*{Scratch intro (?)}

Basic intro to Scratch for first lab.

\subsection*{Minecraft Hour of Code (?)}

\subsection*{Reflections}

\begin{reflect}{Brent, S'16}
  Doing the first lecture together with Hour of Code lab worked really
  well in S'16.

  Next time using Minecraft Hour of Code: tell them they get a C for
  doing X many, a B for finishing all, and an A for doing them all
  with the minimum number of instructions.
\end{reflect}

\newpage

\section{Algorithms and language}

\subsection*{Setup}

\begin{itemize}
\item Bring origami paper (?).
\item Mention TA hours and my office hours
\end{itemize}

\subsection*{Reading review}

\begin{itemize}
\item Review 5 aspects of algorithms (input, output, math,
  conditionals, repetition).  Where did we see them in
  Scratch/Minecraft?
\item Review 3 kinds of errors (syntax, semantic, runtime).  Talk
  about each in context of Scratch/Minecraft (doesn't have syntax
  errors !!!; lots of semantic errors; runtime = running into
  wall/falling into lava)
\item Fact that Scratch/Minecraft doesn't have syntax errors is a
  Really Big Deal.  Imagine if when learning a foreign language, every
  time you made even a small grammatical mistake the other person just
  cut you off and said ``I don't understand.''  That's what it will
  feel like learning Python.  So don't be discouraged---remember what
  you could do with Scratch/Minecraft.  You'll get there with Python
  too.
\item Talk about formal vs natural languages, tokens, parsing
\end{itemize}

\textbf{Quiz Monday on Chapter 1}.

\subsection*{Collatz Conjecture}

\begin{itemize}
\item Write out hailstone function.
\item Note all 5 aspects of algorithm.
\item Try it on some inputs: 4, 8, 6, 11. Draw a tree etc.  Try 27,
  realize we need a computer.
\item Simple algorithms can have very surprising results!  Visit
  Wikipedia page.
\end{itemize}

\subsection*{Origami}

\begin{itemize}
\item Explain what we are going to do: make a dinosaur.
  \begin{itemize}
  \item Everyone will be able to do it by the end of class.
  \item Then write instructions IN ENGLISH and find a confederate to
    follow them.  Can't show them image or video, or your dinosaur.
    Can't help interpret your instructions.  Just tell them to keep
    going as well as they can and get to the end.
  \item Monday: turn in both dinosaurs in class.  Turn in your
    instructions \& writeup on Moodle.
  \end{itemize}

\item Hand out origami paper.
\item Show video \& instruction image on the screen at the same time.
\item Wander around and make sure everyone can do it.
\end{itemize}

\subsection*{Reflections}

\begin{reflect}{Brent, S'16}
  This whole lecture feels sort of disjointed and unmotivated.  The
  concepts are important but maybe

  In F'15 we were scrambling to finish our dinosaurs right at the end.
  In S'16 I therefore tried to go a bit faster and apparently overdid
  it---everyone was done with dinosaurs 10--15 minutes early!
\end{reflect}

\newpage

\section{Drake Equation, Python I}

\subsection*{Setup}

\begin{itemize}
\item \textbf{Quiz}: list 4 of the 5 elements of any algorithm.
\item For this week, read Ch 2 and 3. Probably quiz Wednesday.
\item Remember dinosaurs are due on Monday.  Turn in document on
  Moodle, dinosaurs in class.
\end{itemize}

\subsection*{Drake Equation}

Today, we want the computer to calculate a simple formula for us.  The
class is about communication and intelligence.  Frank Drake, founder
of SETI, came up with an equation to help us see if it's actually
feasible that we'll find ETs to communicate with.
\[ N = R^* f_p n_e f_l f_i f_c L \]
where
\begin{align*}
R^* &= \text{average rate of star formation} \\
F_p &= \text{fraction that have planets} \\
N_e &= \text{number of planets per star} \\
F_l &= \text{fraction that have life} \\
F_i &= \text{fraction that develop intelligence} \\
F_c &= \text{fraction of civs that release to space} \\
L &= \text{how long the signals are transmitted}
\end{align*}

Many people have different numbers for this equation, how to quickly
calculate the answer?  This is our problem, let's use python to solve
it.  Here are the steps we will go through,

\begin{enumerate}
\item Analyze Problem
\item Determine Input/Output
\item Create Pseudocode
\item Implement in Code
\item Test/Debug
\item Maintain
\end{enumerate}

OK, let's go through the steps.

\begin{enumerate}
\item Analyze: We want the user to tell us values for the right hand
  side of the equation, and we will tell them the expected number of
  civs we will encounter.

\item I/O: Inputs are $R^*$, $f_p$, \etc  The output is $N$.

\item Design: Here's our algorithm in psedocode/English:

  Tell the user about the equation \\
  Ask for $R^*$ \\
  Store $R^*$ \\
  Ask for $f_p$ \\
  Store $f_p$ \\
  \etc \\
  Calculate $N$ \\
  Display $N$ to the user.

\item Implement: use Python, but how?

\end{enumerate}

\subsection*{Python I (Strings, Numbers, Variables)}

Show \url{python.org} website.  Load up IDLE, Python 2.7.

\texttt{>>>} is the prompt.  This gives you direct access to the
interpreter.  You say commands one line at a time, and get immediate
answers.  Great for testing simple lines of code and seeing what they
do.

What kinds of things does Python give us?

\begin{itemize}
\item \textbf{Strings}!

  Chunks of text, in quotes. Just like you're reading a book:
  ``Brilliant!'' Harry said.

  Use \texttt{type} to ask what type it is.

  Both types of quotes work (single, double), must end with same you
  start with. And for really big strings, use triple quotes like
  \texttt{""" lkwjelrkjwe """}

  How can we break this? \texttt{'haven't you heard?'}  Use one to
  enclose the other.  If we need both, use \texttt{"""}, or the escape
  character \texttt{\textbackslash}: \texttt{\textbackslash ", \textbackslash t,
    \textbackslash n, \textbackslash\textbackslash}. These are special.

  The red text we see when things go wrong are examples of Syntax
  Errors.

\item \textbf{Numbers}!

  2, 5, 10, integers, \etc  Use \texttt{type}.

  Simple math with numbers like $2+5$. \texttt{"2 + 5"}? not so much.
  Great as a quick calculator. Use () to force order of operations
  differences.

  Integers are great, but what about fractions? \texttt{2.5}
  \texttt{0.95}, \texttt{3e23}.

  One gotcha: $3 / 4$ vs $3.0 / 4.0$.  Use \verb|type| to see the
  types of different things.  Maybe mention mod operator \verb|%|,
  exponentiation \verb|**|.

\item \textbf{Variables}!

  For keeping stuff around in short-term memory.
  \begin{verbatim}
  x = 3
  name = "Brent"
  \end{verbatim}
  These are \emph{assignment} statements.  No output any more. Before,
  any value returned from the interpreter to us, now it goes into
  memory.  Type \texttt{x}, it spits back 3.

  Draw pictures, \texttt{x} points to $3$, it's just a name for the
  value $3$.

  \texttt{x = 4} reassigns \texttt{x} to be a new number.  \texttt{x =
    "Brent"} works too.

  \texttt{y} is a runtime error, give us an \emph{exception}, which
  means something is not right (the \emph{other} meaning of
  ``exceptional''). So doublecheck your names!

  Some names are reserved, we'll see them as we go.
  Names can't be anything, letter or underscore first, then letter,
  number or underscore.

  Be descriptive, a, b, c, d, e, don't really help
  Use \texttt{width}, \texttt{height}, \texttt{num\_of\_cats}, \etc
\end{itemize}

\begin{reflect}{Brent, S' 16}
  Didn't make it to variables.
\end{reflect}

\newpage

\section{Python II, Drake}

\subsection*{Setup}
\begin{itemize}
\item Collect dinosaurs.  Write your name on it!
\item Read Chapters 2 \& 3 --- quiz on Wednesday!
\item Start reading Chapter 4 for Wednesday.
\end{itemize}

\subsection*{Variables}

[Finish up variables.  Show @ prompt in IDLE.  See notes from previous
lecture.]

\subsection*{I/O, .py files}

\verb|print|.  The interpreter is great, but we don’t want the user
having to type python code, we make it easier for them.  We do
calculations, and tell the computer to print the output to the user.

\begin{itemize}
\item \verb|print <expr>| --- instructs Python to display information
  to the user.
\item For example:
  \begin{itemize}
  \item \verb|print "Hello world"|
  \item
\begin{verbatim}
x = 3
y = 4
print x, y
\end{verbatim}
  \item
\begin{verbatim}
print """ kjbwer
Lkjwelrkj
Lkjwelkrj
"""
\end{verbatim}
  \end{itemize}

\item Put stuff in a \verb|.py| file with \verb|print| and run it.

\item \verb|raw_input()|, lets us talk to the user and ask questions
  at the same time.  Example:
\begin{verbatim}
raw_input(“What is your name?”)
\end{verbatim}
\item Note it doesn’t give you back the prompt \verb|>>>|, but waits
  for you to type; whatever you type comes back as a string.

\item  To do anything with this we need to save the result in a variable:
\begin{verbatim}
name = raw_input(“What is your name?”)
\end{verbatim}

\item  But what if we want numbers? Functions to convert strings into
  integers or floating point numbers:
\begin{verbatim}
int("23")
float("32.4")
\end{verbatim}

\item So to get numbers from the user, we can do something like:
\begin{verbatim}
age_st = raw_input("What is your age?")
age = int(age_st)
\end{verbatim}
\end{itemize}

\subsection*{REPLs, PROGRAMS, etc.}

Mark: Useful here to talk about the difference between the REPL loop,
and the input and print connections to the user. Draw diagram. What we
type in python at the \verb|>>>| is the PROGRAM, it goes to the CPU,
and the answers are returned to the PROGRAM. What happens when the
PROGRAM says print and input, these questions come from the CPU to the
USER and the input from the USER goes back to the CPU, which then
returns it to the PROGRAM.

Brent prefers something like the following: in a PROGRAM (saved in a
file) we have to be explicit about I/O: input only comes via
\verb|raw_input| and output only produced via \verb|print|.  The REPL
handles input and ouptut for us: \verb|>>>| prompt does input and
everything is implicitly \verb|print|ed for us.  Draw box representing
program with inputs coming in, outputs going out via these functions.
REPL is like a little wrapper around a smaller box where we don't have
to worry about those things.  [A little confusing to think about what
happens when we e.g. call \verb|raw_input| at the REPL prompt but you
don't really need to worry about that!]

\subsection*{Our first program!}

Let's start programming.  We need a place to save our commands, as a
script, so they are all run at once.  This file is produced by a
programmer.  The user will then take this file and run the program.

New file, open this up, save as \verb|drake.py|, so that the colors
will show up in the code to help us out.

First, put name, class, date at top, in a comment block \verb|#######|.

Add in pseudocode, use \verb|#| to mark comments, python will ignore
these lines.

First just do one value, make the equation for \verb|N| very simple.
Calculate and run with F5, or through menus.

Then revise to make it complete.  Get as far as we can, then let them
finish on their own. [In F'15 this was a HW assignment since it
fell over a weekend; in S'16 this lecture was on Monday, with the
Kepler-Newton lab the next day, so not really any point to making it a HW.]

\begin{reflect}{Brent, S' 16}
  Made it through everything and wrote version of Drake equation
  program that asked for first two variables and multiplied them.
  That's enough to get them through the Kepler \& Newton lab.
\end{reflect}

\newpage

\section{LAB: Kepler \& Newton}

\begin{reflect}{Brent, S' 16}
This lab seemed to go pretty well.  Common errors:
\begin{itemize}
\item Forgetting that e.g. \verb|1/3| is 0, not $0.333\dots$ (to take
  cube root)
\item Dividing by $1.4\times 10^n$ is not the same as multiplying by
  $1.4 \times 10^{-n}$.
\end{itemize}

\end{reflect}

\newpage

\section{More Math and Functions}

\textbf{NOTE: this is here as a separate lecture, but really needs to
  just be done as a mini-lecture at the beginning of the Kepler/Newton
  lab!}

\begin{reflect}{Brent, S' 16}
  In S'16 I had already covered a bunch of this (\verb|type()|, mod,
  div) in previous lectures, so I really only had to spend 15 mintues
  talking about converting between numeric types, and importing
  \verb|math|, style check.
\end{reflect}

\subsection*{Setup}
\begin{itemize}
\item For next time, read chapter 4.
\end{itemize}

Talk about types again.  Review, or show again, use of \verb|type()|
to determine what python thinks the type of a literal or variable is.
For example:
\begin{verbatim}
type("hello") = str
type(3) = int
type(3.14) = float
x = 3
type(x) = int
x = 3.14
type(x) = float
\end{verbatim}

Other types of formulas.  Remember, () very important in math.
Mention exponents are with \verb|**| if not already mentioned. Recall
order of operations.

Mention \verb|%|, mod operator, if not already mentioned.
Talk about \verb|5 / 2 = 2| if not already mentioned.

We saw last time string conversion to \verb|int| and \verb|float|: Can
also use these to go from \verb|float| to \verb|int| and back.  Also,
we can go from either to string with e.g. \verb|str(3)|.

What we've been using are \emph{functions}.  Some come standard with
python, like \verb|print|, \verb|raw_input|, \verb|type|, and type
conversion functions like \verb|int|, \verb|float|, \verb|str|.

Many more possible, and in the near future, you will write your own!
For now, if you want to use functions someone else wrote, you have to
import them.

\textbf{DO THIS AT THE TOP OF THE FILE, NOT SOMEWHERE IN THE MIDDLE!}
It helps clarify what you'll need for the code, and is good etiquette.

Our first \emph{module} today, \verb|math|:
\begin{verbatim}
import math
\end{verbatim}

Lots of nice functions in there.  See
\url{http://docs.python.org/lib/module-math.html}.  For example,
\verb|log|, \verb|arctan|, \verb|ceil|, \verb|degrees|, \verb|sqrt|
and constants like \verb|e|, \verb|pi|

Can access them with dot notation: \verb|module.function|

\verb|math.log(y)| is \emph{natural} log (base $e$). If you want
e.g. base 2, do \verb|math.log(10, 2)|.

Do more examples, e.g. \verb|math.sqrt(math.log(math.pi))|.  Function
composition!

\begin{verbatim}
dist = int(input("What is the distance?"))
\end{verbatim}

On to Lab 3.  You will write scripts like with Drake equation, use
this as a template for your work.

Python Style Check.  Download this and use it.

\newpage
\section{Logic Puzzles \& Boolean logic}

\subsection*{Setup}
\begin{itemize}
\item \textbf{Quiz (S'16)}: what is the value of each of the following
  Python expressions?
  \begin{itemize}
  \item \verb|3 + 4 * 2 - 1|
  \item \verb|3 ** (11 % 3)|
  \item \verb|7.0/2.0 + float(7/2)|
  \end{itemize}
\end{itemize}

\subsection*{Logic puzzles}

Logic puzzle time!  Hand out. Explain rules for Cidouri: connect each
black dot with a single white dot with a horizontal or vertical line.
Lines cannot pass through dots or cross each other.  Have them work
for 5 minutes, then get responses/explanations.

This is a logic puzzle.  Things are true or false, and they have
implications for the other parts of the puzzle.  Find an end that HAS
to be true, work from there.

True and False are part of python too, and lie at the heart of
computer programming.  Line up well with binary numbers, 1 and 0, True
and False.

Propositions can be true or false.
\begin{itemize}
\item Label the columns with letters, rows with numbers.
\item Is there an edge between circle $A_1$ and $B_1$?  Write as $A_1
  \times B_1$. Is there an edge between $A_1$ and $A_3$? Write as $A_1
  \times A_3$.
\end{itemize}
A statement can be made about the puzzle in general, combining these
with logical operators "and", "or", "not".  Use parenthesis again, like
with math:

\newcommand{\orr}{\mathbin{\mathbf{or}}}
\newcommand{\andd}{\mathbin{\mathbf{and}}}
\newcommand{\nott}{\mathbin{\mathbf{not}}}

\[ (A_1 \times B_1 \orr A_1 \times A_3) \andd \nott (A_1 \times B_1 \andd A_1
\times A_3) \]

We write many of these, and we could have a program that verifies if a
solution is a good one or not.

Same sort of thing with other puzzles.

Make sure to leave time to explain all puzzles.  HW2 (due Monday) is
to complete all puzzles.

\newpage

\section{Conditionals}

\subsection*{Setup}

\begin{itemize}
\item{} [S '16: Next time: Quiz on chapter 4.]
\item{} [F'15: logic puzzle HW due today]
\item{} [F'15: quiz 2 today]
\end{itemize}

\subsection*{Booleans}

Today: a new Python type!

We saw Boolean values {\tt True} and {\tt False}, in the puzzles, with
operators {\tt and}, {\tt or}, {\tt not} last time.  Show these in
Python, at prompt.  Note semantics of {\tt or}.

How else can we get Booleans?

\textbf{Comparison operators}: \verb|>|, \verb|<|, \verb|>=|,
\verb|<=|, \verb|!=|, \verb|==|.  Demonstrate on numbers, strings, all
generate Booleans.

\subsection*{Conditionals}

You already know about conditionals.  Here's how we do them in Python.
The computer has a basis to make decisions.  Sets up branches in the
code, execute this or that.

\begin{verbatim}
passwd = input("What is the password? ")
if (passwd == "lemur"):
    print "Here be secrets."
\end{verbatim}

Tabbing is important!  Colon is important!

Introduce random module. ({\tt import random})
\begin{itemize}
\item {\tt random.random()} is number in $[0,1)$. Uniform.
\item {\tt random.randint(n)} is random integer between $0$ and $n$
  (inclusive).  (Aside: how can we write this in terms of {\tt
    random.random()}?)
\end{itemize}
Do coin flip.  Conditionally do something.  Needs to include
\begin{itemize}
\item {\tt else}
\item {\tt elif}
\item nested {\tt if}  (flip two coins in a row)
\end{itemize}

Start with multiply nested {\tt if} and then show them {\tt elif} for
nicer syntax.

For example:
\begin{verbatim}
import random

passwd = raw_input("Enter the password: ")
r = random.random()
print "The random number is", r

if (r < 0.2):
    print "You got heads!"
elif (r >= 0.2 and r < 0.6):
    print "You got tails!"
else:
    print "You got pickles!"
    if (passwd == "pass"):
        print "Here be secrets"
    else:
        print "No secrets for you"
\end{verbatim}

\begin{reflect}{Brent, S' 16}
  This went really fast---I let them go with 13 minutes remaining in
  the period!  Maybe add some interactive components next time?
\end{reflect}


\newpage

\section{Information encoding I}

\subsection*{Setup/administrivia}

\begin{itemize}
\item PUT OFF quiz until Wednesday (after conditional lab).  Hint:
  about finding errors.
\item Collect logic puzzle HW
\item NB: exam 1 a week from today!  In class, closed notes, closed
  computer.  Covers material through today and next class. Class
  before exam will be review.
\end{itemize}

\subsection*{Ada Lovelace}

How many of you have heard of Ada Lovelace?  How about Charles
Babbage?

Ada Lovelace: 1815 (December 10!)--1852, first programmer.  Wrote
programs for Charles Babbage's Analytical Engine (never built, but
definitely works in principle).  Babbage's purpose for the AE was
limited to making tables of numbers, but Lovelace had a much more
expansive and far-seeing vision:

\begin{quote}
``[The Analytical Engine] might act upon other things besides number,
were objects found whose mutual fundamental relations could be
expressed by those of the abstract science of operations, and which
should be also susceptible of adaptations to the action of the
operating notation and mechanism of the engine...

Supposing, for instance, that the fundamental relations of pitched
sounds in the science of harmony and of musical composition were
susceptible of such expression and adaptations, the engine might
compose elaborate and scientific pieces of music of any degree of
complexity or extent.''
\end{quote}

She saw potential for computers to operate on many kinds of
information instead of just making tables of numbers, and we know that
they can indeed do just that!  In fact, she even missed one very
important development, that computers can not only \emph{operate} on
these kinds of information, but allow us to \emph{communicate} using
these kinds of information as well.  Over the next 2 classes we will
consider some of the fundamentals that make this possible.

\subsection*{Binary}

Computers represent numbers using \emph{binary} = base 2 instead of
base 10.  Recall how base 10 works: ones ($=10^0$) place, tens
($=10^1$) place, hundreds ($=10^2$) place, \dots  And ten different
digits.  For example \[ 4397_{10} = 4 \times 10^3 + \dots \] Base 2 is
the same but
\begin{itemize}
\item We use $2$ instead of $10$ (ones place, 2's place, 4's place,
  8's place\dots)
\item We use $2$ digits $(0,1)$ instead of $10$.  Called \emph{bits} =
  \emph{bi}nary dig\emph{its}.
\end{itemize}
Computers do everything in binary since (1) it is no less expressive
than e.g. decimal and (2) from a physical point of view, it is easier
to design hardware that can distinguish two different states
(e.g. high and low voltage) than 10.

Converting base $2$ to base $10$: have them do some
examples. e.g. $10110100_2 = ?$  Then do one example together on the
board, \[ 10110100_2 = 1 \times 2^7 + 1 \times 2^5 + \dots \]

Note 1 \emph{byte} = 8 bits.  Incidentally, one \emph{kilobyte} is not
$1000$ bytes, but actually $2^{10} = 1024$ bytes!  Megabyte is
$2^{20}$ bytes, and so on.

Questions for students (pair \& share etc, using binary cards!):
\begin{enumerate}
\item How many different binary numbers using $n$ bits are there?
  (Write out 16 binary numbers using $4$ bits.)
\item What is the biggest number that can be represented using $n$
  bits?
\item How many bits are required to count up to $n$?
\end{enumerate}

Algorithm for converting from $n$ in base $10$ to binary:
\begin{itemize}
\item Find largest power of two $\leq n$, say, $2^k$.
\item As long as $n > 0$:
  \begin{itemize}
  \item If $2^k \leq n$
    \begin{itemize}
    \item subtract $2^k$ from $n$
    \item write a $1$ (in the $2^k$ place)
    \end{itemize}
  \item else
    \begin{itemize}
    \item  write a $0$ in the $2^k$ place
    \end{itemize}
  \item Decrease $k$ by $1$.
  \end{itemize}
\end{itemize}

(This has all five usual aspects of algorithms: input, output, math,
conditionals, repetition.  We can't quite write this algorithm in
Python yet because we don't know how to do repetition.  Soon!)  Do an
example, e.g. $103_{10}$.

\begin{reflect}{Brent, S'16}
  It seems that in different semesters I end up at a different place
  here. \emph{e.g.} in S'16 I had planned to talk about representation
  of integers on this day, and pick up with floating-point in the next
  lecture, but I only made it up through converting binary. So I've moved
  representation of integers to the next lecture---it actually makes
  more sense there anyway.
\end{reflect}

\section*{LAB: Big Data/Heart Disease}

\begin{reflect}{Brent, S'16}
  \begin{itemize}
  \item Spend a few minutes at the beginning explaining the graphs,
    especially the Boolean ones.
  \item Make sure they don't re-save the data files from Excel---it
    messes up the formatting!  Opening in Excel to \emph{look at} the
    data files is OK, just don't save.
  \item Try to guide them to using nested ifs.
  \item Make sure they understand why it's ineffective to make a huge
    condition with a bunch of ANDs.
  \end{itemize}

  Other than those gotchas, lab seems to go OK.
\end{reflect}

\section{Information encoding II}

\subsection*{Setup}
\begin{itemize}
\item Quiz: program with conditionals, ask them to identify five
  errors.
\end{itemize}

\subsection*{Representing integers}

Integers are represented like this inside the computer.  Show entering
binary numbers directly into Python using \verb|0b10110| notation.

Most modern computers use $64$ bits to represent an integer (\emph{how
  big is that?}); some use $32$. Show Python switching from \verb|int|
to \verb|long|:
\begin{itemize}
\item Some small and very big examples
\item \verb|2**62|
\item \verb|2**63|
\item Apply \verb|type| to each of the above.
\end{itemize}

What about negative numbers?  One possibility: use one bit for sign
(plus or minus).  (Turns out there is a better way, ``$2$'s
complement''; learn about it in CSO or ask if you are curious.)

\subsection*{Representing floating-point numbers}

Back to base $10$: what does $123.45$ mean?
\[ \dots + 4 \times 10^{-1} + 5 \times 10^{-2} \]
We can do binary ``decimals'' (``binarals''?) the same way: \[
1101.011 = 1 \times 2^3 + \dots + 1 \times 2^{-1} + 1 \times 2^{-2} =
13 \frac{3}{8} \]  Also recall ``scientific notation'', e.g. $1.23
\times 10^{17}$.  This is how ``floating point'' numbers are
represented: scientific notation, but in base $2$.  For example, with
$64$ bits:
\begin{itemize}
\item 1 bit for sign ($\pm$)
\item $11$ bits for exponent (base $2$ integer)
\item $52$ bits for value
\end{itemize}
e.g. $-1011011 \times 2^{-3}$.

How would you represent $0.1_{10}$?  Can't represent exactly using
base $2$ (infinite)! Show $0.1 + 0.1 + 0.1$ at python prompt.

\subsection*{Representing text}

Basic idea: use a different number to represent each letter.  ASCII
(American Std. Code for Info. Exch.) --- early 1960's.  Specified
$128$ different characters, each using $7$ bits.  (In many cases 1 bit
left over for parity checking or just set to $0$.)
\begin{itemize}
\item Show ASCII chart.
\item Illustrate chr and ord functions in python.
\end{itemize}
$128$ characters may have been enough for the white, American,
English-speaking men who made it up.  But it sure isn't any more.
Unicode---currently over 120,000 characters. (How many bits needed?
Often uses a more complex scheme to allow different numbers of bits,
\& extending indefinitely without changing existing.  Ask if you're
curious.)

\subsection*{Representing images}

Image = grid of colored points (``pixels'' = \emph{pic}ture
\emph{el}ements).  Each pixel = mix of red, green, blue (additive
primary colors).  Each primary color has 256 possible intensities,
from 0 (off) to 255 (as bright as possible).  So each primary = 8 bits
(1 byte), each pixel = 24 bits.  How many bits/bytes for a $500 \times
500$ image?

\subsection*{Hexadecimal}

Base $16$.  Need 16 symbols: 0--9, a--f. So $a_{16} = 10_{10}$ and so
on. $\mbox{ace}_{16} = ?$

Note, we can group the binary digits into 4s.  $16 = 2^4$.  So each 4
bits corresponds to 1 hexadecimal digit.  Conversion back and forth is
super easy.  We often use hexadecimal as a more convenient way to read
and write binary.  Easier for humans to read and remember.

Show entering hexadecimal directly into Python using \verb|0xace|
notation.

Show example of RGB colors expressed in hexadecimal: show
\verb|style.css| from course website.

\section{Exam 1 review}

\subsection*{Setup}

\begin{itemize}
\item Mention Project 1 coming up!
\end{itemize}

\subsection*{Sample exam}

Have them do sample exam. Go over it.  Things to emphasize:
\begin{itemize}
\item ``Semantic errors'': looking for explanation of how/why the
  program works.
\item Don't just print out a number, print it with a nice message!
\item Ways to organize information in finding bugs
\end{itemize}

\section{Exam 1}

\newpage

\section{Functions I}

\subsection*{Setup}
\begin{itemize}
\item Read chapter 5 in book --- quiz Friday.
\end{itemize}

\subsection*{Exam 1 review}
\begin{itemize}
\item Hand back exams.
\item Go over some solutions:
  \begin{itemize}
  \item Write exemplar solution for programming question
  \item Go over semantic errors question
  \item Come ask me with questions about other problems.
  \end{itemize}
\end{itemize}

\subsection*{Project overview}

Show them the project.  Calendar strangeness due to tracking floating
point value (ratio of revolution to rotation) using integers!  Run an
exemplar and show them what it does.

\subsection*{Functions}

\textbf{Read Chapter 5! Note quiz on Wednesday!}

We have used functions in Python already, like \verb|raw_input|,
\verb|cos|?, other math things?

We can define our own functions!  Same rules for names as
variables. Header line: parens mean no inputs, colon, indent.  Stuff
inside the indent (body) says what happens.

\begin{verbatim}
def say_hi():
    print "Hello there, CSCI 150!"
    print "This is a function."
\end{verbatim}

Function definition makes a variable which is a function object. (Note
hex value in output! =) To call it, use parentheses.  \textbf{Every
  function must always have parentheses after it.}

Functions can be used inside other functions.

\begin{verbatim}
def say_hi_twice():
    say_hi()
    say_hi()
\end{verbatim}

Put things inside a script.  Note the definition itself does not cause
body to be executed.  Note functions must be defined before they are
used.

\subsection*{Why functions?}

Get them to discuss why this is a worthwhile feature.  Why define
functions?  Pair \& share.

From Downey:

\begin{itemize}
\item Creating a new function gives you an opportunity to name a group
  of statements, which makes your program easier to read and debug.
\item  Functions can make a program smaller by eliminating repetitive code.
  Later, if you make a change, you only have to make it in one place.
\item Dividing a long program into
  functions allows you to debug the parts one at a time and then
  assemble them into a working whole.
\item Well-designed functions are often useful for many programs.
  Once you write and debug one, you can reuse it.
\end{itemize}

\subsection*{Parameters and arguments}

Make C to F conversion into function, takes C as argument:
\begin{verbatim}
def C_to_F(temp_C):
    F = C * 9.0 / 5.0 + 32
    print "The temperature in degrees Fahrenheit is ", F
\end{verbatim}

\verb|temp_C| is a variable called a \emph{parameter}.

The values given to functions are called \emph{arguments}.  To call a
function, give one argument for each parameter.  The parameter will
stand for the argument.  Do some examples.  Note you can put arbitrary
expressions for arguments.

Variables created inside functions are local!  Parameters are also
local. Note that \verb|F| and \verb|temp_C| are not defined afterwards.

\subsection*{Reflections}

\begin{reflect}{Brent, S'16}
  They did so well on the exam (about 80\% of the class scored over
  90) that I judged it a waste of class time to go over exam
  solutions.  We ended up making it a bit farther and introducing
  fruitful functions as well.
\end{reflect}

\newpage
\section{Functions II}

\subsection*{Setup/administrivia}
\begin{itemize}
\item Reminder: project 1 due soon
\end{itemize}

\subsection*{Quiz: arguments \& parameters}

\subsection*{Fruitful functions}

Recall function from last time:
\begin{verbatim}
def C_to_F(temp_C):
    F = C * 9.0 / 5.0 + 32
    print "The temperature in degrees Fahrenheit is ", F
\end{verbatim}
This function is actually really awkward.  Takes \verb|temp_C| as a
parameter, but then \emph{prints} the result.  What if we want to
compute this but not print anything?  Make each function do ONE job.
Better:
\begin{verbatim}
def C_to_F(temp_C):
    F = C * 9.0 / 5.0 + 32
    return F
\end{verbatim}
Book calls it a \emph{fruitful function}.  (As a mathematician I would
just call it a \emph{function}\dots) Note we could also get rid of the
\verb|F| and just return the whole expression; matter of
taste/experience.

\verb|return| means ``stop execution of the function right now and
return this value''.  What will this do?

\begin{verbatim}
def C_to_F(temp_C):
    F = C * 9.0 / 5.0 + 32
    return F
    print "The temperature is", F
\end{verbatim}

OK to have multiple return statements.  Let's try a function to tell
us about the temperature (\emph{do something like this, students can
  come up with details}):
\begin{verbatim}
def F_to_qualitative(temp_F):
    if temp_F < 32:
        return "freezing"
    elif temp_F < 50:
        return "cold"
    elif temp_F < 75:
        return "warm"
    elif temp_F < 90:
        return "hot"
    else:
        return "scorching"
\end{verbatim}

Now have the students work in pairs to write a function
\verb|discuss_weather()| which prompts for temperature in C, converts
to F, and makes an appropriate comment.

\begin{verbatim}
def discuss_weather():
    temp_C = int(raw_input("What is the temperature in degrees C? "))
    temp_F = C_to_F(temp_C)
    word = F_to_qualitative(temp_F)
    print "It's very " + word + " today, don't you think?"
\end{verbatim}

Then go through series of variants showing how you can nest function
calls, culminating in

\begin{verbatim}
# DO NOT TRY THIS AT HOME
def discuss_weather_4():
    print "It's very " +\
     F_to_qualitative(C_to_F(int(raw_input("What is the temperature in\
     degrees C? ")))) +\
     " today, don't you think?"
\end{verbatim}

Functions can take multiple parameters.

\begin{verbatim}
def is_divisible(x, y):
    if (x % y == 0):
        return True
    else:
        return False
\end{verbatim}

Try some examples.  Note this is actually written in a redundant way.  We
can just return \verb|x % y == 0| directly.

\subsection*{Reflections}

\begin{reflect}{Brent, S'16}
  This class went well, we had a lot of fun.  Didn't make it to stack
  diagrams (below) but that's OK.  Will probably cover this when
  talking about recursion instead?
\end{reflect}

\subsection*{Flow of execution \& stack diagrams}

Execution starts at the first line of a program and continues.  But
when a function is called, execution jumps to that function, and when
done picks back up where it left off.  This can get complicated, since
functions can call other functions!  In general Python has to keep
track of a ``stack'' of places to resume.  Imagine you were reading a
book but it told you to go read another book and then come back.  Then
THAT book told you to go read another book, etc.  You would keep the
old books, with bookmarks, in a pile.  When done, pick up the next
book from the top of the pile and continue.

Do an example. Write a function that calls \verb|C_to_F| on several
temperatures.  Then write another function which does that twice, and
so on.  Then trace execution.  Introduce an error and see the
``traceback''.  Draw a stack diagram with local variables.

\section*{LAB: Coding bat}

\begin{reflect}{Brent, S'16}
  This seemed to go well, and seemed helpful.  I always think students
  are going to finish these really quickly but it actually seems like
  just the right amount.  One or two actually finished
  \verb|make_bricks|.  Several were working on it by the end.  Others
  were at various points.  Only one or two were really struggling.
\end{reflect}

\noindent
\begin{reflect}{Brent, S'16}
  \textbf{(CRAZY?) IDEA}: Maybe we should switch the order of loops
  and functions.  The basic outline would be to replace the Coding bat
  lab with a version of Doublets where they \emph{don't} use any
  functions.  Then, we learn about functions and give them another lab
  where they get to practice taking existing top-level code and
  abstracting it into functions.  We probably can't have them actually
  work on Doublets because they might not be able to use their own
  code and we don't want to distribute a solution.  So instead we
  could give them a similar sort of word-game program, written as all
  top-level code, and have a lab which walks them through abstracting
  out more and more functions until the whole thing is composed of
  functions.  I'm imagining something where we provide a lot of
  hand-holding with the first few functions they extract, and then
  give them more and more autonomy as the lab proceeds.

  What's the motivation? Students can get the basic concept of writing
  a function that takes inputs and gives an output (with Coding Bat),
  but this does not at all mean that they understand how a function is
  used to abstract out some common pattern, and how information needs
  to flow between functions (into parameters and back out through the
  return value).  Telling them ``do this complicated lab'' and ``you
  have to use only functions'' actually gets them off on the wrong
  foot: they \emph{start} the lab by trying to write functions (which
  often do inconsistent or nonsensical things) rather than starting by
  writing working code and then thinking about how to abstract it into
  functions.
\end{reflect}

\newpage
\section{While loops}

\subsection*{Setup}
\begin{itemize}
\item Reminder: Project 1 due soon! [In S'16 it was due the day
  \emph{before} this lecture, due to winter break.]
\end{itemize}

\subsection*{Guess my number game}

\begin{itemize}
\item First, play the game with class.
\item Try writing a program to play it (human guesses). Have to
  replicate code\dots?
\end{itemize}

\subsection*{While loops}

Introduce while loops---this is a momentous day!  Once we add while
loops you can (in theory) write \emph{any program you can imagine}.
Explain syntax.  Syntax and semantics are very much like \verb|if|
(without \verb|else|): difference is that \verb|if| executes 0 or 1
time, \verb|while| executes any number of times (0 or more).

Do an example: count from 1 to 10.  Play with boundary conditions,
order of statements inside the loop.

Point out three usual components:
\begin{itemize}
\item Declare and initialize variable(s) you need
\item Write the condition with those variables
\item Change those variables somewhere in the loop (often at the
  end)
\end{itemize}

Do another example: keep printing ``hi'' and asking ``Do you want to
stop?'' until says yes --- have students write this code together on
their computers/paper.  Then have two or three share code?  There are
two ways to write this --- using input variable to control loop (but
then you have to initialize it with a bogus value), or using a boolean
(sentinel).

Back to Guess My Number.  Introduce a loop.  Make hi/lo checking into
a function.  Promise: after we learn a bit more about strings we can
make another function to get valid int input from user in a loop.

\textbf{Homework}: write version where computer guesses!

\begin{reflect}{Brent, S'16}
  I don't really like the above homework, I think it's actually too
  hard for most of them at this point, and they already have enough
  assignments where they get practice coding.  I just got them to
  think briefly about how one might implement it.  One student took me
  up on it and implemented a very nice version.
\end{reflect}

\newpage
\section{Strings}

\subsection*{Setup}
\begin{itemize}
\item Mention HW: DNA
\end{itemize}

\subsection*{Guess My Number review}

Open code from last time.  Talk about modifying it for other version.

Remember:
\begin{itemize}
\item Variables are local!
\item No recursion, just loops.  Talk about recursion later.
\end{itemize}

\subsection*{Strings}

We sort of know about strings, but the only thing we can do with them
so far is concatenate them with \verb|+|.

\begin{itemize}
\item Strings are \emph{sequences of characters}.

\item We can reference individual characters in a string by their
  \emph{index}: \verb|mystring[3]|.  Note that \emph{indices start at 0}.

\item We can calculate the length of a string with \verb|len()|.
  (Relationship of \verb|len| to valid indices?)

\item String slices: \verb|mystr[a:b]|.  Have students play around to
  figure out details?
  \begin{itemize}
  \item Characters starting at index $a$, up to \emph{but not
      including} index $b$.
  \item Can leave off first, or second, or both.
  \item Negative indices count backwards from the end.
  \end{itemize}

\item We've seen \verb|+|: concatenation.
\item \verb|*| does repetition.

\item Strings have \emph{methods}, i.e. functions you can call on
  strings.  Do \verb|help(str)|, \verb|help(str.upper)| etc.
\item Show some useful methods: \verb|upper|, \verb|count|,
  \verb|replace|, \verb|find|, \verb|isdigit|
\end{itemize}

\subsection*{More while loop \& function practice}

First, function to ``explode'' a string.  Write it collaboratively.

\begin{verbatim}
def explode(s):
    i = 0
    while i < len(s):
        print s[i]
        i += 1
\end{verbatim}

Even better if they make some bugs in the above, run it and see what
is wrong.

\subsection*{DNA HW intro}

Show them HW, do a bit of it together to show them what they should
do.

\newpage

\section{While loop \& string practice}

\subsection*{Setup}
\begin{itemize}
\item Reminder: DNA Strings HW due Wednesday before class.
\item Reminder: quiz next class on strings.
\item Lab today/tomorrow.  Not easy!  Adding loops takes what we can
  do up a few notches.  Please take the pre-coding design component
  seriously.  It is already up --- take a look at it and start early,
  if you have a chance.
\end{itemize}

\begin{reflect}{Brent, S'16}
  I hadn't been able to fit in the bit about call stacks and
  tracebacks before, so I stuck it in here.  That meant I didn't get
  to the Collatz example at the end.
\end{reflect}

\subsection*{More while loop practice}

Start writing comments before every function.  From now on they should
always write comments for each function!  Comment should have three
things:
\begin{enumerate}
\item Describe inputs
\item Describe output (if any)
\item Describe function behavior
\end{enumerate}

\begin{reflect}{Brent, S'16}
  Potentially, this could be done using docstrings instead of
  comments. Then the function comments will also show up in
  \verb|help| and so on. I (Brent) don't think this is worth the
  trouble: docstrings are Python-specific and have weird rules for
  getting rid of whitespace that other triple-quoted strings don't
  have; and having the description show up in \verb|help(...)| is only
  motivating when working with multiple-module projects (especially
  with code written by others).  In my opinion, it's better
  to just emphasize good old commenting which will serve them well in
  any language and doesn't require introducing any new syntax.
\end{reflect}


\begin{verbatim}
# Inputs:
#   - prompt: string to be used as a prompt
# Output: none
#
# Repeatedly prompt the user for a number using the
# given prompt, until they enter a valid integer.
def int_input(prompt):
    valid = False
    while (not valid):
        uinput = raw_input(prompt)
        if uinput.isdigit():
            valid = True
        else:
            print "That's not a valid integer.  Try again!"
    return int(uinput)
\end{verbatim}

Do version of \verb|int_input| that accepts negative numbers as well
(uses indexing, slicing, \verb|or|).

\begin{verbatim}
import time

# Inputs:
#   - n: int to count down from
# Output: none
#
# Print a countdown from the starting number to zero.
def countdown(n):
    while n > 0:
        print str(n) + "..."
        time.sleep(1)
        n = n - 1
    print "BLASTOFF!"
\end{verbatim}

Talk about \verb|n -= 1| syntax (``decrement'') and \verb|n += 1|
(``increment'').  Can also use \verb|*=|, \verb|/=| etc.

\begin{reflect}{Brent, S'16}
  Got to this point with only 5 minutes left in class.  So I just
  spent the five minutes telling them what the code I had in mind was
  supposed to do, and told them they could try implementing it as
  extra practice if they wanted.
\end{reflect}

Collatz Conjecture.  Idea: write a program to test the conjecture.
What functions will we need?  Bottom-up approach.  Write each function
and then test it independently.
\begin{verbatim}
# Inputs:
#   - n: int
# Output: int, one step of hailstone function on n
#
# Compute the hailstone function.
def hailstone(n):
    if n % 2 == 0:
        return (n/2)
    else:
        return (3*n + 1)

# Inputs:
#   - n: int
# Output: int, number of hailstone steps from n to reach 1
#
# Count the length of the hailstone sequence from the starting
# number to 1.
def collatz(n):
    count = 0
    while n != 1:
        n = hailstone(n)
        count += 1
    return count

# Inputs:
#   - max: int, biggest number to test
# Output: none
#
# Print numbers from 1 to max along with length of their Collatz chain
def check_collatz_up_to(max):
    n = 1
    while n <= max:
        print n, collatz(n)
        n += 1
\end{verbatim}

(What happens if we take out \verb|max| check and just say
\verb|while true|?  A legitimate use of an infinite loop!)

\section*{LAB: Mutation is the Word (Doublets)}

\begin{reflect}{Brent S'16}
  This lab is a lot harder than the previous labs.  Definitely worth
  warning them to start early, emphasizing that they probably won't
  finish within the alloted lab time, can come and ask questions, etc.

  One of the biggest confusions typically has to do with managing
  information flow.  That is, now that we are using functions for
  everything, variables are local, so need to pass things as
  parameters and return values back.  I keep thinking about whether
  there is a gentler way to introduce this but I'm not sure.  If we
  tried to talk about ``managing information flow'' explicitly it
  would just go over their heads until they had actually grappled with
  it in a real program.  So perhaps the shock is unavoidable.

  However, see \textbf{(CRAZY?) IDEA} in reflections above, after
  coding bat lab.  Maybe switch loops and functions, so the doublets
  lab is just writing top-level code with a bunch of loops---really
  focusing on just the loops.  Then the next lab will get them to
  think about abstracting functions from existing code.
\end{reflect}

\newpage
\section{Lists I}

\subsection*{Setup}
\begin{itemize}
\item Quiz (strings)
\item Distribute binary cards.  Learn the name of 0-bit pairing.
\item REMINDER: you have late days you can spend on labs!
\end{itemize}

\subsection*{Quiz 5 (Strings)}
\begin{verbatim}
t = "PLEASEREADME"
r = "astory"
p = t[-2:] * 2
r = r[(19 % 5) - 1:]
z = p[:3] + r
\end{verbatim}

\subsection*{Lab (Doublets) review/hints}

1-bit pairing with binary cards and discuss lab.  Questions?

\subsection*{Lists}

Lists are a lot like strings, but hold arbitrary data instead of
characters.
\begin{itemize}
\item \verb|[]| is the empty list, like \verb|""| is the empty string.
\item Can hold numbers, strings, anything you want.  Show
  \verb|[1,2,3]| syntax.  Show \verb|type| on lists.
\item Elements are ordered, in a sequence.  Indexed from zero.
\item Show \verb|range| function.
\item \verb|+|, \verb|*|, \verb|len()|, slices all work the same
\end{itemize}

\subsection*{Mutating lists}

Strings are \emph{immutable}, lists are \emph{mutable}.

\begin{verbatim}
animals = ['cat', 'dog', 'bird']
animals[1] = 'lemur'
print animals
\end{verbatim}
Draw a stack diagram of the above situation.

Other list mutation methods:
\begin{itemize}
\item \verb|t.append(element)| for adding to end
\item Removal:
  \begin{itemize}
  \item \verb|t.pop(index)| by index, returns deleted elt
  \item \verb|t.remove(element)| by element
  \item \verb|del t[index]| by index, doesn't return anything BUT
	\verb|del t[slice]| works too
  \end{itemize}
\end{itemize}

\begin{reflect}{Brent, S'16}
  Got through this lecture way faster than last time, leaving a good
  10--15 minutes at the end.  So on the fly I came up with another
  exercise as an example of while loops and building a list.  We wrote
  the following code all together.
\end{reflect}

\begin{verbatim}
# Inputs: none
# Output: a list of strings entered by the user
#
# Repeatedly prompt user for inputs until they type
# 'quit', then return a list of all inputs.
def read_inputs():
    inputs = []
    done = False
    while not done:
        word = raw_input('Enter another input, or quit: ')
        if word.lower() == 'quit':
            done = True
        else:
            inputs.append(word)

    return inputs
\end{verbatim}

\newpage
\section{Lists II}

\subsection*{Setup}
\begin{itemize}
\item Project 1 is graded.  Let me know of any questions.  Let me know
  if you want more feedback.
\item Announce Exam 2.
\end{itemize}

\subsection*{Review}
Find partner. \verb|animals| is a list, how to append 3 copies of
\verb|"gerbil"| onto the end?  Go over possibilities.  This might take
a while.  Try to get a bunch of different approaches from different
groups.  Talk about \verb|append|, list repetition operator \verb|*|,
concatenation operator \verb|+|, etc.

\begin{reflect}{Brent, S'16}
  As an illustration, here are all the suggestions my students came up
  with:
\begin{verbatim}
animals = ['skua', 'killer whale', 'spider', 'orangutan', 'goat']

# Ways to add 3 gerbils to the end of the list

# Doesn't quite work:
# animals.append('gerbil ' * 3)

# Runtime error:
# animals.append('gerbil') * 3

# Works!
# while len(animals) < 8:
#    animals.append('gerbil')

# Runtime error (too many arguments to append):
# animals.append('gerbil', 'gerbil', 'gerbil')

# Works!
# animals.append('gerbil')
# animals.append('gerbil')
# animals.append('gerbil')

# Same as 'gerbil' * 3
# animals.append('gerbil' + 'gerbil' + 'gerbil')

# Doesn't change animals
# animals + ['gerbil', 'gerbil', 'gerbil']

# Works!
# animals = animals + ['gerbil', 'gerbil', 'gerbil']

# Works!
# gerbil_list = ['gerbil']
# animals = animals + gerbil_list * 3

# Almost works but not quite.
# animals.append(['gerbil'] * 3)
\end{verbatim}

  I think this is a really great exercise. It really helps develop
  their intuition and their grasp of the details.
\end{reflect}

\subsection*{Lists and strings}

Lists of characters are different than strings!

\begin{itemize}
\item Converting strings to list:
  \begin{itemize}
  \item \verb|list| function.
  \item \verb|split()| method.
\begin{verbatim}
  s = "banana"
  t = s.split("n")
\end{verbatim}
Yields \verb|["ba", "a", "a"]|.

  \end{itemize}
\item List to string: \verb|join()| method on delimiter.
\begin{verbatim}
  t = ['fun', 'sun', 'sub', 'rub']
  delimiter = " -> "
  delimiter.join(t)
\end{verbatim}
\end{itemize}

\subsection*{Mutability, objects, and aliasing}

Warm-up: what happens here?  Find partner \& decide.
\begin{verbatim}
nums  = [1,2,3]
nums2 = nums
del nums[1]
print nums2
\end{verbatim}

Try it at IDLE prompt.  This is really weird.

As hinted last time, the fact that lists are mutable opens a huge can
of worms.  Let's start by pulling back the curtain a little bit.  Up
until now we have talked about variables having values, but we need to
be more precise.  Draw stack diagram with variables referencing two
different lists.  Write these definitions on the board:

\begin{itemize}
\item Variables reference \emph{objects} which have a \emph{value}.
\item Variables which reference objects that have the same value are
  \emph{equal}.  You can test equality with \verb|==|.
\item Variables which reference the \emph{same object} are
  \emph{identical}. You can test identity with \verb|is| (though you
  won't need this in the rest of the course).
\end{itemize}

\begin{reflect}{Brent, S'16}
  Only made it to here---exercise with adding gerbils took a while,
  but I think it was worth it.  Plan to quickly finish up the rest of
  this lecture at the beginning of next class before starting on for
  loops.  I can't take too long, though, since they need for loops for
  their lab on the next day.
\end{reflect}

Do this at the REPL, with accompanying pictures:
\begin{verbatim}
>>> a = [1,2,3]
>>> b = [1,2,3]
>>> a == b
True
>>> a is b
False
>>> del a[0]
>>> a
[2, 3]
>>> b
[1, 2, 3]
>>> a == b
False
>>> a = b
>>> a
[1, 2, 3]
>>> b
[1, 2, 3]
>>> a == b
True
>>> a is b
True
>>> del a[0]
>>> a
[2, 3]
>>> b
[2, 3]
\end{verbatim}
Note in particular that the assignment \verb|a = b| does not copy the
value of \verb|b| into \verb|a|, but just makes \verb|a|
\emph{reference the same thing as} \verb|b|.

This has implications for function parameters as well.  Consider:

\begin{verbatim}
def animal_release(animals):
  while len(animals) > 2:
  animals.pop(0)

print animals
animal_release(animals)
print animals
\end{verbatim}

Ask: what gets printed?  Draw stack diagram to understand.

Some related things:
\begin{itemize}
\item What if you really \emph{do} want to make a \emph{copy} of a
  list, so modifying it doesn't modify the original?  You can write
  \verb|some_list[:]|.  This is a dirty, terrible, disgusting hack.
  Perhaps someday you will understand why I feel this way.
\item Be careful about methods that modify vs functions that make a
  new thing.  e.g. \verb|list.sort()| modifies the list.  But
  \verb|sorted(list)| makes a new list!  And of course
  \verb|string.upper()| makes a new string because strings are
  immutable.  Confusing, don't memorize: the point is you should be
  careful and look up help to see what things do.  Show
  \verb|help(list.sort)|, \verb|help(sorted)|, \verb|help(str.upper)|.
\end{itemize}

[INSERT OPTIONAL RANT ABOUT MUTATION HERE]

\newpage
\section{For Loops}

\subsection*{Setup}

\subsection*{For Loops}

Recall our \verb|explode| function from a week or two ago:
\begin{verbatim}
def explode(s):
  index = 0
  while index < len(s):
    print s[index]
\end{verbatim}
Recall how easy it was to make errors like starting index at the wrong
place, using \verb|<=| instead of \verb|<|, and so on.  Well, it turns
out there is a better way!
\begin{verbatim}
def explode(s):
  for c in s:
    print c
\end{verbatim}
Simply loops through each character in the string one at a time.  Each
time through the loop the next character is assigned to \verb|c|.
This also works for lists:
\begin{verbatim}
animals = ["cow", "chicken", "pig", "rabbit"]
for animal in animals:
  print "And on this farm he had a " + animal + "!"
\end{verbatim}
What do we lose, as compared to a \verb|while| loop?  The index. But
actually, we don't!  Just have to cleverly put together a few things
we have already seen:
\begin{verbatim}
for i in range(len(numbers)):
  numbers[i] *= 2
\end{verbatim}
This is way better than using a \verb|while| loop since it rules out
errors putting the starting or ending the indices at the wrong number.

\subsection*{For loop practice}

\begin{verbatim}
def print_each(s):
    index = 0
    while index < len(s):
        c = s[index]
        print(c)
        index += 1

def print_each_for(s):
    for c in s:
        print(c)

def count(s, c):
    num = 0
    for c2 in s:
        if c2 == c:
            num += 1
    return num

def find(s, c):
    for i in range(len(s)):
        if s[i] == c:
            return i
    return -1

def find_all(s, c):
    all_the_things = []
    for i in range(len(s)):
        if s[i] == c:
            all_the_things.append(i)
    return all_the_things

def nos(s):
    news = ""
    for c in s:
        if c != "s":
            news += c
    return news

def sum(t):
    total = 0
    for n in t:
        total += n
    return total

def mean(t):
    total = 0
    for n in t:
        total += n
    return float(total) / len(t)

def mean2(t):
    return float(sum(t)) / len(t)

def vacation(s):
    for i in range(len(s) - 1):
        if s[i] == s[i + 1]:
            return True
    return False

def classify(strs):
    for s in strs:
        print ('*' * len(s))

def powersof2(n):
    the_powers = []
    for i in range(n):
        the_powers.append(2 ** n)
    return the_powers

def triangle(n):
    total = 0
    for k in range(1,n+1):
        total += k
    return total
\end{verbatim}

\section*{LAB: Caesar's Secrets}

\section{For loop practice}

\subsection*{Setup}

\subsection*{For loops: reading exercises}

CLOSE COMPUTERS.
Find bit-partners. Different one for each exercise.
Go over each and try it.

\begin{verbatim}
def oogie(m):
    p = m[0]
    for g in m:
        if g > p:
            p = g
    return p

def yayaya(q):
    for y in range(len(q) - 1):
        if q[y] == q[y + 1]:
            return True
    return False
\end{verbatim}

Then have them write \verb|is_sorted| function.

Additional practice:

\begin{verbatim}
def pulu(r):
    b = []
    for k in range(r):
        b.append(k ** 3)
    return b
\end{verbatim}

\section{Exam 2 review}

Spend a lecture doing and going over a practice exam.  See
\[ \verb|byorgey/csci150-exams-solutions| \] repository on Bitbucket.

\section{Exam 2}

\begin{reflect}{Brent, S'16}
  In the Fall 2015 semester, Exam 2 was on the Wednesday right before
  Fall Break.  In Spring 2016, it was on a Tuesday two weeks before
  spring break. (The ``normal'' thing would have been to have the exam
  review day on Friday, then the exam on Monday, and no lab on
  Tuesday, but I was at SIGCSE on the Friday, so I gave them the
  practice exam over the weekend while I was at SIGCSE, spent Monday
  reviewing the practice exam, and held the exam during the lab slot
  for that week.)
\end{reflect}

\section{Exam wrap-up, demo project}

\subsection*{Exam wrap-up}

Hand back exams; quickly go over exam solutions.

\subsection*{Demo project}

Introduce Project 2 (word games).

Livecode the beginning of an exemplary Doublets lab.  Things to focus
on:
\begin{itemize}
\item Start with \verb|main|.  Hierarchical decomposition; write stub
  functions.
\item Think about what information each function needs to do its job,
  and what information it returns.
\item Write \verb|get_starting_word| function.
\item Start writing \verb|get_ending_word| function, realize that they
  are going to have a lot of similar code.
\item Abstract common part out into a new function, \verb|get_word|,
  that takes a parameter describing the desired word.
\item Stub out a single round of the game, then talk about adding a
  while loop.
\end{itemize}

\section{Recursion I: intro}

\subsection*{Setup}
\begin{itemize}
\item Write stone-counting rules somewhere (see below)?
\end{itemize}

\subsection*{Stone counting}

Rules:
\begin{enumerate}
\item You may not look inside the bowl.
\item You may only touch one stone.
\item If someone gives you the bowl and asks you how many stones it
  contains, you must eventually give the bowl back and answer
  truthfully.
\item You may give the bowl to someone else and ask them how many
  stones it contains.
\end{enumerate}
Hand the bowl to a student and ask how many stones it contains.

Recursion!  Functions that call themselves.  In a sense you were all
``the same function'' because you were running the same program
(rules).

Factorial function: $n! = 1 \times 2 \times \dots \times (n-1) \times
n$.  Let's implement in Python.
\begin{verbatim}
def fact(n):
    product = 1
    for k in range(1,n+1):
        product *= k
    return product
\end{verbatim}
Now consider $n! = (1 \times 2 \times \dots \times (n-1)) \times n =
(n-1)! \times n$.  Also need to define $0! = 1$.  Can we translate
this directly into Python?
\begin{verbatim}
def fact_rec(n):
    if n == 0:
        return 1
    else:
        result = fact_rec(n-1)
        return result * n
\end{verbatim}
It works!

Show call stack.  Every recursive call has its own copy of local
variables!

Big idea: transform a problem into a slightly simpler version, solve
that simpler problem, do a bit of work to compute an answer to the
original problem.

Typically:
\begin{itemize}
\item Base case(s): simple case where we know the answer without doing
  any work.  Always do this first!
\item Recursive case(s): make a recursive call to solve a slightly
  simpler (i.e. closer to the base case) problem.
\end{itemize}

If time, could do one of the following:
\begin{itemize}
\item Re-code factorial function live, adding a bunch of comments and
  narrating thought process.
\item List sum.
\end{itemize}

% Another example: Fibonacci function.  Show mathematical definition,
% implement in Python.  Show how slow it is, show call tree.  In fact
% recursion is not always the best way to do things.

% If time, brief intro to Python turtle library and Kock snowflake (in
% preparation for fractal lab).

\section{Recursion II: recursive functions}

Start by reinforcing ``base case + recursive case''.  Talk about
``leap of faith'': trust that recursive calls will work without
thinking about how.  Trust your future self.

Up until now all the examples we've done have been math functions.
Let's do a few more ``computer-sciency'' examples processing strings
and lists.  Note, in Python this is not the best way to implement
these functions (though in some languages it is).  Just for
illustration purposes.  If you take 151 you will start to see real,
legitimate uses of recursion, e.g. processing trees.

\begin{itemize}
\item Work in pairs to implement \verb|product|.
\item Have them work in pairs to implement string \verb|reverse|.
\item Have them work in (different?) pairs to implement \verb|is_palindrome|.
\end{itemize}

\section*{LAB: Fractal Recursion}

\newpage

\section{Dictionaries}

\begin{reflect}{Brent, S'16}
  In F'15 we did file I/O first and then dictionaries.  In S'16 it
  worked better to do dictionaries first --- two lectures right before
  spring break, and then on the first day back from spring break cover
  file I/O along with a dictionary review in preparation for the
  dictionary lab on the next day.
\end{reflect}

Recall lists.  Draw an alternate picture of lists as a mapping from
indices to values. For example \verb|['tapir', 'ferret', 'alligator']|
can be thought of as
\begin{verbatim}
0 -> 'tapir'
1 -> 'ferret'
2 -> 'alligator'
\end{verbatim}
$0$, $1$, $2$ are \emph{keys}, animal names are \emph{values}.  A
\emph{mapping} or \emph{association} between keys and values.  A
\emph{dictionary} is the obvious generalization of this to arbitrary
keys instead of just 0, 1, 2 \dots.  For example, suppose we are
keeping these animals in a zoo and we want to keep track of them by
their habitat ID numbers.
\begin{verbatim}
'A23' -> 'tapir'
'B19' -> 'ferret'
'B12' -> 'alligator'
\end{verbatim}
Or we could use numbers as keys.  Important point: keys have to be
immutable.  So we can't use e.g. lists or dictionaries as keys.

Show how to make this dictionary.
\begin{itemize}
\item Write using curly braces and colons.
\item Alternatively, start with empty one and add them.
\end{itemize}

Other things we can do with a dictionary:
\begin{itemize}
\item Get list of all keys with \verb|keys()|
\item Get list of all values with \verb|values()|
\item Check if a key is in the dictionary using \verb|in|
\item Ask for the number of keys with \verb|len|
\item Delete a mapping with \verb|del|
\item Iterate over keys with \verb|for|
\end{itemize}

Show
\url{https://docs.python.org/2.7/library/stdtypes.html#mapping-types-dict}.

Rewrite \verb|frequency_counts(s)| to return a dictionary.  Why this is
better: directly look up by letter value.  No need to use \verb|ord|
or alphabet string etc.

\newpage
\begin{verbatim}
# Input: ...
# Output: ...
def frequency_counts(s):
    freqs = {}
    for c in s:
        if c not in freqs:
            freqs[c] = 0
        freqs[c] += 1

    normalize(freqs)
    return freqs

# Input: ...
# Output: ...
def normalize(d):
    total = 0
    for k in d:
        total += d[k]

    for k in d:
        d[k] /= float(total)
\end{verbatim}

\newpage
\section{Dictionaries practice}

Naive Fibonacci function.  Understand why it is so slow.  Then
use a dictionary to memoize.  Have them write memo code.

Then memoize hailstone function!

\begin{verbatim}

# Fibonacci numbers
# F_0 = 0
# F_1 = 1
# F_n = F_(n-1) + F_(n-2)
#
# 0, 1, 1, 2, 3, 5, 8, 13, 21, 34, 55, ...

# Input: nonnegative int n
# Output: F_n
def fib(n):
    if n <= 1:
        return n
    else:
        return fib(n-1) + fib(n-2)

# This is really slow!  The computer is doing lots of
# repeated, unnecessary, and redundant work.
#
# Idea: get the computer to remember previously
# computed values using a dictionary.
# Key = n, Value = F_n.
#
# (This is called memoization.)

fibs = {0: 0, 1 : 1}

def fib_memo(n):
    if n not in fibs:
        fibs[n] = fib_memo(n-1) + fib_memo(n-2)
    return fibs[n]

# Collatz conjecture

def hailstone(n):
    if n % 2 == 0:
        return n / 2
    else:
        return 3*n + 1

hailstone_lens = {}

# Input: int n
# Output: Number of iterations of 'hailstone' before reaching 1.
def hailstone_length(n):
    if n == 1:
        return 0
    elif n not in hailstone_lens:
        hailstone_lens[n] = 1 + hailstone_length(hailstone(n))

    return hailstone_lens[n]

\end{verbatim}

\newpage

\begin{reflect}{Brent, S'16}
  This is where spring break fell in S'16.
\end{reflect}

\section{File I/O, dictionary review}

\subsection*{Setup}

\begin{itemize}
\item HW for next class: read section I of Zen.
\end{itemize}

Our programs have no state between each time they are called.
Everything is stored in memory during the program, but afterwords,
it's all gone.  This is RAM.  We need hard drive storage to keep it
around.  Save and read things in files.

To write to a file:

\begin{verbatim}
fout = open("myfile.txt", "w")
\end{verbatim}

then to save data in the file, write out as strings:

\begin{verbatim}
fout.write("hello\n")
fout.write("world\n")
fout.close()
\end{verbatim}
Note the newline characters.

To read from a file:

\begin{verbatim}
fin = open("myfile.txt", "r")
\end{verbatim}

Two ways to get stuff out of the file.

\begin{verbatim}
s = fin.read()
\end{verbatim}

which returns a string that includes all the carriage returns, or

\begin{verbatim}
s = ""
for line in fin.readlines():
    s += line.strip()
\end{verbatim}

which strips them out of each line as it is added. \verb|readlines()|
returns a list of strings.

After both, say \verb|fin.close()| to finish accessing the file.

Do some dictionary practice: write a function to read in a file, do
frequency count of words.  Also show them how to split up a line into
words since they will need that in the lab.

\section*{LAB: Sentiment Analysis}

\begin{reflect}{Brent, S'16}
  I wrote this lab fresh in S'16, based on a Nifty Assignment @
  SIGCSE.  It replaced the Gibberish Generator lab from the previous
  semester, which was way too difficult.  Overall I think the new lab
  went over very well.  It seemed to have the right level of
  difficulty and gave the students good practice with dictionaries.
\end{reflect}

\section{Zen Reading discussion}

\subsection*{Setup}

\begin{itemize}
\item HW: system analysis (see below)
\end{itemize}

\subsection*{Discussion}

\begin{itemize}
\item somewhat confusing? yes.
  \begin{itemize}
  \item you picked up in the middle of a journey the author is taking with his son
    and friends across the country on a motorcycle.
  \item along the way he talks about the philosophy of quality, goodness, etc,
    this excerpt is groundwork for that discussion
  \end{itemize}

\item classical vs romantic understanding
  \begin{itemize}
  \item underlying form vs appearance
  \item negative views of each side (dull, uninteresting for
    classical, frivolous, erratic for romantic)
  \item masculine vs feminine association? Sexist?  Not really, he's
    just making an observation about European culture, context means
    these are not determined but cultural associations.
  \end{itemize}
\item How is CSCI related to these understandings? Why did we read
  this?
  \begin{itemize}
  \item classical, definitely but elements of romantic too
  \item creativity of solutions?  choices in problem solving, uniqueness
    of approaches to projects, labs
  \end{itemize}
\item application of classical ``analysis'' to motorcycle, talk through
  some of the details of the diagrams
\item ``impossible to understand unless you already know how one works''
\item What is the ``intellectual scalpel''?
  \begin{itemize}
  \item e.g. knife diving up the motorcycle
  \item no two motorcycle manufacturers divide same way
  \item romantic part of how this choice is made
  \item this skill of using the knife has been what we have been
    practicing all semester
  \end{itemize}
\item back to CSCI connection
  \begin{itemize}
  \item when we have talked about data types, (int, bool, string, list, dict)
    they have been ways we use the knife to separate information
  \item they are all what we will now call OBJECTS
  \item we want to make our own new datatypes for many different situations
  \item two main pieces to make an object, components and functions, just like motorcycle above
    hierarchies
  \end{itemize}
\item component hierarchy
  \begin{itemize}
  \item has-a relationship
  \item eventually get to atomic elements (int, bool, string, list etc)
  \end{itemize}
\item function hierarchy
  \begin{itemize}
  \item is-a relationship
  \item e.g. motorcycle is a vehicle, so is a bicycle, boat, truck, train, spaceship.
  \item all related by function
  \end{itemize}
\end{itemize}

  Let's do an example together: airport.  First has-a hierarchy.  Then
  make some is-a hierarchies.  Talk about components, their functions
  (NOT their purpose!), and how they relate to one another.

  Explain assignment: analyze a system in groups.  Be prepared to
  present next class.

\subsection*{System analysis HW}

Split into groups of 3 or 4 (use binary cards), have each group choose
a system:

\begin{itemize}
\item Hospital
\item Soccer game
\item College class
\item US Government
\item Zoo
\item Airport
\end{itemize}

For Monday, analyze your system.
\begin{itemize}
\item Components?
\item Functions of components?
\item Information flow bewteen components?
\item Has-a hierarchy
\item One or more is-a hierarchies involving some of the components
\end{itemize}

Prepare a presentation, \emph{use projector} (don't waste time drawing
on board) and address all 5 of the above things.

\begin{reflect}{Brent, S'16}

In S'16 there was a 
\end{reflect}

\section{System/hierarchy analysis presentations}

\newpage
\section{Intro to classes \& objects}

Birthday cake problem:
\begin{verbatim}
  # If n lit candles, blow out a random # between
  #   1 and n
  # How many blows does it take (on average)
  #   before they are all out?
\end{verbatim}

\begin{itemize}
\item Introduce idea of objects:
\begin{verbatim}
# We can package up data into "objects"
# Objects have functions, i.e. things they
#   can do. aka "Methods".

# Python "classes" allow us to design our
#   own new kinds of objects.
\end{verbatim}
\item Introduce idea of classes:
\begin{verbatim}
# class = template for objects
# an object is an "instance of" a class
# e.g. class = Car, object = my car
\end{verbatim}
\item Make empty \verb|Cake| class.
\item Show how we can make a \verb|Cake| object.
\item Show how we can set a variable, \verb|num_candles|, inside it.
  Draw stack/memory diagram.
\item Create an \verb|__init__| method with a parameter. (Explain
  \verb|self|; explain \verb|__init__|.)
\item Create \verb|blowout| and \verb|allout| methods.
\item Fill out the rest of the simulation:
\begin{verbatim}
class Cake:

    # __init__ is a special method
    #   that gets called when a
    #   new object is created.
    #   i.e. when Cake() is called.
    def __init__(self, num_candles):
        # self is a special first parameter
        # that gets filled in by Python

        # self.candles is a variable in the
        #   object being created
        #   (candles = ... would make a local var)
        self.candles = num_candles

    # Methods = things that all Cakes can do

    def blowout(self):
        n = random.randint(1, self.candles)
        self.candles -= n

    def allout(self):
        return (self.candles == 0)


def happy_birthday(c):
    count = 0
    while not c.allout():
        c.blowout()
        count += 1

    return count

def average_blows(num_candles, trials):
    total = 0
    for i in range(trials):
        c = Cake(num_candles)
        total += happy_birthday(c)
    return total / float(trials)

def main():
    for num_candles in range(100):
        print num_candles, average_blows(num_candles, 10000)

main()
\end{verbatim}
\item Run it and paste output into Excel, do a regression to show that
  it is close to natural log.
\end{itemize}

\newpage
\section{More classes \& objects}

Traffic light example.  Main point: encapsulation, abstraction.

\begin{verbatim}
# Traffic light
#
# How would you model it in Python?
# What functions should it have?
# How would it work?

# Possible functions:
#   - separate red, green, yellow light functions
#   - interval function
#   - sensor
#   - change to next color
#   - say what the current color is
#   - run
# Variables / state:
#   - current color
#   - timer (time left until next change?)


# Version 1:
#   - easy to write
#   - tedious (imagine cycling through 20 colors)
#   - difficult to change
class TrafficLight:

    # Variables:
    #   - current_color (string)

    # Create a new red traffic light
    def __init__(self):
        self.current_color = "RED"

    # Change to next color
    def change(self):
        if self.current_color == "RED":
            self.current_color = "GREEN"
        elif self.current_color == "GREEN":
            self.current_color = "YELLOW"
        elif self.current_color == "YELLOW":
            self.current_color = "ORANGE"
        elif self.current_color == "ORANGE":
            self.current_color = "RED"
        else:
            print "The sky is falling!!"  # This should never happen

    # Return the current color
    def color(self):
        return self.current_color

# Version 2:
# Use a dictionary of colors
class TrafficLight2:

    # Variables:
    #   - current_color
    #   - color_dict

    def __init__(self):
        self.current_color = "RED"
        self.color_dict = \
          { "RED" : "GREEN",
            "GREEN" : "YELLOW",
            "YELLOW" : "ORANGE",
            "ORANGE" : "RED" }

    def change(self):
        self.current_color = self.color_dict[self.current_color]

    def color(self):
        return self.current_color

# Version 3: list of colors
class TrafficLight3:

    # Variables:
    #   - color_list
    #   - current_index

    def __init__(self):
        self.color_list = ["RED", "GREEN", "YELLOW", "ORANGE"]
        self.current_index = 0

    def change(self):
        self.current_index += 1
        self.current_index %= len(self.color_list)

    def color(self):
        return self.color_list[self.current_index]

def main():
    t = TrafficLight3()
    print t.color()
    for i in range(10):
        t.change()
        print t.color()

main()
\end{verbatim}

\section{Yet More classes \& objects (Monday, 9 Nov)}

Mechanical pencil exercise.  Have them work in teams to code a class
representing a mechanical pencil.  Paste code into Google doc, look
through it together.

\section{Yet Even More classes \& objects (Wednesday, 11 Nov)}

Card and Deck examples.
\begin{itemize}
\item Make \verb|Card| class with \verb|rank|, \verb|suit| (strings),
  and \verb|face_up| (boolean).  Though can leave \verb|face_up| until
  later.  Just has getter methods.  Write \verb|__repr__|. (If usung
  \verb|face_up|, \verb|__repr__| should return \verb|**| or something
  like that if face down.)
\item Start making \verb|Deck| class with list of cards.  Make
  \verb|__init__| (create 52-card deck), \verb|shuffle| (use
  \verb|random.shuffle|), \verb|draw| (draw one card), \verb|deal(n)|
  (draw $n$ cards)
\end{itemize}

\section{Still Yet Even More classes \& objects (Friday, 13 Nov)}

Finish Deck, make Hand, Player, HumanPlayer, ComputerPlayer.

\section{Fill in stuff!}

\section{Queues}

I usually start with a discussion of the queue data structure itself and
why it would be useful. We might want to model lines of people in the
supermarket or at a bank, and determine why each has a different way of
organizing people (multiple lines with one checkout person each, vs one
line with multiple tellers).

I then walk through the methods of a queue
(\verb|add, remove, size, is_empty|) and some important words (front,
back) and write up the super class Queue with all the method bodies
being “pass”.

Easiest way to implement a queue is to use a list, in terms of lines of
code. List already do everything we need, and we're just constraining the
interactions to be with the front [0] and back [-1]. We code up the
ListQueue class together.

Then we test the ListQueue with the tester function and the main function.
As we put stress on the queue with larger numbers, we see it's not a linear
relationship, but quadratic. 10 ** 7 is a good number to stop, since it
takes about 100 seconds to complete. I haven't in the past, but I think I
will make a log-log plot of the size vs the time.

So, maybe a list, because it is very flexible and can access any element,
is too much for this Queue task, and we could do better with another way of
implementing a Queue where we only really care about the front and back.

Then, I discuss the Node abstraction. Break up a list into individual
pieces that fit together like Legos. It's a recursive data type, something
that stores an element and another Node as components.  This requires a
discussion of the word None. We've seen it as the return value of a void
function. Python has a word for “nothing”, and we can use that word when we
need a variable but it has no value.

Conveniently, we can calculate the size of a Node with a very simple
recursive function.

How do we use this to make a Queue? Make the NodeQueue class, and have two
components in init, the front and back, which start off as None.

Write size, very simple,  and \verb|is_empty|, base it on self.front being empty.

First, I try to complete add and remove the natural, but wrong, way. A node
is then someone standing in line. When people stand in lines, they look at
the person in front of them. So we start with the \verb|is_empty| case, and set
set.front = Node(e) and self.back = self.front. This is ok.

We then write the case when there are already people in line. We have a new
element to add to the back, so we write

temp = Node(e)
temp.next = self.back
self.back = temp

to reset the back pointer. All looks good so far.

But we can't remove people from the queue easily. We can get the element we
need from self.front, but who is the next self.front? self.front.next ==
None, and this is a problem. We could walk from the back to get the element
right behind self.front, with complicated two-finger algorithm, and it
takes a long time. Booo.

So instead, we reverse the orientation of everyone in line. You stand in
line backwards and stare at the person behind you. They are your next. If
you do this in real life, people will get a little uncomfortable. Just go
with it, computer scientists are weird sometimes because it's the right
thing to do, like starting to count at 0…

Then we rewrite add to match the code in the queuestuff.py file, and
finally we can write a remove method, to set self.front = self.front.next.
Both are now constant time.

How does this fare with our tests? For small numbers, a little more
expensive. The overhead of making new Node objects all the time is seen.

But as it gets larger, the time is linear with the size. for 10 ** 6 and
beyond, NodeQueue will beat ListQueue hands down. You can go back to plot
them in excel.  I've included an image in the code directory of these two
lines on the log-log plot.

I finish with discussing how this analysis and construction of data
structures is at the core of the science part of computer science, and will
be heavily discussed in CSCI 151 with more formal tools and for more
complicated structures, within the context of Java.
\end{document}
