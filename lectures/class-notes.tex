% -*- compile-command: "pdflatex class-notes.tex" -*-

\documentclass{article}

\usepackage{hyperref}

\begin{document}

\title{CSCI 150: Foundations of Computer Science \\ Fall 2015 Lecture Notes}

\maketitle

\section{Introduction (Wednesday, 26 August)}

\subsection*{Setup}

\begin{itemize}
\item Music: Copland, Fanfare for the Common Man
\item Meet students.
\item Introduce myself.
\item Have students meet each other. (Name, where from, why taking
  this class, one of your favorite things.)  Prize for first student
  to know all names.
\end{itemize}

\subsection*{What is Computer Science?}

This class is Foundations of Computer Science.  What is Computer
Science?  Look at each component separately.
\begin{itemize}
\item \textbf{What is a computer?}  Get some responses.
\item \textbf{What do computers do?}  Lots of things.  Commonality:
  information.  Communicating, transforming, analyzing, storing.
\item \textbf{What is science?}  Get some responses.  Scientists use
  rational investigation \& analysis, mathematics, etc. to study the
  natural world.
\end{itemize}

So what do computer scientists study?  \textbf{NOT computers!}
Rather, \textbf{information} = anything that can be expressed digitally,
i.e. with numbers/symbols.  ``Computer science'' is actually a bad
name, \emph{cf.}\ ``telescope science''.  Information structure of the
universe.

Why study it?  (Maybe get some student responses)
\begin{itemize}
\item Beautiful ideas, new ways of thinking.
\item Many applications!  Can contribute directly to human flourishing.
\item Computers are everywhere.  Understanding principles of CS =
  being an informed, engaged citizen.
\end{itemize}
Could do this without a computer, but computers are excellent enabling
tools.

\href{https://www.youtube.com/watch?v=qYZF6oIZtfc}{\textbf{Watch code.org video}}.

\subsection*{Administrivia}

\begin{itemize}
\item Syllabus review online.
\item Academic integrity.
\item BYOL.
\item Moodle.
\item Office hours.
\item Remember to come to lab!
\end{itemize}

For next time:

\begin{itemize}
\item Read chapter 1 of textbook.
\item Do HW 0, ``who are you'', if not already done.
\end{itemize}

\subsection*{Scratch intro}

Basic intro to Scratch for first lab.

\newpage

\section{Algorithms and language (Friday, 28 August)}

\subsection*{Setup}

\begin{itemize}
\item Music: Attaboy from Goat Rodeo Sessions. 5:42.  Start at 8:04.
\item Bring origami paper.
\end{itemize}

\subsection*{Administrivia}

\begin{itemize}
\item I will be gone next week.  ICFP.  Functional Programming ---
  super cool.  Really get to see intersection of math \& CS. Take CSCI
  490 in the spring (with MATH 240)!
\item In my place, Connor Bell will be lecturing \& will introduce you
  to Python.  Please show him the same respect you would show me!
\item Please download Python for Monday.  (Show python website, linked
  from our webpage.  Use version 2.7.)  \textbf{Bring your laptop!}
\end{itemize}

\subsection*{IRC}

Show IRC channel. Explain what it is for. Explain ground rules. Have
everyone log in \& try it out.

\subsection*{Reading review}

\begin{itemize}
\item Review 5 aspects of algorithms (input, output, math,
  conditionals, repetition).  Where did we see them in Scratch? Didn't
  use conditionals but saw the others.
\item Review 3 kinds of errors (syntax, semantic, runtime).  Talk
  about each in context of Scratch (doesn't have syntax errors !!!; lots
  of semantic errors; runtime = running into wall)
\item Fact that Scratch doesn't have syntax errors is a Really Big
  Deal.  Imagine if when learning a foreign language, every time you
  made even a small grammatical mistake the other person just cut you off and
  said ``I don't understand.''  That's what it will feel like learning
  Python.  So don't be discouraged---remember what you could do with
  Scratch.  You'll get there with Python too.
\item Talk about formal vs natural languages, tokens, parsing
\end{itemize}

\textbf{Quiz Monday on Chapter 1}.

\subsection*{Collatz Conjecture}

\begin{itemize}
\item Write out hailstone function.
\item Note all 5 aspects of algorithm.
\item Try it on some inputs: 4, 8, 6, 11. Draw a tree etc.  Try 27,
  realize we need a computer.
\item Simple algorithms can have very surprising results!  Visit
  Wikipedia page.
\end{itemize}

\subsection*{Origami}

\begin{itemize}
\item Explain what we are going to do: make a dinosaur.
  \begin{itemize}
  \item Everyone will be able to do it by the end of class.
  \item Then write instructions IN ENGLISH and find a confederate to
    follow them.  Can't show them image or video, or your dinosaur.
    Can't help interpret your instructions.  Just tell them to keep
    going as well as they can and get to the end.
  \item Monday: turn in both dinosaurs in class.  Turn in your
    instructions \& writeup on Moodle.
  \end{itemize}

\item Hand out origami paper.
\item Show video \& instruction image on the screen at the same time.
\item Wander around and make sure everyone can do it.
\end{itemize}

\section{XXX}

\section{XXX}

\section{XXX}

\newpage
\section{Conditionals (Wednesday, 9 September)}

\subsection*{Setup}

\begin{itemize}
\item Music: Whitacre ``i thank You God''. 6:56.  Start at 8:03.
\item Talk about Fabienne Serriere.  Show website.
\item Collect puzzle HW.
\end{itemize}

\subsection*{Quiz}
\textbf{Quiz 2}

We saw Boolean values {\tt True} and {\tt False}, in the puzzles, with
operators {\tt and}, {\tt or}, {\tt not} last time.  How else can we get
Booleans?

\textbf{Comparison operators}: \verb|>|, \verb|<|, \verb|>=|,
\verb|<=|, \verb|!=|, \verb|==|.  Demonstrate on numbers, strings, all
generate Booleans.

Now, the computer has a basis to make decisions.  Sets up branches in
the code, execute this or that.

\begin{verbatim}
passwd = input("What is the password? ")
if (passwd == "lemur"):
    print "Here be secrets."
\end{verbatim}

Tabbing is important!  Colon is important!

Introduce random module. ({\tt from random import *})
\begin{itemize}
\item {\tt random()} is number in $[0,1)$. Uniform.
\item {\tt randint(n)} is random integer between $0$ and $n$
  (inclusive).  (Aside: how can we write this in terms of {\tt
    random()}?)
\end{itemize}
Do coin flip.  Conditionally do... something.  Needs to include
\begin{itemize}
\item {\tt else}
\item {\tt elif}
\item nested {\tt if}  (flip two coins in a row)
\end{itemize}

\newpage
\section{Information encoding I (Friday, 11 September)}

\subsection*{Setup/administrivia}

\begin{itemize}
\item Music: Dave Brubeck Quartet, Kathy's Waltz
\item NB: exam 1 a week from today.  In class, closed notes, closed
  computer.  Covers material through today and Monday.  Wednesday will
  be exam review.
\item Collect any remaining logic puzzle HW!
\end{itemize}

\subsection*{Ada Lovelace}

How many of you have heard of Ada Lovelace?  How about Charles
Babbage?

Ada Lovelace: 1815 (December 10!)--1852, first programmer.  Wrote
programs for Charles Babbage's Analytical Engine (never built, but
definitely works in principle).  Babbage's purpose for the AE was
limited to making tables of numbers, but Lovelace had a much more
expansive and far-seeing vision:

\begin{quote}
``[The Analytical Engine] might act upon other things besides number,
were objects found whose mutual fundamental relations could be
expressed by those of the abstract science of operations, and which
should be also susceptible of adaptations to the action of the
operating notation and mechanism of the engine...

Supposing, for instance, that the fundamental relations of pitched
sounds in the science of harmony and of musical composition were
susceptible of such expression and adaptations, the engine might
compose elaborate and scientific pieces of music of any degree of
complexity or extent.''
\end{quote}

She saw potential for computers to operate on many kinds of
information instead of just making tables of numbers.  Over the next 2
classes we will consider some of the fundamentals that make this
possible.

\subsection*{Binary}

Computers represent numbers using \emph{binary} = base 2 instead of
base 10.  Recall how base 10 works: ones ($=10^0$) place, tens
($=10^1$) place, hundreds ($=10^2$) place, \dots  And ten different
digits.  For example \[ 4397_{10} = 4 \times 10^3 + \dots \] Base 2 is
the same but
\begin{itemize}
\item We use $2$ instead of $10$ (ones place, 2's place, 4's place,
  8's place\dots)
\item We use $2$ digits $(0,1)$ instead of $10$.  Called \emph{bits} =
  \emph{bi}nary dig\emph{its}.
\end{itemize}
Computers do everything in binary since (1) it is no less expressive
than e.g. decimal and (2) from a physical point of view, it is easier
to design hardware that can distinguish two different states
(e.g. high and low voltage) than 10.

Converting base $2$ to base $10$: have them do some
examples. e.g. $10110100_2 = ?$  Then do one example together on the
board, \[ 10110100_2 = 1 \times 2^7 + 1 \times 2^5 + \dots \]

Note 1 \emph{byte} = 8 bits.  Incidentally, one \emph{kilobyte} is not
$1000$ bytes, but actually $2^{10} = 1024$ bytes!  Megabyte is
$2^{20}$ bytes, and so on.

Questions for students (pair \& share etc):
\begin{enumerate}
\item How many different binary numbers using $n$ bits are there?
  (Write out 16 binary numbers using $4$ bits.)
\item What is the biggest number that can be represented using $n$
  bits?
\item How many bits are required to count up to $n$?
\end{enumerate}

Algorithm for converting from $n$ in base $10$ to binary:
\begin{itemize}
\item Find largest power of two $\leq n$, say, $2^k$.
\item As long as $n > 0$:
  \begin{itemize}
  \item If $2^k \leq n$, subtract $2^k$ from $n$ and write a $1$
    (in the $2^k$ place). Else, write a $0$.
  \item Decrease $k$ by $1$.
  \end{itemize}
\end{itemize}

(This has all five usual aspects of algorithms: input, output, math,
conditionals, repetition.  We can't quite write this algorithm in
Python yet because we don't know how to do repetition.  Soon!)  Do an
example, e.g. $103_{10}$.

\subsection*{Representing integers}

Integers are represented like this inside the computer.  Show entering
binary numbers directly into Python using \verb|0b10110| notation.

Most modern computers use $64$ bits to represent an integer (\emph{how
  big is that?}); some use $32$. Show Python switching from \verb|int|
to \verb|long|:
\begin{itemize}
\item Some small and very big examples
\item \verb|2**62|
\item \verb|2**63|
\item Apply \verb|type| to each of the above.
\end{itemize}

What about negative numbers?  One possibility: use one bit for sign
(plus or minus).  (Turns out there is a better way, ``$2$'s
complement''; learn about it in CSO or ask if you are curious.)

\newpage
\section{Information encoding II (Monday, 14 September)}

\subsection*{Setup}
\begin{itemize}
\item Music: ?
\item Quiz 3
\end{itemize}

\subsection*{Representing floating-point numbers}

Back to base $10$: what does $123.45$ mean?
\[ \dots + 4 \times 10^{-1} + 5 \times 10^{-2} \]
We can do binary ``decimals'' (``binarals''?) the same way: \[
1101.011 = 1 \times 2^3 + \dots + 1 \times 2^{-1} + 1 \times 2^{-2} =
13 \frac{3}{8} \]  Also recall ``scientific notation'', e.g. $1.23
\times 10^{17}$.  This is how ``floating point'' numbers are
represented: scientific notation, but in base $2$.  For example, with
$64$ bits:
\begin{itemize}
\item 1 bit for sign ($\pm$)
\item $11$ bits for exponent (base $2$ integer)
\item $52$ bits for value
\end{itemize}
e.g. $-1011011 \times 2^{-3}$.

How would you represent $0.1_{10}$?  Can't represent exactly using
base $2$ (infinite)! Show $0.1 + 0.1 + 0.1$ at python prompt.

\subsection*{Representing text}

Basic idea: use a different number to represent each letter.  ASCII
(American Std. Code for Info. Exch.) --- early 1960's.  Specified
$128$ different characters, each using $7$ bits.  (In many cases 1 bit
left over for parity checking or just set to $0$.)
\begin{itemize}
\item Show ASCII chart.
\item Illustrate chr and ord functions in python.
\end{itemize}
$128$ characters may have been enough for the white, American,
English-speaking men who made it up.  But it sure isn't any more.
Unicode---currently over 120,000 characters. (How many bits needed?
Often uses a more complex scheme to allow different numbers of bits,
\& extending indefinitely without changing existing.  Ask if you're
curious.)

\subsection*{Representing images}

Image = grid of colored points (``pixels'' = \emph{pic}ture
\emph{el}ements).  Each pixel = mix of red, green, blue (additive
primary colors).  Each primary color has 256 possible intensities,
from 0 (off) to 255 (as bright as possible).  So each primary = 8 bits
(1 byte), each pixel = 24 bits.  How many bits/bytes for a $500 \times
500$ image?

\subsection*{Hexadecimal}

Base $16$.  Need 16 symbols: 0--9, a--f. So $a_{16} = 10_{10}$ and so
on. $\mbox{ace}_{16} = ?$

Note, we can group the binary digits into 4s.  $16 = 2^4$.  So each 4
bits corresponds to 1 hexadecimal digit.  Conversion back and forth is
super easy.  We often use hexadecimal as a more convenient way to read
and write binary.  Easier for humans to read and remember.

Show entering hexadecimal directly into Python using \verb|0xace|
notation.

Show example of RGB colors expressed in hexadecimal: show
\verb|style.css| from course website.

\end{document}
