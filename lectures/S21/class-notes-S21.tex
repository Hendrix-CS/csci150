% -*- compile-command: "latexmk -pdf class-notes-S21.tex" -*-

\documentclass{article}

\usepackage{hyperref, amsmath, xspace, array}
\usepackage[normalem]{ulem}
\usepackage{bbding}
\usepackage[dvipsnames]{xcolor}
\usepackage{datenumber}
\usepackage{ifthen}
\usepackage{xstring}

\newsavebox{\reflectbox}

\newcommand{\etc}{\emph{etc.}\xspace}
\newcommand{\eg}{\emph{e.g.}\xspace}

\newenvironment{reflect}[1]
{
  \noindent
  \begin{lrbox}{\reflectbox}
    \begin{minipage}[t]{\textwidth}
      \textbf{#1}:
}{
    \end{minipage}
  \end{lrbox}
  \fbox{\usebox{\reflectbox}}
}

\newcommand{\notready}{\textcolor{red}{\XSolidBold}\xspace}
\newcommand{\ready}{\textcolor{Green}{\Checkmark}\xspace}

%%%%%%%%%%%%%%%%%%%%%%%%%%%%%%%%%%%%%%%%%%%%%%%%%%

%% Set this to the date of the first day of class (Wednesday) %%
\setdatenumber{2021}{1}{20}

\newcommand{\shortdayname}{%
  \ifthenelse{\equal{\datedayname}{Thursday}}{Th}{\StrLeft{\datedayname}{1}}}
\newcommand{\shortmonthname}{\StrLeft{\datemonthname}{3}}

\newcommand{\showdate}{\thedateday\ \shortmonthname}
\newcommand{\showdowdate}{\shortdayname\ \showdate}

\newcommand{\advanceMWF}{
  \ifthenelse{\equal{\datedayname}{Monday}}
  {\nextdate\nextdate}
  {\ifthenelse{\equal{\datedayname}{Wednesday}}
    {\nextdate\nextdate}
    {\ifthenelse{\equal{\datedayname}{Thursday}}
      {\nextdate}
      {\nextdate\nextdate\nextdate}}}}

\newcommand{\firstclass}[1]{\section*{#1 (\showdowdate)}}
\newcommand{\class}[1]{\advanceMWF\firstclass{#1}}
\newcommand{\lab}[1]{\section*{LAB: #1 (\showdate\nextdate, \showdate)}}
\newcommand{\holiday}[1]{\advanceMWF\section*{No class: #1 (\showdowdate)}}

\begin{document}

\title{CSCI 150: Foundations of Computer Science \\ Fall 2021 Lecture
  Notes}

\maketitle

\firstclass{\ready Introduction to CS}

\begin{reflect}{Brent, S'19}
  Because of no class on MLK day, to get everything to line up,
  decided to collapse traditional first two days of class into one.
\end{reflect}
\begin{reflect}{Brent, F'19}
  There was a day missing in F'19 as well (labor day?) so this
  schedule worked out well.
\end{reflect}
\begin{reflect}{Brent, S'21}
  Decided to re-expand pre-Python intro into 2 classes again, because
  it fits and also to allow more time to figure out hybrid teaching
  weirdness at the start.

  Had them all introduce themselves; didn't put them in groups since
  they are all spread out over the theater.
\end{reflect}

\subsection*{Community}

\begin{itemize}
\item {}[Meet students before class.]
\item Introduce myself.
\item Get students into groups of 2--3 and introduce themselves.
\end{itemize}

% \item Have students meet each other. \eg use binary cards to
%   put them in groups of 4.  Then switch groups once or twice.
%   Introduce people you have already met.
% \end{itemize}

\subsection*{What is Computer Science?}

This class is Foundations of Computer Science.  What is Computer
Science?  (Terrible name: first, because why do you need to have
``science'' in the title?  \eg ``food product''.)  Look at each
component separately.
\begin{itemize}
\item \textbf{What is a computer?}  Get some responses.  (Can insert
  brief history lesson: Antikythera mechamism: 150--100 BC, analog
  computer.  Many other mechanical computers through the years,
  \eg Babbage Difference Engine, Zuse Z3, Curta.  IBM started off
  making machines for processing data stored on punch cards; big
  industry by 1940s. ``Computers'' referred to people.  First
  electronic computers: WWII, early 1940s.)
\item \textbf{What do computers do?}  Lots of things.  Commonality:
  information.  Communicating, transforming, analyzing, storing.
\item \textbf{What is science?}  Get some responses.  Scientists use
  rational investigation \& analysis, mathematics, \etc to study the
  natural world.
\end{itemize}

So what do computer scientists study?  \textbf{NOT computers!}
Rather, \textbf{information} = anything that can be expressed
digitally, i.e. with numbers/symbols.  ``Computer science'' is
actually a bad name, \emph{cf.}\ ``telescope science''.  Information
structure of the universe. (A much better name is ``Informatics''
which is used \eg in Europe but hasn't caught on in the US.)

Why study it?  (Maybe get some student responses)
\begin{itemize}
\item Beautiful ideas, new ways of thinking.
\item Many applications!  Can contribute directly to human flourishing.
\item Computers are everywhere.  Understanding principles of CS =
  being an informed, engaged citizen.
\end{itemize}
Could do this without a computer, but computers are excellent enabling
tools.

\subsection*{Collatz Conjecture}

\begin{itemize}
\item Write out hailstone function. Something like this:
  \begin{quote}
    Start with a number $n$. \\
    Repeat the following until $n = 1$: \\
    \mbox{} \quad If $n$ is even, divide $n$ by $2$. \\
    \mbox{} \quad Otherwise, multiply $n$ by $3$ and then add $1$. \\
    Output the number of repetitions needed to reach $1$.
  \end{quote}
\item Do an example all together: start with 6.
\item Break students into groups, tell them to try it on some other
  starting numbers.
\item You were carrying out an \emph{algorithm}: a specific sequence
  of steps to solve a problem or accomplish a task.
\item What do you notice?
\item What do you think will happen in general?  Introduce Collatz
  conjecture.
\item Do you think we could get a computer to test this for us?  Sure!
  This class is all about learning how to get a computer to do what we
  want, and how to use them to think about problem-solving.  You may
  not personally care about the Collatz conjecture but there are other
  things you do care about.
\end{itemize}


For next time:

\begin{itemize}
\item Do HW 0: fill out info sheet.  (Linked from webpage.)
\item Review syllabus online.
\item Lab today, tomorrow!
\end{itemize}

\lab{Minecraft Hour of Code}

\class{\ready Introduction to the course}

\subsection*{Setup/administrivia}
\begin{itemize}
\item \ready Homework due Wednesday (Minecraft programming by hand).
  Designed to help you start developing important skills, building and
  executing mental models of what is going on in the computer, to
  predict what code will do without running it.
\item Remind them of survey.  Please fill it out.
\end{itemize}

\subsection*{Aspects of algorithms}

\begin{itemize}
\item 5 aspects of algorithms we are going to see over and over again:
  input, output, math, conditionals, repetition.  Where did we see
  them in Collatz example?  Where did we see them in Minecraft?
\item 3 kinds of errors: syntax, semantic, runtime.  Talk
  about each in context of Minecraft (lots of semantic errors; runtime
  = running into wall/falling into lava?; not many opportunities for
  syntax errors (!!!), except for something like \eg a repeat loop
  block with no blocks inside it.)
\item Fact that Minecraft doesn't have (many) syntax errors is a
  Really Big Deal.  Imagine if when learning a foreign language, every
  time you made even a small grammatical mistake the other person just
  cut you off and said ``I don't understand.''  That's what it can
  feel like learning Python.  So don't be discouraged---remember what
  you could do with Minecraft.  You'll get there with Python too.
\end{itemize}

\subsection*{Course overview}

We're a team!  Go over expectations/responsibilities: theirs and mine.

Syllabus review online.
\begin{itemize}
\item Don't bring laptops
\item Academic integrity
\item Office hours
\item Grade breakdown, labs (location), late days
\item \textbf{Tell them about TA hours in the lab?} (Or schedule when
  I'm going to!)  The TAs are there to help!  Make use of them!
\end{itemize}

\class{\ready Python I}

\subsection*{Setup/administrivia}

\begin{itemize}
\item Show history slides.  Present Antikythera mechanism.
\item Reminder, HW due Wednesday.  Type or scan.  See website.
\item Quiz Friday on basic Python.  You'll get plenty of practice
  before then.
\end{itemize}

\subsection*{Metacognition}

Find someone they don't know.  Introduce yourself and tell about one
club, sport, activity, hobby they do.  Anything you do that
involves skill---knitting, cooking, sports, video games, \emph{etc.}.
(1) How do you improve at that?  Then ask them (2) how do you think
you improve at programming? Use your answers to (1) as an analogy.

Have each group report out and \textbf{introduce each other}.

\begin{reflect}{Brent, S'21}
  Since we are socially distanced I couldn't really put them in
  groups.  Instead just plan to have a few students share something
  they do and how they get better at it.
\end{reflect}

\textbf{How not to do well in this class}.
\begin{itemize}
\item Read a textbook.
\item Read over your notes.
\item Memorize names of things.
\end{itemize}

\textbf{How to do well in this class}.  Try things.  Ask questions.
Get help if you need it.  Write a little bit of code every day.  Key:
\textbf{active} engagement!  This class is about teaching you a new
way of thinking/problem solving, not about teaching you a set of
facts.

\subsection*{Volume of a Box}

We're going to start writing our first Python code!  We'll start with
something very simple: we want the computer to calculate the volume of
a box.  We will enter the width, depth, and height, and it will
multiply them.

\begin{reflect}{Brent, F'19}
  In the past we have used this opportunity to break the problem down
  into steps (\emph{e.g.} print a question, get input from the user,
  calculate the volume, etc.) but since we are now de-emphasizing IO
  this seems less relevant.
\end{reflect}

This involves input, output, and math.  No conditionals or repetition
yet.  Let's start learning the Python we will need to do this.  We'll
actually write the program next class.

\begin{reflect}{Brent, F'18}
  One student said it involved repetition since we repeatedly ask the
  user to enter input!  Good point.  Clarified that when we say
  ``repetition'' we specifically mean the ability to write some
  instructions once and have the computer repeat them more than once.
\end{reflect}

\subsection*{Basic Python (numbers and variables)}

\textbf{Note}: quiz Friday on basic Python!  You will have a chance to
practice on Weds/Thurs in lab.  There isn't a textbook on purpose, to
force you to play/try things instead of using the textbook as a
crutch.  You can use a Kaggle notebook or repl.it, play around.  No
substitute for actually doing stuff.

Show \url{python.org} website.  Load up Kaggle notebook.

In a notebook you have direct access to the Python interpreter.  You
say commands one line at a time, and get immediate answers.  Great for
testing simple expressions/lines of code and seeing what they do.

What kinds of things does Python give us?

\begin{itemize}
\item \textbf{Numbers}!

  2, 5, 10, integers, \etc  Use \texttt{type}.

  Simple math with numbers like $2+5$. \texttt{"2 + 5"}? not so much.
  Great as a quick calculator. Use () to force order of operations.

  Integers are great, but what about fractions? \texttt{2.5}
  \texttt{0.95}, \texttt{3e23}.

  Use \verb|type| to see the types of different things.  Mention
  exponentiation \verb|**|. (Could possibly mention mod \verb|%|, but
  I prefer to leave it until it is needed later in the semester.)

  Note that Python will automatically convert from \verb|int| to
  \verb|float| if necessary (but not vice versa).

\item \textbf{Variables}!

  For keeping stuff around in short-term memory.
\begin{verbatim}
  x = 3
  z = 2.5
\end{verbatim}
  These are \emph{assignment} statements.  No output any more. Before,
  any value returned from the interpreter to us, now it goes into
  memory.  Type \texttt{x}, it spits back 3.

  \begin{reflect}{Brent, S'21}
    Used to introduce syntax like \texttt{x : int = 3} here, but
    Kaggle does absolutely nothing with it.  Going to wait and
    introduce that syntax when we start on PyCharm.
  \end{reflect}

  % Notice how we tell
  % Python what type of values the variable will hold.  (But note we
  % only have to do this when we introduce the variable for the first
  % time, not every time we use it.)

  Draw pictures, \texttt{x} has a little box containing $3$, it's just a
  name for the value $3$.

  \texttt{x = 4} reassigns \texttt{x} to be a new number.
  % \texttt{x =
  %   "Brent"} works too.

  \texttt{y} is a runtime error, gives us an \emph{exception}, which
  means something is not right (the \emph{other} meaning of
  ``exceptional''). So double check your names!

  \textbf{Be sure to emphasize} that the expression to the right of
  the \texttt{=} is \emph{evaluated} to a value which is then assigned
  to the variable.  Example:
\begin{verbatim}
x = 2
y = x + 3
x = 5
\end{verbatim}
  What value will \texttt{y} have afterwards?  It will still just be
  $5$.  It does not change when \texttt{x} changes later.  It just
  holds a number, it does not remember where the number came from.

  Some names are reserved, we'll see them as we go.
  Names can't be anything, letter or underscore first, then letter,
  number or underscore.

  Choose descriptive variable names! a, b, c, d, e, don't really help.
  Use names like \texttt{width}, \texttt{height},
  \texttt{num\_cats}, \etc

\end{itemize}

% \textbf{Mention HW assignment}: Minecraft by hand.  Posted on
% website. Due Wednesday (no class Monday = Labor Day).

\begin{reflect}{Brent, F'19}
  At this point we used to also cover strings, but perhaps we should
  move this later and introduce variables instead.  Strings aren't
  needed yet if we're doing stuff in notebooks.  Here's the notes for
  the stuff we used to cover: \medskip

  Chunks of text, in quotes. Just like you're reading a book:
  ``Brilliant!'' Harry said.

  Use \texttt{type} to ask what type it is.

  Both types of quotes work (single, double), must end with same you
  start with. And for really big strings, use triple quotes like
  \texttt{""" lkwjelrkjwe """}

  How can we break this? \texttt{'haven't you heard?'}  Use one to
  enclose the other.  If we need both, use \texttt{"""}, or the escape
  character \texttt{\textbackslash}: \texttt{\textbackslash ", \textbackslash t,
    \textbackslash n, \textbackslash\textbackslash}. These are special.

  The red text we see when things go wrong are examples of Syntax
  Errors.

\end{reflect}

\newpage

\class{\ready Python II, Box Volume}

\subsection*{Setup/administrivia}
\begin{itemize}
\item Remember: quiz Friday on basic Python syntax, variables, arithmetic.
\item Bring color printouts of Minecraft world.
\end{itemize}

\subsection*{Minecraft pair simulation exercise}

Have the students pair up and exchange programs (which they wrote for
HW).  Give each pair/group a color printout of the minecraft world for
the assignment, to use for reference.  Give them 10--15 minutes to
simulate each others' programs and confirm whether they work properly.
(See whose is shortest? Any interesting/surprising solutions?)  Then
collect the assignments.

\begin{reflect}{Brent, S'21}
  Couldn't really do this since they are socially distanced and I had
  them turn in HW electronically.
\end{reflect}

Why did we do this?  One of the reasons the Minecraft Hour of Code lab
was much easier than the programming we will do for the rest of the
semester we already talked about---because it allows far fewer syntax
errors.  But another, perhaps more important, reason is that it gives
you immediate, visual feedback when you run your program, such that
you can follow along and quickly figure out what's happening when
something goes wrong.  When writing a Python program we won't have
that luxury.  Being able to accurately predict what a program will do
without running it on the computer is an extremely important
skill---in fact, there is actual research that shows this skill is
highly correlated with your performance in this class!  So we will
have lots more opportunities to practice it.

\begin{reflect}{Brent, F'19}
  We used to cover basic I/O here (\texttt{print}, \texttt{input}) but
  now that we're doing stuff in notebooks it's irrelevant until later.
\end{reflect}

\subsection*{Our first function!}

We want to write some Python code to compute the volume of a box.
We're going to use a \emph{function} which is like making a little
machine or box that takes some \emph{inputs} in one end and produces
an \emph{output} or \emph{result} on the other end.  Draw a generic
picture.

\begin{verbatim}
def box_volume(width: float, height: float, depth: float) -> float:
  volume = width * height * depth
  return volume
\end{verbatim}

The first line says what the function is called, the names and types
of its inputs, and the type of its output.  Draw a picture: a machine
labelled \verb|box_volume| with three labelled inputs and one
output.

We can make variables to hold intermediate results.  We output the
final result using the \verb|return| keyword.

Test it out at the prompt with different inputs!

Once this is working, write some other functions:
\begin{itemize}
\item Compute the surface area.
\item Compute the length of the interior diagonal
  ($\sqrt{w^2 + h^2 + d^2}$).  (Lets us explicitly talk about how to do
  roots as a fractional power, needed for Kepler/Newton lab.)
\end{itemize}

Show them the \verb|math| library:

\begin{verbatim}
import math
\end{verbatim}

Lots of nice functions in there.  See
\url{http://docs.python.org/lib/module-math.html}.  For example,
\verb|log|, \verb|arctan|, \verb|ceil|, \verb|degrees|, \verb|sqrt|
and constants like \verb|e|, \verb|pi|.  Can access them with dot
notation: \verb|module.function|.

\verb|math.log(y)| is \emph{natural} log (base $e$). If you want
\eg base 2, do \verb|math.log(10, 2)|.

Do more examples, \eg \verb|math.sqrt(math.log(math.pi))|.  Function
composition!

\begin{itemize}
\item Write a function to compute the volume of a cylinder.  Show how
  to import \texttt{math} to use \texttt{math.pi}.
\end{itemize}

\begin{reflect}{Brent, S'17}
  I used the box example and this lecture went very well.  Things we
  discussed/questions the students had included:

  \begin{itemize}
  \item Order of operations; we played around with putting parens in
    different places and seeing whether we still got the correct
    answer.
  \item We noted that we get a runtime error if we convert with
    \verb|int| and it does not look like an int.  I asked students
    what we would need to catch this and re-ask the user; we would
    need conditionals and repetition.  Promised we would be able to do
    this later in the course.
  \end{itemize}
\end{reflect}

\newpage

\lab{Kepler \& Newton}

\textbf{Talk to TAs before lab about common errors so they can help
  give good hints when they see them.}

\begin{reflect}{Brent, S' 16}
This lab seemed to go pretty well.  Common errors:
\begin{itemize}
\item Forgetting that \eg \verb|1/3| is 0, not $0.333\dots$ (to take
  cube root).  [This is no longer a problem with Python 3.]
\item Dividing by $1.4\times 10^n$ is not the same as multiplying by
  $1.4 \times 10^{-n}$.
\item Bad parenthesization.
\item Algebra mistakes.
\end{itemize}

\end{reflect}

\begin{reflect}{Brent, S' 17}
  I had forgotten to tell them a few minor things but cleared them up
  easily during lab.  Some interesting errors I saw this time around:
  \begin{itemize}
  \item \verb|6.67 * 10e-11|.  Hint: \verb|10e-11| and \verb|1e-11|
    are not the same.  Either \verb|6.67e-11| or
    %
    \verb|6.67 * (10 ** -11)| are both fine.
  \item As usual many students missed the ``kilo'' in ``149 million
    kilometers'' and were off by a factor of 1000.
  \end{itemize}
\end{reflect}

\begin{reflect}{Brent, F'18}
  Went pretty well.  Some errors from this time:
  \begin{itemize}
  \item They didn't know about creating project vs creating a file,
    lots of projects named \texttt{orbit\_kepler.py} (not a big deal).
  \item Converting from days to seconds but then using the old days
    variable in the formula later.
  \item \verb|(4 * pi) ** 2| vs \verb|4 * pi**2|.
  \item \verb|x / y * z| vs \verb|x / (y * z)|.
  \end{itemize}
\end{reflect}

\newpage

\class{\ready Booleans}

\subsection*{Setup}

\begin{itemize}
\item \ready \textbf{Quiz (Python arithmetic and variables)}.  See
  \texttt{csci150-instructors/quiz/basic.tex}.  Go over quiz answers
  afterwards (\eg have them confer with neighbors, then have pairs
  confer with other pairs, \etc) and allow them to update their
  answers if they want.
\item \ready HW over the weekend: Boolean practice. Due Wednesday,
  also quiz. (Advice: finish by Monday, ask for help if you need it before
  the quiz on Wednesday.)
\end{itemize}

\subsection*{Strings}

Today: two new Python types!  First: {\tt str} (string).  A string is
how Python represents text: a string is just a sequence of characters.
Write by enclosing in double quotes (or single quotes).  Show some
simple examples.  Show calling \verb|type| on a string.

\subsection*{Booleans}

Next: {\tt bool}! (Named for George Boole, 2 November 1815 -- 8
December 1864, English mathematician \& philosopher, published ``The
Laws of Thought'' in 1854 which introduced Boolean algebra.)

Boolean values {\tt True} and {\tt False}.  Show these at the Python
prompt.  Operators {\tt and}, {\tt or}, {\tt not}.  Draw truth tables
on the board.

How else can we get Booleans?  \textbf{Comparison operators}:
\verb|>|, \verb|<|, \verb|>=|, \verb|<=|, \verb|!=|, \verb|==|.
Demonstrate on numbers, strings, all generate Booleans.  See if they
can guess what comparisons do for strings.  Come up with a hypothesis,
test it.

\newpage
\class{\ready Conditionals}

\subsection*{Setup/administrivia}

\begin{itemize}
\item CS history! (Lovelace + Babbage)
\item Reminder: HW due and quiz on Wednesday (Booleans)
\end{itemize}

\subsection*{Conditionals}

You already know about conditionals.  Here's how we do them in Python.

\begin{verbatim}
def get_secrets(passwd: str) -> str:
  if passwd == "lemur":
    msg = "Here are all the secrets!"
  else:
    msg = "No secrets for you!"

  return msg
\end{verbatim}

Tabbing is important!  Colon is important!  The thing after {\tt if}
can be \emph{any expression which evaluates to {\tt true} or {\tt
    false}}.  So what kinds of things could go there?  Comparisons,
\verb|and|, \verb|or|, a variable storing a \verb|bool| value, a
function that returns a \verb|bool|, \etc

We actually don't need the \verb|msg| variable.  \verb|return| quits
the function immediately and returns the given value:

\begin{verbatim}
def get_secrets(passwd: str) -> str:
  if passwd == "lemur":
    return "Here are all the secrets!"
  else:
    return "No secrets for you!"
\end{verbatim}

In certain cases we don't even need an \verb|else|, for example:
\begin{verbatim}
def get_secrets(passwd: str) -> str:
  if passwd == "lemur":
    return "Here are all the secrets!"

  return "No secrets for you!"
\end{verbatim}
Note how this depends on fact that \verb|return| quits the function
instantly.

Introduce \verb|random| module. ({\tt import random})
\begin{itemize}
\item {\tt random.random()} is number in $[0,1)$. Uniform.
\item {\tt random.randint(n)} is random integer between $0$ and $n$
  (inclusive).  (Aside: how can we write this in terms of {\tt
    random.random()}?)
\end{itemize}
Do coin flip.  Conditionally do something.  Needs to include
\begin{itemize}
\item {\tt else}
\item {\tt elif}
\item nested {\tt if}  (flip two coins in a row)
\end{itemize}

Start with multiply nested {\tt if} and then show them {\tt elif} for
nicer syntax.

For example:
\begin{verbatim}
import random

# Note this is a function that doesn't take any inputs!
def raffle() -> str:
  r = random.random()
  if r < 0.1:
      return "Wow, you are unlucky!  Try again next time."
  else:
      if r > 0.9:
          return "You get a car!!!11"
      else:
          if r > 0.7:
              return "You get a bike!"
          else:
              if r > 0.5:
                  return "You get a book!"
              else:
                  if r > 0.3:
                      return "You get a sock!"
                  else:
                      return "You get a stick of gum!"
\end{verbatim}

And then rewrite it to:
\begin{verbatim}
# Here's a better way to write the same thing, using elif:

def raffle2() -> str:
  r = random.random()
  if r < 0.1:
    return "Wow, you are unlucky! Try again next time."
  else:
    if r > 0.9:
        return "You get a car!!!11"
    elif r > 0.7:
        return "You get a bike!"
    elif r > 0.5:
        return "You get a book!"
    elif r > 0.3:
        return "You get a sock!"
    else:
        return "You get a stick of gum!"
\end{verbatim}

\newpage
\class{\ready Tracing I}

\subsection*{Setup}

\begin{itemize}
\item (Collect Boolean HW)
\item QUIZ: Booleans (see \verb|csci150-instructors/quiz/booleans.pdf|)
\end{itemize}

\subsection*{Tracing}

We've already mentioned that being able to manually trace the
execution of a program and predict what it is going to do is a very
important skill.  You need to be able to do this in order to figure
out how to write a program to do what you want, or to figure out why a
program is not doing what you expect.  Today we're going to introduce
a formal way to do this for Python programs.  It won't be too hard to
start out, but will get more complicated as the semester progresses
and we add new Python features, so make sure you can do it well and
accurately!

For now, template just has two areas: scratch work in upper left,
variables in upper right (with types!).  (Relative positioning is
important because we're going to add extra stuff to this basic
template later in the semester!).  Work through simple example, \eg:
\begin{verbatim}
x = 3
y = x + 2
x = 7
y
\end{verbatim}
which leads to \bigskip

\fbox{
\begin{minipage}{0.9\textwidth}
  \begin{minipage}[t]{0.5\linewidth}
    \textbf{Scratch} \medskip

    $y = x+2 \to y = 3+2 \to y = 5$
  \end{minipage}
  \begin{minipage}[t]{0.45\linewidth}
    \textbf{Variables} \medskip

    $x: \mathrm{int}\;|$ \xout{3} 7 \\
    $y: \mathrm{int}\;|$ 5
  \end{minipage} \bigskip

\end{minipage}
} \bigskip

\begin{reflect}{Brent, S'21}
  Teaching in Cabe Theater with no good blackboard or whiteboard, so I
  used \url{https://www.autodraw.com/} to draw stuff live in class.
\end{reflect}

The exact contents of ``scratch'' is not important.  But the contents
of the variable section should look exactly like that.

Have them do one on their own paper:

\begin{verbatim}
hello = "zebra"
shoes = 73
z = 5
msg = "hello"
if shoes > z * 7 or msg == "hi":
    z = 7
    if msg == hello:
        z = z + 12
    else:
        z = shoes + z
    shoes = 2
else:
    shoes = z
\end{verbatim}

Then go over it all together.

\begin{reflect}{Brent, F'19}

  We used to use this:

  \begin{verbatim}
z: str = "5"
msg: str = "hello"
print("z is " + z)
z = z + "2"
y: int = int(z) + 2
print(msg + "! " + z + " " + str(y))
\end{verbatim}

  But now that we do notebooks first they haven't seen \verb|print| or
  string manipulation yet.  So we do conditionals first to allow for
  something complex enough to trace.
\end{reflect}

Have them compare with a partner.  Then go over it quickly on the
computer.

\begin{reflect}{Brent, S'21}
  Due to COVID and social distancing, couldn't really put them in
  groups.  But going over all of this up to here took the whole
  period anyway.
\end{reflect}

Now give them 5-10 minutes with the same partner to make up a tracing
exercise.  Then we will swap with another group.  Try to trick them!


Let them swap and try to trace the other pair's code.  If you think
there is a syntax error or some other kind of error, talk to the group
that made it to clarify what they meant or how to fix it.

To wrap up, explain that one of the reasons we did this was to force
them into the mode where they turn off the part of their brain that
tries to figure out what the code \emph{means} and just use the part
of their brain that can emulate the computer and follow the code
step-by-step to see what it \emph{does}.  This is an important skill,
but it can be difficult especially when you are the one who wrote the
code, because it's hard to forget about what you \emph{want} the
program to mean.  I hope you will see tracing this way as not just
something we are making you do in class, but as a tool you can use \eg
in the lab when your code isn't doing what you expect.

\begin{reflect}{Brent, F'19}
  I had them work in pairs to create programs to try to trick another
  pair and then swap and try to trace the execution of the other
  pair's code.  This was a great success!  They came up with all kinds
  of creative ways to try to trick each other and it seemed like they
  got really good practice thinking about the details of Python
  execution.
\end{reflect}
\begin{reflect}{Brent, S'19}
  The above went really well---it doesn't seem like a lot of material
  but they asked a lot of great questions about exactly what would
  happen in various scenarios.  We still had about 10 minutes left but
  I judged it didn't make sense to do the below (make up an exercise
  and swap) in the remaining time, so I just talked about the HW and
  let them go 5 minutes early.
\end{reflect}

\newpage
\lab{Big Data/Heart Disease}

\begin{reflect}{Brent, S'16}
  \begin{itemize}
  \item Spend a few minutes at the beginning explaining the graphs,
    especially the Boolean ones.
  \item Make sure they don't re-save the data files from Excel---it
    messes up the formatting!  Opening in Excel to \emph{look at} the
    data files is OK, just don't save.
  \item Try to guide them to using nested ifs.
  \item Make sure they understand why it's ineffective to make a huge
    condition with a bunch of ANDs.
  \item Make sure they understand why it's ineffective to have a
    string of independent ifs, where each one sets \texttt{prediction}
    (overriding whatever \texttt{prediction} was set to before).
  \end{itemize}

  Other than those gotchas, lab seems to go OK.
\end{reflect}

\begin{reflect}{Gabe, F'16}
\begin{itemize}
\item First lab section: Seems a bit short - took most students 2 hours.
\item Second lab section: Nobody finished in less than 2.5 hours. Go figure!
\item The students found the histograms very confusing. In particular, the
axes were unclear for the string-valued elements.
\end{itemize}
\end{reflect}
\begin{reflect}{Mark, F'16}
\begin{itemize}
\item can't get 100\% on 4a, since {\tt 63, False} is found in both categories. Is possible to get 100\% on 4b and 4c.
\item the chart for chest\_pain has one of the data points swapped, second bar is "asymptomatic", should have been first.
\end{itemize}
\end{reflect}
\begin{reflect}{Brent, S'17}
This lab seemed to go very well this year.  The students were
definitely engaged and learning a lot.  A few started leaving after 2
hours but probably 2/3 of the class was still working at the 3 hour
mark (but most were just finishing or very close).
\end{reflect}

\newpage

\holiday{Spring Break}

\class{\ready Binary and information encoding}

\subsection*{Setup/administrivia}

\begin{itemize}
\item \ready \textbf{Quiz}: conditionals and tracing
  (\texttt{csci150-instructors/quiz/quiz3-alt-cond-practice.py}).
\item Exam 1 on Friday!  In class, closed notes, closed computer.
  Covers material through today. Practice exam posted on website; for
  homework please do it for Wednesday (not to turn in).  Wednesday
  will be review.  \textbf{Important}: This class is about
  \textbf{doing} more than \textbf{knowing}.  One of the \textbf{worst
    ways} you can study is to just review your notes, because it gives
  you a false sense of familiarity/security without actually giving
  you any practice actually writing code.  Much better way to study is
  to \textbf{do} things: more tracing exercises; rewrite code from
  class or lab; write some code of your own.  Review book/notes as
  needed, to answer specific questions/confusions you come across.
\item Show them my virtual office in Gather!  Have online students
  help test it.
\end{itemize}

\begin{reflect}{Brent, S'21}
  Because of ``Spring break'' day, only 1 day for Information encoding
  so just planning to skip it instead.  We will probably be a bit
  behind and can use this day for more practice etc\dots

  Well, so that's what I thought when planning out the course, but in
  fact we are not behind, so I will just go through a basic version of
  it as lecture, minus having them all work on answering questions in
  groups etc.
\end{reflect}

\begin{reflect}{Brent, F'18}
  Skipped this aside; there was too much other stuff going on at the
  beginning of class (quiz, handing back stuff, etc.) and I wanted to
  make sure I got through the binary stuff (which I did).
\end{reflect}
\begin{reflect}{Brent, F'19}
  This time I did actually do the aside which was fun.
\end{reflect}
\begin{reflect}{Brent, F'21}
  Actually put this in my history slides and had already presented it
  earlier.
\end{reflect}

\subsection*{Aside: history of information}

How many of you have heard of Ada Lovelace?  How about Charles
Babbage?

Ada Lovelace: 1815--1852, first programmer.  Wrote
programs for Charles Babbage's Analytical Engine (never built, but
definitely works in principle).  Babbage's purpose for the AE was
limited to making tables of numbers, but Lovelace had a much more
expansive and far-seeing vision:

\begin{quote}
``[The Analytical Engine] might act upon other things besides number,
were objects found whose mutual fundamental relations could be
expressed by those of the abstract science of operations, and which
should be also susceptible of adaptations to the action of the
operating notation and mechanism of the engine...

Supposing, for instance, that the fundamental relations of pitched
sounds in the science of harmony and of musical composition were
susceptible of such expression and adaptations, the engine might
compose elaborate and scientific pieces of music of any degree of
complexity or extent.''
\end{quote}

She saw potential for computers to operate on many kinds of
information instead of just making tables of numbers, and we know that
they can indeed do just that!  In fact, she even missed one very
important development, that computers can not only \emph{operate} on
these kinds of information, but allow us to \emph{communicate} using
these kinds of information as well.  Over the next 2 classes we will
consider some of the fundamentals that make this possible.

\subsection*{Binary}

Computers represent numbers using \emph{binary} = base 2 instead of
base 10.  Recall how base 10 works: ones ($=10^0$) place, tens
($=10^1$) place, hundreds ($=10^2$) place, \dots  And ten different
digits.  For example \[ 4397_{10} = 4 \times 10^3 + \dots \] Base 2 is
the same but
\begin{itemize}
\item We use $2$ instead of $10$ (ones place, 2's place, 4's place,
  8's place\dots)
\item We use $2$ digits $(0,1)$ instead of $10$.  Called \emph{bits} =
  \emph{bi}nary dig\emph{its}.
\end{itemize}
Computers do everything in binary since (1) it is no less expressive
than \eg decimal and (2) from a physical point of view, it is easier
to design hardware that can distinguish two different states
(\eg high and low voltage) than 10.

Converting base $2$ to base $10$: have them do some
examples. \eg $10110100_2 = ?$  Then do one example together on the
board, \[ 10110100_2 = 1 \times 2^7 + 1 \times 2^5 + \dots \]

Note 1 \emph{byte} = 8 bits.  Incidentally, one \emph{kilobyte} is not
$1000$ bytes, but actually $2^{10} = 1024$ bytes!  Megabyte is
$2^{20}$ bytes, and so on.

Some questions.  Pair and work on these. [Find your bit-partner:
figure out the correct base-10 number for your partner and find them
that way!]

\begin{enumerate}
\item How many different binary numbers are there using $4$ bits?
\item How many different binary numbers using $n$ bits are there?
  (Write out 16 binary numbers using $4$ bits.)
\item What is the biggest number that can be represented using $n$
  bits?
\item Convert to binary: $103_{10}$. (Don't tell them how to do it;
  see if they can figure it out.  Write a list of powers of 2 on the
  board for them.)
\end{enumerate}

Talk about answers to first three questions.  Promise we will talk
about conversion next class.

% \newpage
% \class{\notready Information encoding II}

% \subsection*{Setup}
% \begin{itemize}
% \item \notready \textbf{Assign practice exam}---not collected or graded, but we
%   will go over it on Wednesday.  In your best interest to go through
%   it and come prepared!
% \end{itemize}

Remember converting $103_{10}$ to binary last class?  You probably
naturally came up with something like the standard greedy algorithm.
Go over algorithm:

\begin{itemize}
\item Find largest power of two $\leq n$, say, $2^k$.
\item As long as $n > 0$:
  \begin{itemize}
  \item If $2^k \leq n$
    \begin{itemize}
    \item subtract $2^k$ from $n$
    \item write a $1$ (in the $2^k$ place)
    \end{itemize}
  \item else
    \begin{itemize}
    \item  write a $0$ in the $2^k$ place
    \end{itemize}
  \item Decrease $k$ by $1$.
  \end{itemize}
\end{itemize}

(This has all five usual aspects of algorithms: input, output, math,
conditionals, repetition.  We can't quite write this algorithm in
Python yet because we don't know how to do repetition.  Soon!)  Do an
example together on the board, \eg $103_{10}$ or something else. I've
found it seems to help some students to show them how to write out
blanks to represent different places in the binary number, then fill
them in with 1's or 0's, like so:
\[ \underset{32}{\underline{\phantom{XX}}} \; \underset{16}{\underline{\phantom{XX}}} \; \underset{8}{\underline{\phantom{XX}}} \; \underset{4}{\underline{\phantom{XX}}} \; \underset{2}{\underline{\phantom{XX}}} \; \underset{1}{\underline{\phantom{XX}}} \]
though I emphasize that I don't require them to do this.

\subsection*{Hexadecimal}

\begin{reflect}{Brent, F'19}
  In the past I've sometimes not gotten to hexadecimal and then had to
  send them an email about it because it's on the exam!  Not ideal.
  So I'm moving this section earlier to be sure to go over it.  The
  other stuff about representing text, images, etc. is fun and we can
  get through as much as there is time for.
\end{reflect}

Base $16$.  Need 16 symbols: 0--9, a--f. So $a_{16} = 10_{10}$ and so
on. $\mbox{ace}_{16} = ?$

Note, we can group the binary digits into 4s.  $16 = 2^4$.  So each 4
bits corresponds to 1 hexadecimal digit.  Conversion back and forth is
super easy.  We often use hexadecimal as a more convenient way to read
and write binary.  Easier for humans to read and remember.

Show entering hexadecimal directly into Python using \verb|0xace|
notation.

Show example of RGB colors expressed in hexadecimal: show
\verb|style.css| from course website.

\subsection*{Representing integers}

Integers are represented in binary inside the computer.  Show entering
binary numbers directly into Python using \verb|0b10110| notation.

Most modern computers use $64$ bits to represent an integer (\emph{how
  big is that?}); some use $32$. Show Python doesn't care, abstracted
it away.  In Python 3, everything is \texttt{int}; there is also
\texttt{long} which can be as big as you want.

What about negative numbers?  One possibility: use one bit for sign
(plus or minus).  (Turns out there is a better way, ``$2$'s
complement''; learn about it in CSO or ask if you are curious.)

\subsection*{Representing text}

Basic idea: use a different number to represent each letter.  ASCII
(American Std. Code for Info. Exch.) --- early 1960's.  Specified
$128$ different characters, each using $7$ bits.  (In many cases 1 bit
left over for parity checking or just set to $0$.)
\begin{itemize}
\item Show ASCII chart.
\item Illustrate chr and ord functions in python.
\end{itemize}
$128$ characters may have been enough for the white, American,
English-speaking men who made it up.  But it sure isn't any more.
Unicode---currently over 137,000 characters. (How many bits needed?
Often uses a more complex scheme, UTF-8, to allow different numbers of
bits per character, so files using ASCII don't need to be re-encoded.
Ask if you're curious.)

\subsection*{Representing images}

Image = grid of colored points (``pixels'' = \emph{pic}ture
\emph{el}ements).  Each pixel = mix of red, green, blue (additive
primary colors). (Have you ever put a tiny drop of water on your phone
screen to see the pixels magnified?  It's pretty interesting, try it
if you haven't!) Each primary color has 256 possible intensities, from
0 (off) to 255 (as bright as possible).  So each primary = 8 bits (1
byte), each pixel = 24 bits.  How many bits/bytes for a
$500 \times 500$ image?

\begin{reflect}{Brent, F'18}
  Got too carried away talking about ASCII and Unicode, images,
  \emph{etc.} (there was also a bunch of stuff at the start of class:
  handing things back, a quiz, \emph{etc.}) so I didn't actually make
  it to talking about hexadecimal.  (Apparently neither did Lars.)  So
  I sent them an email afterwards, reproduced in the LaTeX comment below.
\end{reflect}

% Hi all,

% There was one thing I didn't quite get to at the end of class today.
% It's not hard to explain, but important.

% "Hexadecimal" means base 16, just like "decimal" means base 10 and "binary" means base 2.  To write numbers in base 16, we need 16 "digits"---so we use the digits 0-9 as usual and then the letters A-F.  So A represents 10, B represents 11, and so on, up to F which represents 15.  (Often lowercase letters are used as well, it doesn't matter if they are upper- or lowercase.)

% For example, the hexadecimal number 3ACE means

% 3 * 16^3 + A * 16^2 + C * 16^1 + E * 16^0
% = 3 * 4096 + 10 * 256 + 12 * 16 + 14
% = 15054

% in base 10.

% Every hexadecimal digit corresponds to exactly 4 bits, so hexadecimal is often used as a convenient and more human-friendly way to represent sequences of bits.  For example, 3ACE is equivalent to 0011101011001110 --- which do you find easier to read?

% You may have seen hexadecimal notation before --- for example when a computer crashes it often displays a long sequence of numbers and letters, which is a hexadecimal number representing the memory address where the crash occurred.  Another place where you see hexadecimal notation is in writing down colors.  We talked in class about how a color is represented by 3 bytes; each byte is 8 bits which can be written using two hexadecimal digits.  For example, the color "lime" can be represented as AAFF32.  AA represents the amount of red (A * 16 + A = 10*16 + 10 = 170), FF is the amount of green (F * 16 + F = 15*16 + 15 = 255), and 32 is the amount of blue (note this is 32 in base 16, which is not the same as 32 in base 10!  3 * 16 + 2 = 50 base 10).

% If you have any questions we can go over them on Wednesday!

\begin{reflect}{Brent}
  There probably won't be time to cover anything past this point.
  Including it here just in case there's extra time.
\end{reflect}

\subsection*{Representing floating-point numbers}

Back to base $10$: what does $123.45$ mean?
\[ \dots + 4 \times 10^{-1} + 5 \times 10^{-2} \]
We can do binary ``decimals'' (``binarals''?) the same way: \[
1101.011 = 1 \times 2^3 + \dots + 1 \times 2^{-1} + 1 \times 2^{-2} =
13 \frac{3}{8} \]  Also recall ``scientific notation'', \eg $1.23
\times 10^{17}$.  This is how ``floating point'' numbers are
represented: scientific notation, but in base $2$.  For example, with
$64$ bits:
\begin{itemize}
\item 1 bit for sign ($\pm$)
\item $11$ bits for exponent (base $2$ integer)
\item $52$ bits for value
\end{itemize}
\eg $-1011011 \times 2^{-3}$.

How would you represent $0.1_{10}$?  Can't represent exactly using
base $2$ (infinite)! Show $0.1 + 0.1 + 0.1$ at python prompt.

\subsection*{Video, audio}

Can mention something about how video and audio are represented.
\begin{itemize}
\item Video: sequence of frames, each frame is an image.  Of course
  there is lots of redundancy to take advantage
  of---\emph{compression}.
\item Audio: sequence of ``samples''.
\end{itemize}

\newpage
\class{\ready Exam 1 review}

\subsection*{Setup}

\begin{itemize}
\item Note we \emph{do} still have lab today/tomorrow!  Should be a
  nice review for Exam 1.
\end{itemize}

\subsection*{Sample exam}

Go over sample exam solutions.

\newpage
\lab{\ready This Day in History}

This can be used as a comprehensive review of everything on Exam 1.

\begin{reflect}{Brent, F'18}
  For this lab they need both mod and integer division, so we covered
  them briefly at the start of lab.  I also explicitly showed them how
  to use mod 100 to find the last two digits of a number.  That aspect
  of the lab went pretty well, most students figured out how to use
  mod appropriately.

  Common mistakes (aside from the usual confusion over the rule for
  leap year) included:
  \begin{itemize}
  \item Taking the mod of \verb|dow| too early, \emph{i.e.} writing
    additional code that added offset factors to \verb|dow| in later
    steps and not reducing it mod 7 afterwards.
  \item The usual things re conditions \emph{e.g.} writing \texttt{if month
    == 3 or 5}, or \texttt{if month = 3}, or writing a sequence of
    \verb|if| instead of \texttt{if... elif... elif}.
  \end{itemize}

  As an aside, during the lab I wondered how to write the condition
  for a leap year without using nested conditionals.  Here are four
  different Boolean expressions I came up with that all work:
\begin{verbatim}
(y % 4 == 0) and ((y % 100 == 0) == (y % 400 == 0))
(y % 4 == 0) and not (y % 100 == 0 and y % 400 != 0)
(y % 4 == 0) and (y % 100 != 0 or y % 400 == 0)
(y % 4 == 0) != ((y % 100 == 0) != (y % 400 == 0))
\end{verbatim}
  Obviously we would never expect students to come up with any of these!
\end{reflect}

\begin{reflect}{Brent, F'19}
  Now that we have converted to using notebooks at the beginning of
  the course, we included a section introducing them to mod and
  integer division at the start of the lab, which seemed to work very
  well!
\end{reflect}

\newpage
\class{\ready Exam 1}

\begin{itemize}
\item Note, no HW this weekend.
\item \notready AFTER exam (e.g. on Saturday), email them to tell them
  about \textbf{Project 1}.
\end{itemize}

\begin{reflect}{Brent, S'18}
  Because of pushing exam 1 back one day, I only have \emph{two}
  lectures instead of three to prepare for Guess My Number lab.  What
  I plan to do is introduce while loops first (1 lecture), and then a
  basic intro to functions (1 lecture).  Functions are not actually
  super necessary for the lab, so it's OK to cover more about
  functions in another class after the lab.

  In the past I have used functions while doing examples for while
  loops (\eg testing the Collatz conjecture), but that's not
  critical---it's easy to do the examples without using loops.  It
  might even give nice material to motivate the process of function
  abstraction later.
\end{reflect}

\begin{reflect}{Brent, F'19}
  Now that they already have plenty of experience using functions with
  notebooks, we can just spend the two lectures introducing them to
  PyCharm, I/O, and while loops.
\end{reflect}

\newpage
\class{\ready PyCharm, I/O, while loops}

\subsection*{Setup}
\begin{itemize}
\item Hand back exams (if graded).
\item \textbf{Assign Project 1}.  Linked from website. Due 1 week from
  today (midnight on Monday).
% \item Explain required feedback meetings this week or next.  We will
%   send out an email with sign-up link.
\end{itemize}

\subsection*{PyCharm}

Introduce PyCharm.  Store code in a \verb|.py| file.  (Kind of like
one giant code cell in a notebook.)  Notebooks are great for
interactive things, explanations, documents with computational
content, etc., but sometimes you just want a Python program to run on
your computer and do something.  PyCharm also has a lot more nice
features to help you write Python programs, e.g. it can check your
types. From now on we'll use both, the next few labs will use PyCharm.

\emph{Quickly} show them:
\begin{itemize}
\item \verb|print|
\item \verb|input|
\item \verb|int|, \verb|float| conversion functions.
\item Format strings.
\end{itemize}
Note that the project has some instruction on these as well, they will
get more practice.

Show them how to run code in PyCharm.  Show them how to get a prompt.
Show them how to make comments!

\subsection*{While loops}

Introduce while loops---this is a momentous day!  Once we add while
loops you can (in theory) write \emph{any program you can imagine}.
Note, this also means things get harder!  If you thought stuff up to
exam 1 was easy, don't get lulled into false sense of security. Spend
a little bit talking about how to study.

Explain syntax.  Syntax and semantics are very much like \verb|if|
(without \verb|else|): difference is that \verb|if| executes 0 or 1
time, \verb|while| executes any number of times (0 or more).

Do an example: count from 1 to 10.  Play with boundary conditions,
order of statements inside the loop.

\newpage
\class{\ready More while loops, Collatz conjecture}

\subsection*{Setup}
\begin{itemize}
\item History: ENIAC.  Watch ENIAC programmers trailer.
\item Reminder: projects!  Due Monday.
\item Put some code with while loop up for them to figure out what it
  does as they come in?
\end{itemize}

\subsection*{While loops}

Review while loops. Point out three usual components:
\begin{itemize}
\item Declare and initialize variable(s) you need
\item Write the condition with those variables
\item Change those variables somewhere in the loop (often at the
  end)
\end{itemize}

Do another example: keep printing ``hi'' and asking ``Do you want to
stop?'' until says yes --- have students write this code together on
their computers/paper.  Then have two or three share code?  There are
two decent ways to write this --- using input variable to control loop
(but then you have to initialize it with a bogus value), or using a
boolean (sentinel).

\subsection*{Collatz}

Recall Collatz Conjecture.  Idea: write a program to test the
conjecture.  Start by writing inner loop with a particular starting
number.  Then put that in a loop to do it repeatedly for different
numbers.

\textbf{Note:} if not introduced previously, can use this opportunity
in passing to introduce abbreviated assignment forms like \verb|+=|.

\begin{verbatim}
n: int = 1

max: int = int(input("Test numbers up to: "))

while n < max:
    count: int = 0
    m: int = n
    while m != 1:
        if m % 2 == 0:
            m //= 2
        else:
            m = 3*m + 1
        count += 1

    print(f'{n}: {count}')

    n += 1
\end{verbatim}

(What happens if we take out \verb|max| check and just say
\verb|while true|?  A legitimate use of an infinite loop!)

\newpage
\lab{Guess My Number}

Have the students write both versions of the game (in the first
version, the computer picks the number; in the second, the computer
guesses).

\begin{reflect}{Brent, F'18}
  Went pretty well I think.  As usual:
  \begin{itemize}
  \item I saw quite a few students tried to use recursion to do the
    ``play again'' looping.
  \item Of course many students struggled to figure out how to have
    the computer play.
  \end{itemize}
\end{reflect}

\newpage
\class{\ready Functions Revealed}

\subsection*{Setup/administrivia}

\begin{itemize}
\item \notready HW due Wednesday, with quiz (function \& loop reading)
\item Reminder: project 1 due Monday (@ midnight)!
\end{itemize}

\subsection*{More on functions}

\begin{itemize}
\item From now on, \emph{everything} must be in a function!  Make a
  \verb|main()| function like on the lab, call to \verb|main()| at the
  very end.

\item \textbf{They must write a comment before every function!}
  Comment should clearly explain what the function does.  Points for
  ``documentation'' category of grading rubric.

\item Function definition makes a variable which is a function
  object. (Note hex value in output! =) To call it, use parentheses.
  \textbf{Every function call must always have parentheses after it.}

\item Put function definitions in a script.  Note the definition
  itself does not cause the function to be executed.  Python goes
  through the file from top to bottom; when it reaches a definition it
  simply \emph{remembers} it for later.  This means functions must be
  defined before they are used, but it is perfectly OK to call a
  function from inside another function \emph{before} it is defined.
  So order of function definitions doesn't matter; pick an order that
  makes sense.

\item The inputs to a function in its definition are called
  \emph{parameters}.  The actual inputs we give when we call a
  function are called \emph{arguments}.

\item Variables created inside functions are \textbf{local}!
  Parameters are also local.  Show some examples.

\item Note we can put types on variables as well as function
  declaration.
\end{itemize}

Along the way show PyCharm error highlighting.

\subsection*{Flow of execution \& tracing with stack diagrams}

Execution starts at the first line of a program and continues.  But
when a function is called, execution jumps to that function, and when
done picks back up where it left off.  This can get complicated, since
functions can call other functions!  In general Python has to keep
track of a ``stack'' of places to resume.  Imagine you were reading a
book but it told you to go read another book and then come back.  Then
THAT book told you to go read another book, etc.  You would keep the
old books, with bookmarks, in a pile.  When you finally finish a book,
pick up the next book from the top of the pile and continue.

Do an example. Make some functions that call other functions,
etc. Introduce an error and see the ``traceback''.

Put this on board/screen (see \texttt{stacktrace.py} in lecture notes
directory):
\begin{verbatim}
def g(z: int):
    z = z*3
    print(z)

def f(x: int):
    g(x)
    g(x + 2)
    print(x)

f(3)
\end{verbatim}
First have them try to predict output.  Compare with others, see if
they can agree.  Then show them method for formally tracing the
execution, with space for \textbf{scratch}, \textbf{stack}, and
\textbf{output}.  Stack grows down from the top.  See template on next page.
They will do more examples on HW!

\newpage
\noindent
  \begin{tabular}{| >{\centering\bfseries}p{2.8in} | >{\centering\arraybackslash\bfseries}p{2.8in} |}
    \hline
    Scratch & Stack \\
    \hline & \\[1in]
    \cline{2-2} & \\[1in]
    \cline{2-2} & \\[1in]
    \cline{2-2} & \\[1in]
    \cline{2-2} & \\[1in]
    \hline
  \end{tabular} \bigskip

  \noindent \textbf{Printed output}

\newpage
\class{\ready Function Abstraction}

\subsection*{Setup}
\begin{itemize}
\item Reminder: HW due Wednesday with quiz: functions (tracing with
  stack) and loops
\item Project 1 due today (any time today).
\end{itemize}

\subsection*{Why functions?}

Get them to discuss why this is a worthwhile feature.  Why define
functions?  Pair \& share.

From Downey:

\begin{itemize}
\item Creating a new function gives you an opportunity to name a group
  of statements, which makes your program easier to read and debug.
\item  Functions can make a program smaller by eliminating repetitive code.
  Later, if you make a change, you only have to make it in one place.
\item Dividing a long program into
  functions allows you to debug the parts one at a time and then
  assemble them into a working whole.
\item Well-designed functions are often useful for many programs.
  Once you write and debug one, you can reuse it.
\end{itemize}

\subsection*{Function abstraction}

Start with this code to play a simple guessing game (see
\verb|code/hilow.py|):

\begin{verbatim}
import random

repeat = "yes"
while repeat == "yes":
    print("This is the game of HIGH and LOW. The computer will pick a")
    print("random number, and you will need to guess if the next number")
    print("picked by the computer will be higher than the current")
    print("number. You get 10 chances to guess.")
    count = 0
    correct = 0
    r = random.randint(1, 10)
    while count < 10:
        print(f"The current number is {r}.")
        guess = input("Do you think the next number will be higher (yes or no)? ")
        next_r = random.randint(1, 10)
        print(f"The next number is {next_r}.")
        if (guess == "yes" and next_r > r) or (guess == "no" and next_r <= r):
            correct += 1
            print("You were right!")
        else:
            print("Sorry, you were wrong.")
        r = next_r
        count += 1
    print(f"You got {correct} out of {count} guesses correct!")
    repeat = input("Do you want to play again? ")
print("Thanks for playing!")
\end{verbatim}

Get them to make suggestions, work on decomposing it into functions.
Things to emphasize along the way:

\begin{itemize}
\item Functions are for decomposing tasks \emph{hierarchically}.
  Functions aren't for chaining: \emph{i.e.}\ a function should \emph{not}
  call ``the thing that happens next'' as the last line. Can show them
  something like this example:
\begin{verbatim}
# BAD

def f1():
    do some stuff
    f2()

def f2():
    do more stuff
    f3()

def f3():
    yet more stuff
\end{verbatim}
  Instead it should be decomposed like so:
\begin{verbatim}
# GOOD

def main():
    f1()   # first step
    f2()   # second step
    f3()   # third step

def f1():
    do some stuff
    # f1 doesn't know or care what comes next after it.
    # That's main's job.

def f2():
    do more stuff

def f3():
    yet more stuff
\end{verbatim}
\item Managing information flow is very important.  Decide which
  function ``owns'' a variable.  To get it to another function---give
  it to the function as an argument.  To get a value back---return it,
  then save it in a variable.
\end{itemize}

\newpage
\class{Strings}

\subsection*{Setup}
\begin{itemize}
\item Collect HW (function/loop tracing).
\item \notready Quiz: function tracing
  (\texttt{csci150-instructors/quiz/function-tracing.py}).  Draw
  template on board to remind them, then put code on projector.
\end{itemize}

\begin{reflect}{Brent}
  Note, this continues on Friday, so it's OK if we don't quite get
  through everything, or go fast and have time to start doing more
  stuff from the beginning of Friday etc.
\end{reflect}

We know about strings, but we can't actually do anything with them yet.
\begin{itemize}
\item Strings are \emph{sequences of characters}.

\item We can reference individual characters in a string by their
  \emph{index} using brackets: \verb|mystring[3]|.  Note that
  \emph{indices start at 0}.

\item We can calculate the length of a string with \verb|len()|.
  (Check: what is the relationship of \verb|len| to valid indices?)

\item String slices: \verb|mystr[a:b]|.  Play around a bit to figure
  out what they do.
  \begin{itemize}
  \item Get the characters starting at index $a$, up to \emph{but not
      including} index $b$.  (This seems weird---at first you will just
    have to memorize this fact.  There are good reasons for it though,
    makes a lot of string indexing more elegant.)
  \item Can leave off first, or second (or both, though that is
    useless).
  \item Negative indices count backwards from the end.
  \end{itemize}

\item We can \emph{concatenate} strings, \emph{i.e.} glue two strings
  into one, with \verb|+|.

\item \verb|*| does repetition.

\item Strings have \emph{methods}, i.e. functions you can call on
  strings.  Do \verb|help(str)|, \verb|help(str.upper)| etc.

\item Point out some useful methods: \verb|upper|, \verb|count|,
  \verb|replace|, \verb|find|, \verb|isdigit|

\end{itemize}

(If time, do more examples from Friday.)

\newpage
\lab{Mutation is the Word (Doublets)}

\begin{reflect}{Brent S'16}
  This lab is a lot harder than the previous labs.  Definitely worth
  warning them to start early, emphasizing that they probably won't
  finish within the alloted lab time, can come and ask questions, etc.

  One of the biggest confusions typically has to do with managing
  information flow.  That is, now that we are using functions for
  everything, variables are local, so need to pass things as
  parameters and return values back.  I keep thinking about whether
  there is a gentler way to introduce this but I'm not sure.  If we
  tried to talk about ``managing information flow'' explicitly it
  would just go over their heads until they had actually grappled with
  it in a real program.  So perhaps the shock is unavoidable.

  However, see \textbf{(CRAZY?) IDEA} in reflections above, after
  coding bat lab.  Maybe switch loops and functions, so the doublets
  lab is just writing top-level code with a bunch of loops---really
  focusing on just the loops.  Then the next lab will get them to
  think about abstracting functions from existing code.
\end{reflect}

\begin{reflect}{Brent, S'17}
  The lab went a lot better this time (though it was still hard!),
  after (1) prefacing it with the ``Guess My Number'' lab which gave
  them practice writing while loops, and (2) going over the process of
  function abstraction in class, and encouraging them to first get a
  working doublets program and THEN abstract it out into functions.
\end{reflect}

\begin{reflect}{Brent, S'18}
  Used a similar approach as last time, and it seemed to go well.
\end{reflect}

\begin{reflect}{Gabe, F'18}
  I think [the lab] would benefit from some restructuring, along the
  following lines:
  \begin{itemize}
  \item Step 1: Pseudocode.
  \item Step 2: Basic implementation. No mention of functions or error
    handling. Just get it working for the case of the perfectly
    behaved user.
  \item Step 3: Incorporate error handling.
  \item Step 4: Refactor into functions.
  \end{itemize}
  I explained it this way verbally, which was helpful I think, and
  perhaps it would be easier for them to follow if we made it
  explicit. We might also consider refactoring into functions after
  the basic implementation but before the error handling.
\end{reflect}

\begin{reflect}{Brent, F'19}
  Lab has now been restructed as above and seemed to go very well.
  Everyone seemed to be making productive progress.
\end{reflect}

\newpage
\class{\ready More string \& loop practice; sentinel loops}

\subsection*{Setup}
\begin{itemize}
\item \ready Show them HW: CodingBat string \& loop practice
\item Show them Python Tutor.
\end{itemize}

Today's focus is on getting the students to do more hands-on exercises
writing code using strings and loops, and to introduce the concept of
a sentinel loop.

\subsection*{While loop and string practice}

First, write a function to ``explode'' a string, \emph{i.e.} print out
each character on its own line.  Have the students write it together
and then type it up collaboratively.

\begin{verbatim}
def explode(s: str):
    i: int = 0
    while i < len(s):
        print(s[i])
        i += 1
\end{verbatim}

Even better if they make some bugs in the above, run it and see what
is wrong.

Next, recall \verb|int_input| function from a previous lab: it
repeatedly prompts the user for a number until they actually enter a
number, and returns it.  Get their suggestions on writing it.  They
will probably suggest something like this:

\begin{verbatim}
# Keep prompting the user for a number until they enter a valid one,
# then convert to an int and return it.
def int_input(prompt: str) -> int:
    user_input: str = input(prompt)
    while not user_input.isdigit():
        print("Please enter a number!  Try again. ")
        user_input = input(prompt)

    return int(user_input)
\end{verbatim}

Point out that this version of \texttt{int\_input} works, but it has a
duplicated line of code. We can do better using a ``sentinel loop'':

\begin{verbatim}
# Keep prompting the user for a number until they enter a valid one,
# then convert to an int and return it.
def int_input(prompt: str) -> int:
    valid: bool = False
    while not valid:
        user_input: str = input(prompt)
        if not user_input.isdigit():
            print("Give me some digits!")
        else:
            valid = True

    return int(user_input)
\end{verbatim}

The code for this version of \texttt{int\_input} is a little longer,
but it's better for two reasons: it allows us to avoid duplicating
code, and it also makes it much easier to generalize to more
complicated conditions.

Note that \texttt{int("-6")} works fine but our function does not
allow it.  Challenge them to modify \texttt{int\_input} to allow this
as well.  Eventually should come up with something like this:

\begin{verbatim}
# Keep prompting the user for a number until they enter a valid one,
# then convert to an int and return it.
def int_input(prompt: str) -> int:
    valid: bool = False
    while not valid:
        user_input: str = input(prompt)
        if (user_input[0] == '-' and user_input[1:].isdigit()):
            valid = True
        elif user_input.isdigit():
            valid = True
        else:
            print("That's not a number!")

    return int(user_input)
\end{verbatim}

% \subsection*{DNA HW intro}

% Show them HW, do a bit of it together to show them what they should
% do.  Make sure they understand that they should do everything in a
% notebook.

\subsection*{More string and loop practice}

\begin{reflect}{Brent, F'19}
  We did not actually get to any of these; just leaving them here in
  case they come in useful later.
\end{reflect}

Some more potential practice functions.  Get them to write these on
paper with partners, then go over them (perhaps even have students
come up to type them in).

\begin{verbatim}
# Compute the sum 1 + 2 + ... + n.
def sumUpTo(n: int) -> int:
    s: int = 0
    i: int = 1
    while i <= n:
        s += i
        i += 1
    return s
\end{verbatim}

\begin{verbatim}
# Keep prompting the user for a number, and stop when they enter
# zero.  Then print the sum of all the numbers the user entered.
def sumUserNums():
    s: int = 0
    n: int = 1
    while (n != 0):
        n = int(input("Please enter a number (0 to stop): "))
        s += n
    print("The sum is: " + str(s))
\end{verbatim}


\begin{verbatim}
# Print a triangle of asterisks of height n, like
#
# *
# **
# ***
#
# etc.  (Hint: build a string for each row and then print it.)
def triangle(n: int):
    rowCount: int = 1
    while rowCount <= n:
        row: str = ''
        i: int = 1
        while i <= rowCount:
            row += '*'
            i += 1
        print(row)

        rowCount += 1
\end{verbatim}


\newpage
\class{\ready Lists I}

\subsection*{Setup}
\begin{itemize}
\item String quiz on Wednesday.
\end{itemize}

\subsection*{Lists}

Lists are a lot like strings, but hold arbitrary data instead of
characters.
\begin{itemize}
\item \verb|[]| is the empty list, like \verb|""| is the empty string.
\item Can hold numbers, strings, anything you want.  Show
  \verb|[1,2,3]| syntax.  Show \verb|type| on lists.
\item Elements are ordered, in a sequence.  Indexed from zero.
\item Show \verb|list(range(n))|.
\item \verb|+|, \verb|*|, \verb|len()|, slices all work the same.
\end{itemize}

\subsection*{Mutating lists}

Strings are \emph{immutable}, lists are \emph{mutable}.

\begin{verbatim}
animals = ['cat', 'dog', 'bird']
animals[1] = 'lemur'
print(animals)
\end{verbatim}
Draw a stack diagram of the above situation.

Other list mutation methods:
\begin{itemize}
\item \verb|t.append(element)| for adding to end
\item Removal:
  \begin{itemize}
  \item \verb|t.pop(index)| by index, returns deleted elt
  \item \verb|t.remove(element)| by element
  \item \verb|del t[index]| by index, doesn't return anything BUT
	\verb|del t[slice]| works too
  \end{itemize}
\end{itemize}

\begin{reflect}{Brent, S'17}
  I pared this down even more---I didn't even show them \verb|pop|,
  \verb|remove|, or \verb|del|, just showed them that they exist and
  told them to look them up.
\end{reflect}

Talk about the type of a list. Python lets you do this:
\begin{verbatim}
weird_list = [3, "hi", 2.5, ["one", "two"]]
\end{verbatim}
but you shouldn't!  There are better ways to put things of different
types together (objects), we will learn about that later in the
semester.  A list should only hold things that are all of the same
type. However, notice that if we ask for the \verb|type| of a list
Python only says \verb|list|.  If we say \verb|from typing import *|
then we can use \verb|List| which lets us specify the type of the
items in the list:

\begin{verbatim}
def sum_list(nums: List[int]) -> int:
\end{verbatim}

Now have them implement the \verb|sum_list| function.

Next, have them implement this:
\begin{verbatim}
# Repeatedly prompt the user for an input until they type 'quit'.
# Then return the list of all the inputs the user gave (not including
# the word 'quit').
def read_inputs() -> List[str]:
\end{verbatim}

Might be some difficulty with including a stray empty string at the
beginning, or including the word \verb|"quit"|.  Use this as another
opportunity to emphasize the use of sentinel loops.

\begin{reflect}{Brent, S'17}
  First I had them try to write \verb|explode_list| with partners,
  then we wrote it all together.  Then, I had them work on
  \verb|read_inputs|, then got a volunteer to come up and type while
  the class told them what to do.  This worked super well!
\end{reflect}

\begin{reflect}{Brent, F'19}
  I left this out since I just had them write \verb|explode_string|
  recently.
\end{reflect}

\begin{verbatim}
from typing import *

# Input: items (list)
# Print the items in the list, one per line.
def explode_list(items: List[str]):
    n = 0
    while n < len(items):
        print(items[n])
        n += 1
\end{verbatim}

\begin{reflect}{Brent S'16}
  This is the version from S'16, using a sentinel loop:
\end{reflect}

\begin{verbatim}
# Inputs: none
# Output: a list of strings entered by the user
#
# Repeatedly prompt user for inputs until they type
# 'quit', then return a list of all inputs.
def read_inputs():
    inputs = []
    done = False
    while not done:
        word = input('Enter another input, or quit: ')
        if word.lower() == 'quit':
            done = True
        else:
            inputs.append(word)

    return inputs
\end{verbatim}

\begin{reflect}{Brent, S'17}
  And these are the versions from S'17.  Note they don't use sentinel
  loops.
\end{reflect}
\begin{verbatim}
# Inputs: none
# Output: list of strings entered by the user
#
# Repeatedly prompt the user for inputs until
# they type 'quit', then return a list of all their inputs.
def read_inputs():
    answer = input("What would you like to list? ")
    stuff = []
    while answer != "quit":
        stuff.append(answer)
        answer = input("What would you like to list? ")
    return stuff

def read_inputs2():
    answer = ''
    stuff = []
    while answer != "quit":
        answer = input("What would you like to list? ")
        if answer != 'quit':
            stuff.append(answer)
    return stuff
\end{verbatim}

\newpage
\class{\ready Lists II}

\subsection*{Setup}

\begin{itemize}
\item \ready \textbf{Quiz} (strings): see
  \texttt{csci150-instructors/quiz/string.py}
\item Quiz retakes!
\end{itemize}

\subsection*{Lists and strings}

First, a small aside.  Note that strings and lists of characters are
not actually the same.

\begin{itemize}
\item To convert a string to a list, you can use either:
  \begin{itemize}
  \item the \verb|list| function (every character is its own list
    item), or
  \item the \verb|split()| method.
\begin{verbatim}
  s = "banana"
  t = s.split("n")
  u = list(s)
\end{verbatim}
    Yields \verb|t = ["ba", "a", "a"]| and
    \verb|u = ['b', 'a', 'n', 'a', 'n', 'a']|.  \verb|split()| with no
    arguments splits on whitespace.  For example,
    \[ \verb|"hello      there CSCI     150!".split()| \]
    yields
    \verb|['hello', 'there', 'CSCI', '150!']|.

  \end{itemize}
\item To convert a list to a string, use the \verb|join()| method on
  the delimiter (which can be the empty string).
\begin{verbatim}
  t = ['fun', 'sun', 'sub', 'rub']
  delimiter = " -> "
  delimiter.join(t)

  ls = ['This is a line', 'This is another line', 'A third line']
  '\n'.join(ls)  # Turn a list of lines into one big string
\end{verbatim}
  \verb|''.join(...)| is the inverse of \verb|list(...)|; for example,
  \[ \verb|''.join(list('banana')) == 'banana'|. \]
\end{itemize}

\subsection*{List appending practice}

Have students find a partner. Suppose \verb|animals| is a list of
strings, how can you append 3 copies of \verb|"gerbil"| onto the end?
That is, suppose we have
\begin{verbatim}
animals = ['dog', 'cat', 'horse']
\end{verbatim}
We want to run some code so that afterwards \verb|animals| will
contain the list
\begin{verbatim}
['dog', 'cat', 'horse', 'gerbil', 'gerbil', 'gerbil']
\end{verbatim}
Put the students in groups, try to come up with as many different ways
as you can.  Try to also come up with some ways that seem like they
might work but don't!

Get groups to report out.  Each group share some proposed code, but
don't tell us whether it is supposed to work or not!  We can all
discuss and then try it and see.

This might take a while.  Should naturally lead to discussing
\verb|append|, list repetition operator \verb|*|, concatenation
operator \verb|+|, etc.

\begin{reflect}{Brent, S'16}
  As an illustration, here are all the suggestions my students came up
  with:
\begin{verbatim}
animals = ['skua', 'killer whale', 'spider', 'orangutan', 'goat']

# Ways to add 3 gerbils to the end of the list

# Doesn't quite work:
# animals.append('gerbil ' * 3)

# Runtime error:
# animals.append('gerbil') * 3

# Works!  (But only assuming len(animals) was 5 to start.)
# while len(animals) < 8:
#    animals.append('gerbil')

# Runtime error (too many arguments to append):
# animals.append('gerbil', 'gerbil', 'gerbil')

# Works!
# animals.append('gerbil')
# animals.append('gerbil')
# animals.append('gerbil')

# Same as 'gerbil' * 3
# animals.append('gerbil' + 'gerbil' + 'gerbil')

# Gives the right list but doesn't change animals
# animals + ['gerbil', 'gerbil', 'gerbil']

# Works!
# animals = animals + ['gerbil', 'gerbil', 'gerbil']

# Works!
# gerbil_list = ['gerbil']
# animals = animals + gerbil_list * 3

# Almost works but not quite.
# animals.append(['gerbil'] * 3)
\end{verbatim}

  I think this is a really great exercise. It really helps develop
  their intuition and their grasp of the details.
\end{reflect}

\begin{reflect}{Brent, S'17}
  The exercise was not as useful/fun this time around.  The students
  made fewer interesting mistakes, which I suppose is a good thing.
  In retrospect maybe I should have deliberately introduced more
  incorrect versions for them to think about.
\end{reflect}

\begin{reflect}{Brent, F'19}
  It went really well this time, after I changed it so that (1) groups
  are specifically instructed to try to come up both with things that
  work as well as things that don't, (2) groups could share either one
  without revealing which it is supposed to be.
\end{reflect}

\subsection*{More list practice}

\textbf{If} there's class time left (I don't plan to have any left),
could have them practice writing one or more of these functions.
Otherwise we will do more list function writing practice on Friday.

\begin{verbatim}
from typing import *

# Find the sum of a list of integers.
def listsum(nums: List[int]) -> int:
    i = 0
    s = 0
    while i < len(nums):
        s += nums[i]
        i += 1
    return s

# Find the product of a list of integers.
def listprod(nums: List[int]) -> int:
    i = 0
    p = 1
    while i < len(nums):
        p *= nums[i]
        i += 1
    return p

# Find the maximum value in a nonempty list of
# integers.
def listmax(nums: List[int]) -> int:
    m = nums[0]
    i = 1
    while i < len(nums):
        if nums[i] > m:
            m = nums[i]
        i += 1
    return m
\end{verbatim}

\newpage
\lab{Todo manager}

\begin{reflect}{Brent, F'19}
  Lab seemed to go really well this semester.  It was exhausting since
  I spent the entire time going from group to group helping---as soon
  as I finished helping one group there was always another group
  wanting help.  But everyone seemed to be making good productive
  progress.
\end{reflect}

\newpage
\class{\notready Lists III}

\subsection*{Setup}

\begin{itemize}
\item \notready Practice exam is posted on the class website.  Talk
  about some good study strategies.  Study \emph{for} the practice
  exam, then take it under test conditions!  Come in on Monday ready
  to go over it with other students.
\end{itemize}

Have them work in pairs (randomly assigned?) to write some list
practice functions.  Pick some of the following.

\begin{verbatim}
from typing import *

# Find the sum of a list of integers.
def listsum(nums: List[int]) -> int:
    i = 0
    s = 0
    while i < len(nums):
        s += nums[i]
        i += 1
    return s

# Find the product of a list of integers.
def listprod(nums: List[int]) -> int:
    i = 0
    p = 1
    while i < len(nums):
        p *= nums[i]
        i += 1
    return p

# Find the maximum value in a nonempty list of
# integers.
def listmax(nums: List[int]) -> int:
    m = nums[0]
    i = 1
    while i < len(nums):
        if nums[i] > m:
            m = nums[i]
        i += 1
    return m

# Find the index of the given string in the list, or return -1 if the
# list does not contain the string.
def listfind(search_term: str, lst: List[str]) -> int:
    i = 0
    while i < len(lst):
        if lst[i] == search_term:
            return i
        i += 1
    return -1

# Return a new list which contains only the even numbers from the
# input.  For example, onlyevens([1,5,6,2,3,3,8]) should return [6,2,8].
def onlyevens(lst: List[int]) -> List[int]:
    result: List[int] = []
    i: int = 0

    while i < len(lst):
        if lst[i] % 2 == 0:
            result.append(lst[i])
        i += 1

    return result
\end{verbatim}

\newpage
\class{\notready For loops}

\subsection*{Setup}
\begin{itemize}
\item Reminder: project 2 due one week from Friday.
\end{itemize}

\subsection*{For Loops}

Recall our \verb|explode| function from a week or two ago (or just
about every list- or string-processing function we have written ever):
\begin{verbatim}
def explode(s: str):
  i: int = 0
  while i < len(s):
    print(s[i])
    i += 1
\end{verbatim}
Recall how easy it was to make errors like starting index at the wrong
place, using \verb|<=| instead of \verb|<|, and so on.  Well, it turns
out there is a better way!
\begin{verbatim}
def explode(s: str):
  for c in s:
    print(c)
\end{verbatim}
Simply loops through each character in the string one at a time.  Each
time through the loop the next character is assigned to \verb|c|.
This is way better than using a \verb|while| loop since it rules out
errors putting the starting or ending indices at the wrong number.

This also works for lists:
\begin{verbatim}
animals: List[str] = ["cow", "chicken", "pig", "rabbit"]
for animal in animals:
  print("And on this farm he had a " + animal + "!")
\end{verbatim}

General form:
\begin{verbatim}
for <new variable> in <list expression>:
    ... do stuff with variable ...
\end{verbatim}

The loop will execute a number of times equal to the length of the
list.  Each time, \verb|<new variable>| will be assigned the next
element of the list.

Have students talk in groups to come up with questions.  Get one
question from each group.  Some of the things that came up in F'18
included:

\begin{itemize}
\item Can you have if statements etc. inside a for loop?  Answer: Of
  course.  Show an example.
\item What is actually happening under the hood to make this possible?
  Answer: it's really just a shorthand for the usual sort of while
  loop we would write.
\item What happens if you change the list while looping over it?
  (Answer: who knows, don't.)
\item Can you use the variable after the end of the loop? (Answer:
  yes; don't.)
\end{itemize}

\subsection*{For loop practice}

Write this function together:
\begin{verbatim}
# Count the number of times c occurs in s.  For example,
# count('banana', 'n') = 2
# count('banana', 'z') = 0
def count(haystack: str, needle: str):
    num_occurrences: int = 0
    for c in haystack:
        if c == needle:
            num_occurrences += 1
    return num_occurrences
\end{verbatim}

Then give them this to practice writing in groups:
\begin{verbatim}
# Return a new list with only the strings of length 5 or more.
# e.g. filter_long(['hi','there','xyz','abcdefg']) = ['there','abcdefg']

def filter_long(strs: List[str]) -> List[str]:
    result: List[str] = []
    for s in strs:
        if len(s) >= 5:
            result.append(s)
    return result
\end{verbatim}

\begin{reflect}{Brent, F'19}
  In F'19 we had time for another one so I also had them do this one:
\end{reflect}

\begin{verbatim}
# Say whether the list contains the int we're looking for
# e.g. contains([1,2,3,4], 3) = True
#      contains([1,2,3,4], 7) = False
def contains(haystack: List[int], needle: int) -> bool:
    for n in haystack:
        if n == needle:
            return True
    return False
\end{verbatim}

\newpage
\class{\notready More for loops: index ranges}

\subsection*{Setup}
\begin{itemize}
\item Reminder: project 2 due one week from Friday.
\item Hand back exams.  Announce next checkin meetings.
\end{itemize}

Show them, or recall, an example like this:
\begin{verbatim}
animals = ['sheep', 'horse', 'cow', 'pig']

for a in animals:
    a += '!'
\end{verbatim}

It looks like the above code adds an exclamation point to every animal
in the list, but it doesn't.  Trace the execution on the board to see
why: the variable \verb|a| changes but not the actual strings in the list!

What if we \emph{wanted} to change the animals?  We could do it with a
while loop of course, but is there a way to do it with a for loop?

Yes, there is, but it's a bit more indirect: we have to get access to
the indices, since the only way to change an element in a list is to
say something like \verb|animals[2] = ...| We can do this using
\verb|range|.

\begin{itemize}
\item Show them \verb|range|.  Type \verb|range(4,10)| at the prompt;
  doesn't evaluate.  We can turn it into a list with \verb|list| to
  see what it is: \verb|[4, 5, 6, 7, 8, 9]|.  Just like a slice: start
  at first index, up to but not including the last.
\item If we only give one index, it's like a slice without the first
  index: start at zero, go up to but not including the last.
  e.g. show \verb|range(10)|.
\item Hence we can get all the indices of a list with
  \verb|range(len(lst))|.
\end{itemize}

Like so:
\begin{verbatim}
for i in range(len(animals)):
    animals[i] += '!'
\end{verbatim}

There are other reasons we might want to talk about the indices
explicitly:
\begin{itemize}
\item To use or remember an index for later
\item To loop over two lists at the same time
\item To compare items that are next to each other
\end{itemize}

We'll see examples of all of these.  Let's practice.  Do one together first:
\begin{verbatim}
# Return the index of the first occurrence of character target in s, or -1 if it
# is not found.
def find(s: str, target: str):
    for i in range(len(s)):
        if s[i] == target:
            return i
    return -1
\end{verbatim}
Notice how we needed the index so we could return it!  This won't
work (maybe show this one first to show why it doesn't work):
\begin{verbatim}
def find(s: str, target: str):
    for c in s:
        if c == target:
            return ???  # What do we return here?
    return -1
\end{verbatim}

Now have them do one:
\begin{verbatim}
# Return a list of all the indces where target occurs in s.
def find_all(s: str, target: str) -> List[int]:
    all_the_things: List[int] = []
    for i in range(len(s)):
        if s[i] == target:
            all_the_things.append(i)
    return all_the_things
\end{verbatim}

Another one, give them this as a challenge and let them make lots of
interesting mistakes, then go over the solution:
\begin{verbatim}
# Return true if the list is sorted from smallest to biggest.
# e.g.  is_sorted([2,5,6,9,9]) = True
#       is_sorted([2,5,3,9,9]) = False

def is_sorted(nums: List[int]) -> bool:
    for i in range(len(nums) - 1):
        if nums[i] > nums[i+1]:
            return False
    return True
\end{verbatim}

Talk about how we also need indices for this one, because otherwise we
only get access to one element at a time, can't ask about neighboring
elements.  Also note why we do \verb|range(len(nums) - 1)|, so the
\verb|nums[i+1]| doesn't throw an out-of-bounds error.

% \begin{verbatim}
% def nos(s):
%     news = ""
%     for c in s:
%         if c != "s":
%             news += c
%     return news

% def sum(t):
%     total = 0
%     for n in t:
%         total += n
%     return total

% def mean(t):
%     total = 0
%     for n in t:
%         total += n
%     return float(total) / len(t)

% def mean2(t):
%     return float(sum(t)) / len(t)

% def vacation(s):
%     for i in range(len(s) - 1):
%         if s[i] == s[i + 1]:
%             return True
%     return False

% def classify(strs):
%     for s in strs:
%         print ('*' * len(s))

% def powersof2(n):
%     the_powers = []
%     for i in range(n):
%         the_powers.append(2 ** n)
%     return the_powers

% def triangle(n):
%     total = 0
%     for k in range(1,n+1):
%         total += k
%     return total
% \end{verbatim}


% \subsection*{For loops: reading exercises}

% CLOSE COMPUTERS.
% Find bit-partners. Different one for each exercise.
% Go over each and try it.

% \begin{verbatim}
% def oogie(m: List[int]) -> int:
%     p = m[0]
%     for g in m:
%         if g > p:
%             p = g
%     return p

% def yayaya(q: List[str]) -> bool:
%     for y in range(len(q) - 1):
%         if q[y] == q[y + 1]:
%             return True
%     return False
% \end{verbatim}

% Then have them write \verb|is_sorted| function.

% Additional practice:

% \begin{verbatim}
% def pulu(r: int) -> List[int]:
%     b = []
%     for k in range(r):
%         b.append(k ** 3)
%     return b
% \end{verbatim}

% \begin{reflect}{Brent, S'17}
%   Made it through all of these (including writing \verb|is_sorted|);
%   in fact, we started by finishing \verb|find(c,s)| from last class
%   (which they hadn't quite gotten to work yet).  I think it was
%   a really useful exercise.
% \end{reflect}

% \begin{reflect}{Brent, S'18}
%   Didn't make it through as many due to some technical hiccups, but
%   had them read oogie and yayaya, then had them write
%   \verb|is_sorted|.  Will need to spend a bit of time going over
%   \verb|is_sorted| next class.
% \end{reflect}

\lab{\notready Caesar's Secrets}

\begin{reflect}{Brent, S'17}
  This lab seemed to go well, though it's definitely hard.  Most
  students were not finished by the end of the lab period.
\end{reflect}

\begin{reflect}{Brent, S'18}
  Only had time to introduce for loops at the very end of the lecture
  right before the lab.  So spent some time at the beginning of the
  lab doing more exercises writing for loops, with mixed success (I
  think it went better in Mark's lab than in mine).  In the end most
  students were making progress on the lab by the end; a few had
  finished; a few were just getting encrypt/decrypt to work.
\end{reflect}

\begin{reflect}{Brent, F'18}
  Much, much better this year, with more time spent on lists and for
  loops leading up to it!
\end{reflect}

\newpage

\class{\notready Mutability, objects, and references}

\subsection*{Setup}

\begin{itemize}
\item Assign HW: for loop reading practice (due Wednesday)
\item Quiz Wednesday on for loops.
\item Project 2 due today!
\end{itemize}

Warm-up: what happens here?  Find partner \& decide.
\begin{verbatim}
nums  = [1,2,3]
nums2 = nums
nums[1] = 0
print(nums2)
\end{verbatim}

Try it at python prompt.  This is really weird!

As hinted before, the fact that lists are mutable (as opposed to
strings, which are immutable) opens a huge can of worms.  Let's start
by pulling back the curtain a little bit.  Up until now we have talked
about variables having values, but we need to be more precise.  Draw
stack diagram with variables referencing two different lists.

\begin{itemize}
\item Variables actually don't contain values, they contain
  \emph{references} to objects.  A reference ``points'' to a certain
  object (really, a reference is just a number which represents a
  location in the computer's memory).
\item Copying the value of a variable into another variable just
  copies the \emph{reference} (\emph{i.e.} the memory address), so they now
  point to the same object!
\item Variables which reference objects that have the same value are
  \emph{equal}.  You can test equality with \verb|==|.
\item Variables which reference the \emph{same object} are
  \emph{identical}. You can test identity with \verb|is| (though you
  won't need this in the rest of the course).
\item Note since strings, ints, floats, and booleans are immutable, it
  didn't matter before; there is no way to tell the difference between
  two variables with separate copies of a string and two variables
  referencing the same string.  So we will continue to think of
  variables as holding ints, strings, etc., not references to them
  (since it doesn't matter).  But for lists and other things to come
  it matters quite a bit.
\end{itemize}

Do this at the REPL, with accompanying pictures drawn on the board:
\begin{verbatim}
>>> a = [1,2,3]
>>> b = [1,2,3]
>>> a == b
True
>>> a is b
False
>>> a[0] = 5
>>> a
[5, 2, 3]
>>> b
[1, 2, 3]
>>> a == b
False
>>> a = b
>>> a
[1, 2, 3]
>>> b
[1, 2, 3]
>>> a == b
True
>>> a is b
True
>>> a[0] = 5
>>> a
[5, 2, 3]
>>> b
[5, 2, 3]
\end{verbatim}
Note in particular that the assignment \verb|a = b| does not copy the
value of \verb|b| into \verb|a|, but just makes \verb|a|
\emph{reference the same thing as} \verb|b|.

Some related things:
\begin{itemize}
\item What if you really \emph{do} want to make a \emph{copy} of a
  list, so modifying it doesn't modify the original?  You can write
  \verb|some_list[:]|.
\item Be careful about methods that modify vs functions that make a
  new thing.  \eg \verb|list.sort()| modifies the list.  But
  \verb|sorted(list)| makes a new list!  And of course
  \verb|string.upper()| makes a new string because strings are
  immutable.  Confusing, don't memorize: the point is you should be
  careful and look up help to see what things do.  Show
  \verb|help(list.sort)|, \verb|help(sorted)|, \verb|help(str.upper)|.
\end{itemize}

[INSERT OPTIONAL RANT ABOUT MUTATION HERE]

\subsection*{Tracing with stack and heap}

This has implications for function parameters as well.  Consider:

\begin{verbatim}
def release(animals: List[str]):
  while len(animals) > 2:
    animals.pop(0)

print(animals)
release(animals)
print(animals)
\end{verbatim}

Formally, to see what's going on, we are going to enhance our tracing
diagrams one last time: we are going to add a \emph{heap}.  The
``heap'' is the area of memory where objects are stored. Variables are
stored on the stack and contain references to things in the heap.
Every time we create a new list, we draw it in the heap (anywhere!).

Have them try another example:

\begin{verbatim}
def slicer(nums: List[int]) -> List[int]:
  first3 = nums[:3]
  first3[2] *= 3
  return first3

def main():
  my_nums = [1,2,8,6,7,12]
  my_nums = slicer(my_nums)
\end{verbatim}

Then go over it carefully.

\newpage
\class{\notready Dictionaries}

\subsection*{Setup}

\begin{itemize}
\item Remember HW (for loop reading practice) and quiz Wednesday.
\end{itemize}

\subsection*{Dictionaries}

Recall lists.  Draw an alternate picture of lists as a mapping from
indices to values. For example \verb|['tapir', 'ferret', 'alligator']|
can be thought of as
\begin{verbatim}
0 -> 'tapir'
1 -> 'ferret'
2 -> 'alligator'
\end{verbatim}
$0$, $1$, $2$ are \emph{keys}, animal names are \emph{values}.  A
\emph{mapping} or \emph{association} between keys and values.  A
\emph{dictionary} is the obvious generalization of this to arbitrary
keys instead of just 0, 1, 2 \dots.  For example, suppose we are
keeping these animals in a zoo and we want to keep track of them by
their habitat ID numbers.
\begin{verbatim}
'A23' -> 'tapir'
'B19' -> 'ferret'
'B12' -> 'alligator'
\end{verbatim}
Or we could use numbers as keys.  Important point: keys have to be
immutable.  So we can't use \eg lists or dictionaries as keys.

Show how to make this dictionary.
\begin{itemize}
\item Write using curly braces and colons.
\item Alternatively, start with empty one and add them.
\end{itemize}

Dictionaries are like functions mapping keys to values.  Consequences:
\begin{itemize}
\item There can only be one value per unique key.  (If you want
  multiple values, use a list.)
\item But there can be multiple keys with the same value.
\item You can only look up a value by key, not a key by value.
\end{itemize}

Other things we can do with a dictionary (illustrate these at the
Python prompt and/or with some sample code):
\begin{itemize}
\item Get list of all keys with \verb|keys()|
\item Get list of all values with \verb|values()|
\item Check if a key is in the dictionary using \verb|in|
\item Ask for the number of keys with \verb|len|
\item Delete a mapping with \verb|del|
\item Iterate over keys with \verb|for|
\end{itemize}

Show
\url{https://docs.python.org/3.7/library/stdtypes.html#mapping-types-dict}.

Note (as of Python version 3.6) the keys will always be in the same
order you put them in.

\begin{reflect}{Brent, S'17}
  Only made it to here.  Did frequency counting the following class.
\end{reflect}

Do as much of this as there is time for; finish next class.

Rewrite \verb|frequency_counts(s)| to return a dictionary.  Why this is
better: directly look up by letter value.  No need to use \verb|ord|
or alphabet string etc.

\newpage
\begin{verbatim}
from typing import *

# Given a string s, return a dictionary mapping each character in the
# string to the proportion of times it occurs.
#
# For example,
#   frequency_counts('hello') = {'e': 0.2, 'h': 0.2, 'o': 0.2, 'l': 0.4}

def frequency_counts(s: str) -> Dict[str, float]:
    counts: Dict[str, int] = {}
    for c in s:
        if c not in counts:
            counts[c] = 0
        counts[c] += 1

    freqs: Dict[str, float] = {}
    for c in counts:
        freqs[c] = counts[c] / len(s)

    return freqs
\end{verbatim}

\begin{reflect}{Brent, S'18}
  Made it through writing \verb|frequency_count| (but not
  \texttt{normalize}).
\end{reflect}

\begin{reflect}{Brent, F'18}
  Didn't quite get through writing \verb|frequency_counts|.  Note last
  time I decomposed it in terms of a function
  \verb|normalize(Dict[str,int]) -> Dict[str,float]|, but this time I
  found myself in the moment just writing the version above.  This
  actually makes some sense since we don't have to recompute the sum
  of all counts, we can just divide by \verb|len(s)|.  In a later
  class had the students write \verb|normalize| themselves.
\end{reflect}

\newpage
\class{\notready File I/O, dictionary practice}

Our programs have no state between each time they are called.
Everything is stored in memory during the program, but afterwards,
it's all gone.  This is RAM.  We need hard drive storage to keep it
around.  Save and read things in files.

To write to a file:

\begin{verbatim}
fout = open("myfile.txt", "w")
\end{verbatim}

then to save data in the file, write out as strings:

\begin{verbatim}
fout.write("hello\n")
fout.write("world\n")
fout.close()
\end{verbatim}
Note the newline characters.

To read from a file:

\begin{verbatim}
fin = open("myfile.txt", "r")
\end{verbatim}

Two ways to get stuff out of the file.

\begin{verbatim}
s = fin.read()
\end{verbatim}

which returns a string that includes all the carriage returns, or

\begin{verbatim}
s = ""
for line in fin.readlines():
    s += line.strip()
\end{verbatim}

which strips them out of each line as it is added. \verb|readlines()|
returns a list of strings.

After both, say \verb|fin.close()| to finish accessing the file.

\subsection*{Dictionary practice}

FINISH writing \verb|frequency_count| and \verb|normalize| from last
class.  Maybe run them on the contents of a file.

\newpage
\lab{Sentiment Analysis}

\begin{reflect}{Brent, S'16}
  I wrote this lab fresh in S'16, based on a Nifty Assignment @
  SIGCSE.  It replaced the Gibberish Generator lab from the previous
  semester, which was way too difficult.  Overall I think the new lab
  went over very well.  It seemed to have the right level of
  difficulty and gave the students good practice with dictionaries.
\end{reflect}

\newpage
\class{\notready Exam 2 review}

\begin{itemize}
\item \notready \textbf{Assign} Project 2.  Due Friday, October 1.
\item Put them in pairs and have them go through exam together.
\end{itemize}

\class{\notready Exam 2}

\begin{reflect}{Brent, S'16}
  In the Fall 2015 semester, Exam 2 was on the Wednesday right before
  Fall Break.  In Spring 2016, it was on a Tuesday two weeks before
  spring break. (The ``normal'' thing would have been to have the exam
  review day on Friday, then the exam on Monday, and no lab on
  Tuesday, but I was at SIGCSE on the Friday, so I gave them the
  practice exam over the weekend while I was at SIGCSE, spent Monday
  reviewing the practice exam, and held the exam during the lab slot
  for that week.)
\end{reflect}

\begin{reflect}{Brent, S'17}
  In Spring 2017, Exam 2 was on a Wednesday a week and a half before
  spring break.  For lab the following day (Thursday) I had them start
  on their projects, \emph{i.e.} not leave lab until they had a solid
  project idea.

  As usual this exam was a doozy. Average class score on the exam was
  71, median 73.5, max 100 ($\times 2$), min 21 (ouch).
\end{reflect}

\begin{reflect}{Brent, S'18}
  Again the Wednesday 1.5 weeks before spring break; lab the same \&
  following days again was used to start on project 2.

  Average was much higher this time, 83.  I made it a bit easier by
  not requiring them to *write* any for loops (though they still had
  to read several).
\end{reflect}

\begin{reflect}{Brent, F'18}
  Again on the Wednesday right before fall break.  This time the exam
  didn't even have for loops at all, since we spent more time
  practicing with lists and put off for loops until after the exam.
  Average was back down in the 71--73 range, min 28, max 100---pretty
  much the same distribution as S'17.
\end{reflect}

\begin{reflect}{Brent, F'19}
  Mark's section had a higher average this year (84), mine averaged
  76, still higher than previous years.  Low was a 40.
\end{reflect}

\holiday{Spring break}

\newpage
\section*{\notready Heap tracing examples, dictionary practice}

\notready EXTRA MATERIAL, put this somewhere!!

Go through some more examples of tracing with stack \& heap:

\begin{verbatim}
from typing import *

## This is an example to use to follow tracing with the Heap
## The Heap will contain any lists and dictionaries that are built,
## as their variables are references to the object.  The variables names
## themselves still go in the stack, with an arrow to the Heap.
## Strings, ints, floats will go in the stack

## List version
## change_list will add a few values to a list
def change_list(lst: List[int]) -> List[int]:
    for i in range(4):
        lst.append(i**2)

    lst.append(42)

    return lst

def main():
    mylist : List[int] = [0,5,-2,8]
    print(mylist)
    templist : List[int] = mylist
    print("Change: ", change_list(templist))
    print(mylist)
    templist[1] = -17
    print(mylist)

main()

## String Version
## This works differently.  Since strings are immutable, any time a string
## is modified, a new instance of the string is created.

def change_string(s : str) -> str:
    for i in range(4):
        s += str(i**2)

    s += '42'

    return s

def main1():
    mystring : str = 'abcd'
    print(mystring)
    tempstring : str = mystring
    print("Change: ", change_string(tempstring))
    print(mystring)
    ## We cannot write an equivalent to templist[1] = -17,
    ## since we cannot individually change the value of a single character
    ## in a string

main1()
\end{verbatim}

Then have students write
\verb|normalize(d: Dict[str, int]) -> d: Dict[str, float]|.  Requires
first adding up all counts.  Emphasize that we don't want to modify
the original dict, make a new one to return!

\begin{verbatim}
# Given a dictionary of counts, return a new normalized dictionary
# where each key maps to its frequency, by dividing each count by the
# total count.
def normalize(counts: Dict[str, int]) -> Dict[str, float]:

    total: int = 0
    for k in counts:
        total += counts[k]

    freqs: Dict[str, float] = {}
    for k in counts:
        freqs[k] = counts[k] / total

    return freqs
\end{verbatim}

\newpage
\class{\notready Project 2 demo}

Introduce Project 2 (word games).

\begin{reflect}{Brent, S'18}
  Did this at the beginning of lab instead of using class time on it.
  Handed back exams and then went straight into recursion; this means I
  will have three lectures on recursion and then give them the Friday
  right before spring break off (so I can work on a conference paper
  submission which is due that day).
\end{reflect}

Livecode the beginning of an exemplary Doublets lab.  Things to focus
on:
\begin{itemize}
\item Start with \verb|main|.  Hierarchical decomposition; write stub
  functions.
\item Think about what information each function needs to do its job,
  and what information it returns.
\item Write \verb|get_starting_word| function.
\item Start writing \verb|get_ending_word| function, realize that they
  are going to have a lot of similar code.
\item Abstract common part out into a new function, \verb|get_word|,
  that takes a parameter describing the desired word.
\item Stub out a single round of the game, then talk about adding a
  while loop.
\end{itemize}

\begin{reflect}{Brent, S'17}
  Instead of exam wrap-up I actually covered file I/O here, instead of
  waiting until after spring break as I did in spring 2016. I think
  it was useful for many students for their projects, if they wanted
  to \eg pick random words, generate files with different word
  lists, etc.
\end{reflect}


\newpage
\class{\notready Intro to classes \& objects}

\subsection*{Setup}

\begin{itemize}
\item Reminder: HW on dict tracing due Wednesday, quiz.
\end{itemize}

\begin{reflect}{Brent}
  Totally OK if this doesn't all fit into one lecture (it probably
  won't).  Basically we have 3 lectures to get through \verb|Cake| and
  \verb|TrafficLight|.
\end{reflect}

Introduce the Birthday Cake Problem: you have a cake with $n$ lit
candles.  You choose a uniform random number from $1$ to $n$
(\emph{i.e.} roll an ``$n$-sided die'') and blow out that many
candles.  Now $m$ candles are left; choose a random number from $1$ to
$m$ and blow out that many; and so on.  How many iterations do you
need, on average, to blow out all the candles?  \emph{e.g.} you might
get lucky and choose $n$, and require only one step; or you might get
really unlucky and choose $1$ every time, and require $n$ steps.
Usually it will be something in between.  What's the average?

Play the game a few times (perhaps using \verb|random.randint()| to
choose number of candles).  Get their guesses/intuition: how many
iterations will be required for $10$ candles? For $100$? $1000$? For
$n$ candles in general?

If this were a math class we could analyze and solve this
mathematically (in fact there is a nice solution!).  But instead we're
going to write some code to \emph{simulate} this situation and find
out the answer empirically.

\begin{reflect}{Brent, F'18}
  Considered using new \verb|dataclasses|
  (\url{https://docs.python.org/3/library/dataclasses.html#module-dataclasses})
  which could be nice, but I'm not sure we can yet guarantee that all
  students will have Python 3.7.  Some machines might still have 3.6
  or 3.5.  But I should definitely reconsider this for the spring or
  fall.
\end{reflect}

\begin{itemize}
\item Introduce idea of objects:
  \begin{itemize}
  \item We can package up data into \emph{objects}.
  \item Objects contain some variables, \emph{i.e.} things they
    remember.  Variables in an object are also called \emph{fields}.
  \item Objects have functions, \emph{i.e.} things they can
    do. Functions of an object are also called \emph{methods}.
  \item We have already seen objects: strings, lists,
    dictionaries\dots.  Some object methods we have seen include
    things like \verb|lower()|, \verb|find()|, \verb|append()|, \dots
  \end{itemize}
\item Introduce idea of classes:
  \begin{itemize}
  \item Python \emph{classes} allow us to design our own new kinds of
    objects.
  \item A class is a \emph{template} or \emph{blueprint} for objects
  \item An object is an ``instance of'' a class.  For example,
    we could have a class ``car'', and an object which is an instance
    of that class might be my blue Toyota Matrix.
  \end{itemize}
\item Make an empty \verb|Cake| class.
\item Show how we can make a \verb|Cake| object by writing something
  like \verb|a_cake: Cake = Cake()|.
\item Show how we can set a variable, \verb|num_candles|, inside it:
  \verb|a_cake.num_candles = 4|. Draw a stack/heap diagram:
  \verb|a_cake| variable on the stack, \verb|Cake| object in the heap,
  containing a \verb|num_candles| variable.
\item Create an \verb|__init__| method with a parameter. (Explain
  \verb|self|; explain \verb|__init__|.)
\item Create \verb|blowout| method which picks a random number of
  candles to blow out and subtracts from \verb|num_candles|.
  Emphasize difference between local variable (where we store the
  random number chosen) and \verb|num_candles| which lives in the
  object.
\end{itemize}

\newpage
\class{\notready More classes \& objects}

\subsection*{Setup}
\begin{itemize}
\item Collect HW (dictionary \& heap reading)
\item \notready Quiz (see \texttt{csci150-instructors/quiz/dict-heap-quiz.py})
\end{itemize}

\subsection*{More Cake}

\begin{itemize}
\item Review code from Monday.  Get them to write \verb|allout|.
\item Fill out the rest of the simulation.  Maybe get them to write
  some of the methods.  Below is what I've ended up with in previous
  years.
\begin{verbatim}
import random

class Cake:

    # __init__ is a special method
    #   that gets called when a
    #   new object is created.
    #   i.e. when Cake() is called.
    def __init__(self, num_candles):
        # self is a special first parameter
        # that gets filled in by Python.
        # It is always a reference to the current object.

        # self.candles is a variable in the
        #   object being created
        #   (candles = ... would make a local var)
        self.candles = num_candles

    # Methods = things that all Cakes can do

    def blowout(self):
        n = random.randint(1, self.candles)
        self.candles -= n

    def allout(self) -> bool:
        return (self.candles == 0)

# Blow out candles on the cake until they are all out, and report how
# many iterations it took
def happy_birthday(c: Cake) -> int:
    count: int = 0
    while not c.allout():
        c.blowout()
        count += 1

    return count

# Blow out candles on a lot of cakes and report the average number of
# iterations needed
def average_blows(num_candles: int, trials: int) -> float:
    total: int = 0
    for i in range(trials):
        c: Cake = Cake(num_candles)
        total += happy_birthday(c)
    return total / float(trials)

def main():
    for num_candles in range(100):
        print num_candles, average_blows(num_candles, 10000)

main()
\end{verbatim}
\item Emphasize how the encapsulation lets us work using a higher
  level vocabulary: instead of talking about low-level things
  like variables and picking random numbers and adding and
  subtracting, we can build a vocabulary for talking about doing
  things with birthday cakes, and then we can write our simulation in
  terms of this higher-level vocabulary (``blow out some candles'',
  ``keep looping until all the candles are out'', etc.)

\item Run it and paste output into Excel, do a regression to show that
  it is close to natural log! [In fact there is a nice analytical
  argument showing that the expected number of iterations required to
  blow out $n$ candles is exactly $H_n = \sum_{1 \leq k \leq n} 1/k$,
  the $n$th harmonic number, and
  $\lim_{n \to \infty} H_n = \ln n + \gamma$, where
  $\gamma \approx 0.577216\dots$ is the Euler-Mascheroni constant.
  But there's no need to tell them any of this.]
\end{itemize}

\newpage
\lab{Die Hard/Water Jug}

\begin{reflect}{Brent, S'17}
  This lab seemed to go pretty well this year.  The new step that
  forced them to write a test function helped a LOT!  I know of
  several groups that started out with a buggy implementation of the
  WaterJug class and then ended up debugging it through writing tests
  and getting the tests to pass---precisely what we would hope for
  such an exercise.  We should probably figure out ways of
  incorporating this approach into other labs as well.
\end{reflect}

\newpage
\class{\notready Classes \& objects III}

\subsection*{Setup}

\begin{itemize}
\item \notready Assign HW: class design practice (see
  \texttt{homework/class-design}).  Quiz Wednesday.
\end{itemize}

\subsection*{Traffic lights}

\begin{verbatim}
# Traffic light
#
# How would you model it in Python?
# What functions should it have?
# How would it work?

# Possible functions:
#   - separate red, green, yellow light functions
#   - interval function
#   - sensor
#   - change to next color
#   - say what the current color is
#   - run
# Variables / state:
#   - current color
#   - timer (time left until next change?)


# Version 1:
#   - easy to write
#   - tedious (imagine cycling through 20 colors)
#   - difficult to change
class TrafficLight:

    # Variables:
    #   - current_color (string)

    # Create a new red traffic light
    def __init__(self):
        self.current_color = "RED"

    # Change to next color
    def change(self):
        if self.current_color == "RED":
            self.current_color = "GREEN"
        elif self.current_color == "GREEN":
            self.current_color = "YELLOW"
        elif self.current_color == "YELLOW":
            self.current_color = "ORANGE"
        elif self.current_color == "ORANGE":
            self.current_color = "RED"
        else:
            print "The sky is falling!!"  # This should never happen

    # Return the current color
    def color(self):
        return self.current_color

# Version 2:
# Use a dictionary of colors
class TrafficLight2:

    # Variables:
    #   - current_color
    #   - color_dict

    def __init__(self):
        self.current_color = "RED"
        self.color_dict = \
          { "RED" : "GREEN",
            "GREEN" : "YELLOW",
            "YELLOW" : "ORANGE",
            "ORANGE" : "RED" }

    def change(self):
        self.current_color = self.color_dict[self.current_color]

    def color(self):
        return self.current_color

# Version 3: list of colors
class TrafficLight3:

    # Variables:
    #   - color_list
    #   - current_index

    def __init__(self):
        self.color_list = ["RED", "GREEN", "YELLOW", "ORANGE"]
        self.current_index = 0

    def change(self):
        self.current_index += 1
        self.current_index %= len(self.color_list)

    def color(self):
        return self.color_list[self.current_index]

def main():
    t = TrafficLight3()  # we can switch and it works the same
    print t.color()
    for i in range(10):
        t.change()
        print t.color()

main()
\end{verbatim}

\newpage
\class{\notready Extended example I}

\begin{reflect}{Brent, F'19}
  I did the elevator simulation again, Mark did card games with Uno
  and blackjack.  The specific example doesn't matter that much, as
  long as they see designing and implementing multiple classes working
  together.
\end{reflect}

As a slightly larger example involving several classes working
together let's implement a simple elevator simulation.

Here's the code we wrote in class:

\begin{verbatim}
class Passenger:

    # Variables:
    # current_floor: int
    # destination_floor: int
    # in_elevator: bool
    # weight: int
    # ID: int

    def __init__(self, ID: int, current_floor: int, destination_floor: int, weight: int):
        self.ID = ID
        self.current_floor = current_floor
        self.destination_floor = destination_floor
        self.weight = weight

        self.in_elevator = False

    def __repr__(self):
        return ('Passenger(%d, %d, %d, %d)'
                % (self.ID, self.current_floor, self.destination_floor, self.weight))

    def get_ID(self) -> int:
        return self.ID

    def get_dest_floor(self) -> int:
        return self.destination_floor
\end{verbatim}

\begin{verbatim}
from typing import *
from Passenger import *

class Elevator:

    # going_up (up/down): bool
    # destinations: List[int]
    # total_weight: int
    # weight_limit: int
    # current_floor: int
    # door_open: bool
    # max_floor: int
    # passengers: List[Passenger]

    def __init__(self, max_floor: int):
        self.max_floor = max_floor

        self.going_up = True
        self.current_floor = 1
        self.passengers = []

    # Move one floor in the current direction
    # Let any passengers out who want to get out
    # Turn around if @ top or bottom floor.
    def move(self):
        if self.going_up:
            self.current_floor += 1
        else:
            self.current_floor -= 1

        if self.current_floor == 1:
            self.going_up = True
        elif self.current_floor == self.max_floor:
            self.going_up = False

        new_passengers: List[Passenger] = []
        for p in self.passengers:
            if p.get_dest_floor() == self.current_floor:
                print("Passenger %d gets off" % p.get_ID())
            else:
                new_passengers.append(p)
        self.passengers = new_passengers

    # Let a passenger on the elevator
    def add_passenger(self, p: Passenger):
        self.passengers.append(p)

    # Return the current floor
    def get_current_floor(self):
        return self.current_floor
\end{verbatim}

And finally we made a tester:

\begin{verbatim}
from Passenger import *
from Elevator import *

def main():
    e = Elevator(5)

    for i in range(20):
        e.move()
        print(e.get_current_floor())

main()
\end{verbatim}

Next time we will test passengers getting off the elevator, and make a
Building class.  Then extend the simulation in various ways.

\newpage
\class{\notready Extended example II}

\subsection*{Setup}

\begin{itemize}
\item \notready Quiz: class design.  See \texttt{csci150-instructors/quiz/classes.py}.
\end{itemize}

\subsection*{Extended example, continued}

Here's how the code ended up in Spring 2019:

\newpage
\noindent \texttt{Passenger.py}:

\begin{verbatim}
class Passenger:

    # Variables:
    # current_floor: int
    # destination_floor: int
    # in_elevator: bool
    # weight: int
    # ID: int

    def __init__(self, ID: int, current_floor: int, destination_floor: int, weight: int):
        self.ID = ID
        self.current_floor = current_floor
        self.destination_floor = destination_floor
        self.weight = weight

        self.in_elevator = False

    def __repr__(self):
        return ('Passenger(%d, %d, %d, %d)'
                % (self.ID, self.current_floor, self.destination_floor, self.weight))

    def get_ID(self) -> int:
        return self.ID

    def get_current_floor(self) -> int:
        return self.current_floor

    def get_dest_floor(self) -> int:
        return self.destination_floor

    def get_in(self):
        self.in_elevator = True

    def get_out(self):
        self.in_elevator = False
\end{verbatim}

\newpage
\noindent \texttt{Elevator.py}:

\begin{verbatim}
from typing import *
from Passenger import *

class Elevator:

    # going_up (up/down): bool
    # destinations: List[int]
    # total_weight: int
    # weight_limit: int
    # current_floor: int
    # door_open: bool
    # max_floor: int
    # passengers: List[Passenger]

    def __init__(self, max_floor: int):
        self.max_floor = max_floor

        self.going_up = True
        self.current_floor = 1
        self.passengers = []

    # Move one floor in the current direction
    # Let any passengers out who want to get out
    # Turn around if @ top or bottom floor.
    def move(self):
        if self.going_up:
            self.current_floor += 1
        else:
            self.current_floor -= 1

        print("FLOOR %d" % self.current_floor)

        if self.current_floor == 1:
            self.going_up = True
        elif self.current_floor == self.max_floor:
            self.going_up = False

        new_passengers: List[Passenger] = []
        for p in self.passengers:
            if p.get_dest_floor() == self.current_floor:
                print("Passenger %d gets off" % p.get_ID())
                p.get_out()
            else:
                new_passengers.append(p)
        self.passengers = new_passengers

    # Let a passenger on the elevator
    def add_passenger(self, p: Passenger):
        self.passengers.append(p)
        p.get_in()

    # Return the current floor
    def get_current_floor(self):
        return self.current_floor
\end{verbatim}

\newpage
\noindent \texttt{Building.py}:

\begin{verbatim}
from Passenger import *
from Elevator import *

import random
import time

class Building:

    # Variables:
    # num_floors: int
    # num_elevators: int
    # waiting_passengers: Dict[int, List[Passenger]]
       # key = floor
       # value = passengers waiting on that floor
    # elevator: Elevator
       # (later, List[Elevator])

    def __init__(self, num_floors: int, num_passengers: int):
        self.num_floors = num_floors
        self.elevator = Elevator(self.num_floors)

        self.waiting_passengers = {}

        for i in range(num_passengers):
            p: Passenger = Passenger(i, random.randint(1, num_floors), random.randint(1, num_floors), random.randint(1, 300))

            # Create the list for p's floor if it doesn't already exist
            if p.get_current_floor() not in self.waiting_passengers:
                self.waiting_passengers[p.get_current_floor()] = []

            # Add p to the line of passengers waiting on that floor
            self.waiting_passengers[p.get_current_floor()].append(p)

    # Move the elevator and put waiting people on it
    def update(self):

        # Move the elevator one floor (and let passengers off)
        self.elevator.move()

        # Put people on the elevator
        if self.elevator.get_current_floor() in self.waiting_passengers:
            for p in self.waiting_passengers[self.elevator.get_current_floor()]:
                print("Passenger %d gets on the elevator" % p.get_ID())
                self.elevator.add_passenger(p)

        # No one is waiting on that floor anymore
        self.waiting_passengers[self.elevator.get_current_floor()] = []

    def run(self):
        while True:
            self.update()
            time.sleep(1)
\end{verbatim}

\newpage
\noindent \texttt{Test.py}:

\begin{verbatim}
from Passenger import *
from Elevator import *
from Building import *
import time

def main():
    # e = Elevator(5)
    #
    # e.add_passenger(Passenger(101, 1, 3, 90))
    # e.add_passenger(Passenger(297, 1, 2, 150))
    # e.add_passenger(Passenger(311, 1, 5, 3))
    #
    # for i in range(20):
    #     time.sleep(2)
    #     e.move()
    #     print(e.get_current_floor())

    b = Building(10, 30)
    b.run()

main()
\end{verbatim}

\newpage
\lab{Graphics \& Animation}

\newpage
\class{Extended example III}

\begin{itemize}
\item \notready HW assignment?  Practice with writing/reading code
  involving classes?  Quiz Wednesday?
\end{itemize}

\newpage
\class{\notready Recursion I: intro}

\subsection*{Setup}
\begin{itemize}
\item Write stone-counting rules somewhere, e.g. on board (see below)
\item ASSIGN final project.
\item ASSIGN practice exam (for Friday).
\end{itemize}

\subsection*{Stone counting}

Rules:
\begin{enumerate}
\item You may not look into the bowl or at any pieces.
\item You may only touch one piece.
\item If someone gives you the bowl and asks you how many pieces it
  contains, you must eventually give the bowl back and answer
  truthfully.
\end{enumerate}
(There will usually be some attempts to find loopholes; modify the
rules as necessary.)  Then hand the bowl to a student and ask how many
pieces it contains.

Recursion!  Functions that call themselves.  In a sense you were all
``the same function'' because you were running the same program
(rules).

Factorial function: $n! = 1 \times 2 \times \dots \times (n-1) \times
n$.  Let's implement in Python.
\begin{verbatim}
def fact(n):
    product = 1
    for k in range(1,n+1):
        product *= k
    return product
\end{verbatim}
Now consider $n! = (1 \times 2 \times \dots \times (n-1)) \times n =
(n-1)! \times n$.  Also need to define $0! = 1$.  Can we translate
this directly into Python?
\begin{verbatim}
def fact_rec(n):
    if n == 0:
        return 1
    else:
        result = fact_rec(n-1)
        return result * n
\end{verbatim}
It works!

Show call stack.  Every recursive call has its own copy of local
variables!

Big idea: transform a problem into a slightly simpler version (or more
than one), solve that simpler problem, do some work to compute an
answer to the original problem.

Typically:
\begin{itemize}
\item Base case(s): simple case where we know the answer without doing
  any work.  Always do this first!
\item Recursive case(s): make a recursive call to solve a slightly
  simpler (i.e. closer to the base case) problem.
\end{itemize}

Then re-code factorial function or something else live, adding a bunch
of comments and narrating thought process.

\begin{reflect}{Brent, F'18}
  Actually ended up writing a list product function; someone suggested
  the case with length 1 as the base case.
\end{reflect}

\newpage
\class{\notready Recursion II: recursive functions}

\subsection*{Setup}

\begin{itemize}
\item \notready Quiz on classes/objects reading/tracing?
\end{itemize}

\subsection*{Recursive functions}

Start by reinforcing ``base case + recursive case''.  Talk about
``leap of faith'': trust that recursive calls will work without
thinking about how.  Trust your future self.  DON'T think about
unwinding entire recursion stack.  Just think about ONE level.

% Up until now all the examples we've done have been math functions.
% Let's do a few more ``computer-sciency'' examples processing strings
% and lists.  Note, in Python this is not the best way to implement
% these functions (though in some languages it is).  Just for
% illustration purposes.  If you take 151 you will start to see real,
% legitimate uses of recursion, \eg processing trees.  (The one
% legitimate use of recursion using only what you know is what we will
% be doing in lab---making pretty pictures!)

Finish implementation of list product from last class.
\begin{itemize}
\item Even simpler base case---does the slicing work appropriately?
\item What should the product of an empty list be?
\end{itemize}

\begin{itemize}
\item Have them work in pairs to implement \verb|sum|.
\item Have them work in pairs to implement string \verb|reverse|.
  \begin{itemize}
  \item Go over base case at some point then let them continue working.
  \end{itemize}
\end{itemize}

\begin{reflect}{Brent, S'17}
  Talked about thought process for writing recursive functions,
  re-coded factorial function.  Lots of great questions.  Then looked
  at \texttt{sum} function again which we actually wrote in the
  previous class (in retrospect it was probably premature to do
  that).  Then only got as far as having them work on writing a
  \texttt{product} function on paper.  Ended with a spirited debate
  about what \texttt{product([])} ought to be.  Plan to pick up where
  we left off next time.
\end{reflect}

\begin{reflect}{Brent, F'18}
  They all eventually figured out \verb|reverse| which was good.
\end{reflect}

\newpage
\lab{Enron's Secrets}

\newpage
\class{\notready Recursion III (W 3/13)}

\subsection*{Setup}
\begin{itemize}
\item Collect HW.
\item \notready Quiz --- need to write a HW to go with it.
\end{itemize}

\begin{reflect}{S'19}
  This will work particularly well if we're doing some kind of
  filesystem exploration lab.  But it looks kind of like that's not
  going to happen YET (maybe for next fall).  But it's still fun to
  show them.
\end{reflect}

Introduce idea of trees/hierarchies.  Want to model this in Python.
This won't be on any exam, but it's good to see a more realistic
example of recursion in Python.  In Python I would not actually find
the product of a list or reverse a string etc. using recursion, but I
really would make recursive trees.

\begin{verbatim}
# Trees

class Tree:

    # self.value  --- value @ the root
    # self.children: List[Tree]

    def __init__(self, value, children):
        self.value = value
        self.children = children

    # Return the number of items in the whole tree
    def size(self) -> int:

        # Base case is not necessary

        # if len(self.children) == 0:
        #     return 1
        # else:

        total: int = 1
        for child in self.children:
            total += child.size()

        return total

    # Add up all the values in the whole tree
    def sum(self) -> int:
        total: int = self.value
        for child in self.children:
            total += child.sum()
        return total

    # Print the tree in outline form, e.g.
    #
    # cat
    #   horse
    #   leaf
    #     potato
    #   sock

    def print_outline(self):
        self.print_outline_indented(0)

    def print_outline_indented(self, indent: int):
        print((' ' * indent) + str(self.value))
        for child in self.children:
            child.print_outline_indented(indent + 2)


def main():
    t = Tree(3, [Tree(2, []), Tree(5, [Tree(19, []), Tree(22, []), Tree(10, [])])])
    print(t.size())
    print(t.sum())

    t2 = Tree('cat', [Tree('horse', []), Tree('leaf', [Tree('potato', [])]), Tree('sock', [])])
    print(t2.size())
    print(t2.sum())

    t2.print_outline()

main()
\end{verbatim}

% \newpage
% \section*{\notready LAB: Fractal Recursion (3/13--14)}

% \begin{reflect}{S' 17}
%   This lab is fun and seemed to go reasonably well, though attendance
%   was poor due to being right before spring break and having project 2
%   due the same day.  In the end I'm also not sure how much it really
%   teaches them about recursion.  But to be honest I don't know how to
%   incorporate recursion more fully into the curriculum without
%   changing the programming language we use.
% \end{reflect}

% \begin{reflect}{F' 18}
%   Lab went fine again.  In the future might want to think about
%   developing a new lab that actually uses some kind of recursive data
%   structure (e.g. rose trees to represent
%   outlines/hierarchies/filesystem).
% \end{reflect}

\newpage
\section*{\notready Recursion IV: Fib memoization (F 3/15)}

Just have fun with it!

\begin{verbatim}
# Fibonacci numbers
#
# First two are 0, 1
# To get the next, add the two previous
#
# 0, 1, 1, 2, 3, 5, 8, 13, 21, 34, 55, ...

# Problem: this function redoes a lot of redundant work!
# Solution: ("memoization") - keep a dictionary that
# remembers previous results.

from typing import *

# Original version (slow!)
def fib(n: int) -> int:
    if n <= 1:
        return n
    else:
        return fib(n-1) + fib(n-2)


fib_dict: Dict[int, int] = {0: 0, 1: 1}

# Memoized version
def fib_m(n: int) -> int:
    if n in fib_dict:
        return fib_dict[n]
    else:
        fib_dict[n] = fib_m(n-1) + fib_m(n-2)
        return fib_dict[n]
\end{verbatim}

\newpage
\class{\notready Josh guest lecture?}
\class{\notready Josh guest lecture?}

\lab{Tornados}

\holiday{Spring break}

\class{\notready Exam 3 review}
\class{\notready Exam 3}

% \newpage
% \class{\notready Queues I}

% Start with a discussion of the queue data structure itself (sequence,
% aka list, where we can only add to one end and remove from the other
% end; first-in-first-out, aka FIFO).  Talk about why it is useful:
% model lines of people in the supermarket or at a bank (determine why
% each has a different way of organizing people (multiple lines with one
% checkout person each, vs one line with multiple tellers)?), processes
% waiting for a turn to use the processor, network packets waiting to be
% sent, etc.  (Also, if it's just a list with \emph{fewer} things we are
% allowed to do, why bother?  Why restrict ourselves?  Don't we want the
% most powerful things we can get?  Well, power cuts both ways: more
% power also means more opportunities to make mistakes; ``with great
% power comes great responsibility''.  We actually want just enough
% power to get the job at hand done, and no more!)

% Go over important vocabulary: front aka head, back aka tail; and go
% over methods:
% \begin{itemize}
% \item \verb|enqueue(item)|: add a new item to the back of the queue
% \item \verb|dequeue()|: remove the next item from the front of the
%   queue (error if queue is empty)
% \item \verb|size() -> int|: how many items are in the queue
% \item \verb|is_empty() -> bool|: is the queue empty?
% \end{itemize}

% Let's implement it!  Get students to come up with suggestions?  They
% will probably suggest using a list, which is definitely easiest (in
% terms of lines of code). Lists already do everything we need, we're
% just constraining the interactions to be with the front and back. Get
% students to decide which end is front and which end is back.  Code up
% the \verb|ListQueue| class together.

% Note, students may also want to add \verb|self.num_items| to keep
% track of the size of the queue, which is fine; can point out that in
% this case we can just use \verb|len|; using \verb|self.num_items|
% means we have more work to make sure we keep it correctly updated.
% Also, if students want to use index \verb|0| as the back and
% \verb|n-1| as the front (unlikely but possible), can use
% \verb|self.items.insert(0,item)| and \verb|self.items.pop()|.
% \begin{verbatim}
% class ListQueue:
%     def __init__(self):
%         self.items = []

%     # Add a new item to the "back".
%     def enqueue(self, item):
%         self.items.append(item)

%     # Remove an item from the "front" and return it.
%     def dequeue(self):
%         return self.items.pop(0)

%     # Return the number of items in the queue
%     def length(self):
%         return len(self.items)

%     # Return whether or not the queue is empty
%     def is_empty(self):
%         return self.length() == 0
% \end{verbatim}

% Then make a \verb|main| method to test \verb|ListQueue|:

% \begin{verbatim}
% import time

% def main():
%     q = ListQueue()

%     for power in range(1,7):

%         # Get starting time
%         start = time.time()
%         # Compute 10^power
%         n = 10 ** power

%         # Add and remove 10^power times
%         for i in range(n):
%             q.add(i)
%         for i in range(n):
%             q.remove()

%         # Get ending time
%         end = time.time()

%         # Print power and elapsed time in seconds
%         print power, end - start

% main()
% \end{verbatim}

% As we put stress on the queue with larger numbers, we see it's not a
% linear relationship. $10^7$ is a good place to stop, since it takes
% (on the order of) 100s of seconds to complete.  Could make a log-log
% plot of the size vs the time, discover that the relationship is
% quadratic.  Give a hand-wavy explanation why the performance is
% quadratic: depends on how lists are implemented; adding or removing at
% the beginning of the list requires shifting everything, so
% $1 + 2 + \dots + n$.

% Perhaps we could do better with a more specialized implementation,
% since we only really care about the front and back.

% \newpage
% \class{Queues II}

% Discuss the \verb|Node| abstraction. Break up a list into individual
% pieces that fit together like Legos. It's a recursive data type,
% something that stores an element and another \verb|Node| as
% components.  This requires a discussion of the word \verb|None|. We've
% seen it as the return value of a void function. Python has a word for
% ``nothing'', and we can use that word when we need a variable but it has
% no value.

% How do we use this to make a Queue? Make the \verb|LinkedQueue| class,
% and have two components in \verb|__init__|, \verb|front| and
% \verb|back|, which start off as \verb|None|.

% Write \verb|size| (one of two ways: by adding recursive method to
% \verb|Node|, or iterative \verb|size| method which walks down list
% until current reference is \verb|None|) and \verb|is_empty|, base it
% on \verb|self.front| being empty.

% First, try to complete \verb|add| and \verb|remove| the natural, but
% wrong, way. A node is then someone standing in line. When people stand
% in lines, they look at the person in front of them. So we start with
% the \verb|is_empty| case, and set \verb|set.front = Node(e)| and
% \verb|self.back = self.front|. This is ok.

% We then write the case when there are already people in line. We have a new
% element to add to the back, so we write

% \begin{verbatim}
% temp = Node(e)
% temp.next = self.back
% self.back = temp
% \end{verbatim}
% to reset the \verb|back| pointer. All looks good so far.

% But we can't remove people from the queue easily. We can get the
% element we need from \verb|self.front|, but who is the next
% \verb|self.front|? \verb|self.front.next == None|, and this is a problem. We
% could walk from the back to get the element right behind \verb|self.front|,
% with complicated two-finger algorithm, and it takes a long time
% (totally negating any benefit from doing something other than a
% list). Boo.

% So instead, we reverse the orientation of everyone in line. You stand in
% line backwards and stare at the person behind you. They are your next. [If
% you do this in real life, people will get a little uncomfortable. Just go
% with it, computer scientists are weird sometimes because it's the right
% thing to do, like starting to count at 0\dots]

% Then we rewrite \verb|add| to match, and finally we can write a remove
% method, to set \verb|self.front = self.front.next|.  Both are now
% constant time.

% How does this fare with our tests? For small numbers, a little more
% expensive. The overhead of making new \verb|Node| objects all the time
% is seen.

% But as it gets larger, the time is linear with the size. For $10^6$
% and beyond, \verb|LinkedQueue| will beat \verb|ListQueue| hands
% down. (Go back and plot in Excel?)

% Finish with discussing how this analysis and construction of data
% structures is at the core of the science part of computer science, and will
% be heavily discussed in CSCI 151 with more formal tools and for more
% complicated structures, within the context of Java.

% \newpage
% \lab{Stuckness \& Debugging}

% \begin{reflect}{Brent, F'18}
%   The lab went really well---Fish and I answered basically zero
%   questions; the students struggled with things until they figured
%   them out.  Definitely heard many excited moments of enlightenment or
%   exultation as a fix worked.  I think they got a lot out of it.
% \end{reflect}

% \newpage
% \section{\notready LAB: Tornado sirens (Th 11/29)}

% \newpage
% \class{\notready Something fun?}

\class{\notready Free/project help}

\end{document}
