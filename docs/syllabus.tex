% -*- mode: LaTeX; compile-command: "pdflatex syllabus.tex" -*-

\documentclass{article}

\usepackage{longtable}
\usepackage{booktabs}
\usepackage{hyperref}
\usepackage{url}
\usepackage{multirow}
\usepackage{graphicx}
\usepackage[backgroundcolor=yellow]{todonotes}

\presetkeys{todonotes}{inline}{}

\begin{document}

{\Large
\noindent CSCI 150 \smallskip

\noindent Foundations of Computer Science \bigskip
}

\noindent Spring 2017 \\
Lecture: MWF 1:10-2:00, MC Reynolds 110\\
Lab: Th 8:10-11:00, Snoddy Computer Lab\\
Website: \url{http://ozark.hendrix.edu/~yorgey/150/} \\
Course text: \emph{Think Python} by Allen Downey \\
\indent \url{http://ozark.hendrix.edu/~yorgey/150/ThinkPython-CSCI150-S17.pdf}
\medskip

\noindent Instructor: Brent Yorgey, MC Reynolds 310 \\
Office hours: any time my door is open, or make an appointment at \\
\indent \url{http://byorgey.youcanbook.me} \\
Email: \texttt{yorgey@hendrix.edu} \medskip

\section*{Course Description}

Introduction to computational problem-solving and computer
programming. Topics include imperative programming constructs
(variables, loops, conditionals, functions, recursion, file
processing), basic object-oriented constructs (classes, objects), and
some fundamental algorithms and data structures (dictionaries, arrays,
linked lists, regular expressions). Students learn through studying
the Python programming language.

\section*{Evaluation}

Evaluation will be based on
\begin{itemize}
\item 25\%: Weekly labs
\item 35\%: Three projects
\item 15\%: Quizzes, HW, and participation
\item 5\%: Midterm exam 1
\item 10\%: Midterm exam 2
\item 10\%: Midterm exam 3
\end{itemize}

\section*{Labs}

Much of your experience with programming in this course will be
through weekly labs, which will comprise 25\% of your final grade. Lab
attendance is required. Labs take place in the Snoddy Computer Lab, in
the Bailey Library.  Each lab will be assigned on Thursday morning
with time allotted to work through the materials, and will be due by
the following \textbf{Tuesday at 10pm}.

Each student has four late days to spend throughout the semester as
they wish.  Simply inform me any time \emph{prior} to the due date for an
assignment that you wish to use a late day; you may then turn in the
assignment up to 24 hours late with no penalty.  Multiple late days
may be used on the same assignment.  There are no partial late days;
turning in an assignment 2 hours late or 20 hours late will both use 1
late day.

On these labs, you may work with a partner on the lab assignments if
you choose. Their name must be listed on any code you hand in as joint
work.  A partnership need only turn in a single copy of the
assignment.  If students working as partners wish to turn in a lab
late, \emph{both} students must use a late day.

\section*{Projects}

You will have three projects in this course, one about every five
weeks, for a total of 35\% of your final grade (the first project is
5\%, the second 10\%, and the final project 20\%). These projects will
cover concepts we have discussed in class and in labs, and will be due
approximately one week after they are assigned.

\textbf{You must work individually on the first two projects}. You may
discuss concepts and ideas with your classmates, but the code you turn
in must be your own. You will be graded not only on correctness, but
also technique, documentation and evaluation of your solution. Further
details on the grading standards and handin instructions for each
project will be given when they are assigned.

% \begin{longtable}[c]{@{}lllll@{}}
% \toprule\addlinespace
% Project & Name & Pct & Assigned & Due
% \\\addlinespace
% \midrule\endhead
% 1 & Today in History & 5\% & Feb 9 & Feb
% 16
% \\\addlinespace
% 2 & Word Games & 10\% & Mar 8 & Mar 15
% \\\addlinespace
% 3 & Final Project & 20\% & Apr 11 & Finals Day
% \\\addlinespace
% \bottomrule
% \end{longtable}

\section*{Attendance and submission policies}

Prompt lecture attendance is expected.  Although I typically do not
take formal attendance, unexcused absences may be reflected in your
class participation grade.  If you must be absent for some reason,
please let me know in advance.

If you are absent from lecture (whether excused or unexcused) it is
\textbf{your responsibility} to obtain notes from other student(s).
Do not come to me and ask ``what did I miss?''.\footnote{I am likely
  to answer that you missed the part where that is your
  responsibility.} On the other hand, if after obtaining notes you
have specific questions or confusions regarding the topics covered, I
would be happy to talk with you.

\section*{Disabilities}

It is the policy of Hendrix College to accommodate students with
disabilities, pursuant to federal and state law. Students should
contact Julie Brown in the Office of Academic Success (505.2954;
\texttt{brownj@hendrix.edu}) to begin the accommodation process. Any
student seeking accommodation in relation to a recognized disability
should inform the instructor at the beginning of the course.

\section*{Academic Integrity}

All Hendrix students must abide by the College's
\href{https://www.hendrix.edu/studentlife/handbook.aspx?id=67121}{Academic
Integrity Policy} as well as
\href{https://www.hendrix.edu/studentlife/handbook.aspx?id=42308}{the
College's Computer Policy}, both of which are outlined in the Student
Handbook.

For specific ways the Academic Integrity policy applies in this
course, please refer to the
\href{http://ozark.hendrix.edu/~yorgey/ac-integrity-policy.html}{Computer
  Science Academic Integrity Policy}.

The short version is that academic integrity violations such as
copying code from another student or the Internet are \textbf{easy to
  detect}, will be \textbf{taken
very seriously}, and carry a default recommended sanction of
\textbf{failure in the course}.

If you have any questions about how the Academic Integrity policy
applies in a particular situation, please contact me.

\section*{Course outline}

A rough outline of the semester is as follows.  This schedule may
change.  Please check the course website for corresponding readings
from your textbook.


\begin{tabular}{cllp{3in}}
  Week & Month & Topics \\
  \hline \\
  1 & Jan & Introduction \\
  2 &     & Boolean logic, conditionals \\
  3 & Feb & Information encoding \\
  4 &     & EXAM 1, functions \\
  5 &     & While loops, strings \\
  6 &     & Lists \\
  7 & Mar & For loops \\
  8 &     & EXAM 2, recursion \\
  9 &     & File I/O, dictionaries \\
  & & \\
  & & \emph{Spring break} \\
  & & \\
  10 &    & Hierarchies, debugging \\
  11 & Apr & Objects and classes \\
  12 &     & Objects and classes \\
  13 &     & Objects and classes, EXAM 3 \\
  14 &     & Queues \\
  15 & May & Final projects
\end{tabular}

\section*{Learning objectives}

By the end of the course you will be able to:

\begin{itemize}
\item Read, understand and execute a computer program written in Python.
\item Read a set of requirements for a computer program in English, and
    write a short Python program (100 lines or less) that corresponds to
    them.
\item Test a Python program and identify and fix programming errors.
\item Identify some errors in a Python program without testing it.
\item Without using a computer, write a very short Python code fragment
    (10 lines or less) that correctly implements a set of requirements.
\item Understand and apply variables, loops, strings, lists, conditionals,
    and functions.
\item Write programs to perform mathematical calculations.
\item Understand the concept of a module.
\item Write a Python program that is separated into at least two modules.
\item Understand the concepts of class and object, and distinguish between
    them.
\item Write a Python program including objects of at least one
    student-designed class.
\item Write and understand appropriate comments in a Python program.
\item Understand the concept of an algorithm and compare the efficiency of
    different algorithms for a simple task.
\end{itemize}

\end{document}
