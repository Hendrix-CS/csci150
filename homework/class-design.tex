\documentclass[]{article}
\usepackage{lmodern}
\usepackage{amssymb,amsmath}
\usepackage{ifxetex,ifluatex}
\usepackage{fixltx2e} % provides \textsubscript
\ifnum 0\ifxetex 1\fi\ifluatex 1\fi=0 % if pdftex
  \usepackage[T1]{fontenc}
  \usepackage[utf8]{inputenc}
\else % if luatex or xelatex
  \ifxetex
    \usepackage{mathspec}
  \else
    \usepackage{fontspec}
  \fi
  \defaultfontfeatures{Ligatures=TeX,Scale=MatchLowercase}
\fi
% use upquote if available, for straight quotes in verbatim environments
\IfFileExists{upquote.sty}{\usepackage{upquote}}{}
% use microtype if available
\IfFileExists{microtype.sty}{%
\usepackage[]{microtype}
\UseMicrotypeSet[protrusion]{basicmath} % disable protrusion for tt fonts
}{}
\PassOptionsToPackage{hyphens}{url} % url is loaded by hyperref
\usepackage[unicode=true]{hyperref}
\hypersetup{
            pdfborder={0 0 0},
            breaklinks=true}
\urlstyle{same}  % don't use monospace font for urls
\usepackage{color}
\usepackage{fancyvrb}
\newcommand{\VerbBar}{|}
\newcommand{\VERB}{\Verb[commandchars=\\\{\}]}
\DefineVerbatimEnvironment{Highlighting}{Verbatim}{commandchars=\\\{\}}
% Add ',fontsize=\small' for more characters per line
\newenvironment{Shaded}{}{}
\newcommand{\KeywordTok}[1]{\textcolor[rgb]{0.00,0.44,0.13}{\textbf{#1}}}
\newcommand{\DataTypeTok}[1]{\textcolor[rgb]{0.56,0.13,0.00}{#1}}
\newcommand{\DecValTok}[1]{\textcolor[rgb]{0.25,0.63,0.44}{#1}}
\newcommand{\BaseNTok}[1]{\textcolor[rgb]{0.25,0.63,0.44}{#1}}
\newcommand{\FloatTok}[1]{\textcolor[rgb]{0.25,0.63,0.44}{#1}}
\newcommand{\ConstantTok}[1]{\textcolor[rgb]{0.53,0.00,0.00}{#1}}
\newcommand{\CharTok}[1]{\textcolor[rgb]{0.25,0.44,0.63}{#1}}
\newcommand{\SpecialCharTok}[1]{\textcolor[rgb]{0.25,0.44,0.63}{#1}}
\newcommand{\StringTok}[1]{\textcolor[rgb]{0.25,0.44,0.63}{#1}}
\newcommand{\VerbatimStringTok}[1]{\textcolor[rgb]{0.25,0.44,0.63}{#1}}
\newcommand{\SpecialStringTok}[1]{\textcolor[rgb]{0.73,0.40,0.53}{#1}}
\newcommand{\ImportTok}[1]{#1}
\newcommand{\CommentTok}[1]{\textcolor[rgb]{0.38,0.63,0.69}{\textit{#1}}}
\newcommand{\DocumentationTok}[1]{\textcolor[rgb]{0.73,0.13,0.13}{\textit{#1}}}
\newcommand{\AnnotationTok}[1]{\textcolor[rgb]{0.38,0.63,0.69}{\textbf{\textit{#1}}}}
\newcommand{\CommentVarTok}[1]{\textcolor[rgb]{0.38,0.63,0.69}{\textbf{\textit{#1}}}}
\newcommand{\OtherTok}[1]{\textcolor[rgb]{0.00,0.44,0.13}{#1}}
\newcommand{\FunctionTok}[1]{\textcolor[rgb]{0.02,0.16,0.49}{#1}}
\newcommand{\VariableTok}[1]{\textcolor[rgb]{0.10,0.09,0.49}{#1}}
\newcommand{\ControlFlowTok}[1]{\textcolor[rgb]{0.00,0.44,0.13}{\textbf{#1}}}
\newcommand{\OperatorTok}[1]{\textcolor[rgb]{0.40,0.40,0.40}{#1}}
\newcommand{\BuiltInTok}[1]{#1}
\newcommand{\ExtensionTok}[1]{#1}
\newcommand{\PreprocessorTok}[1]{\textcolor[rgb]{0.74,0.48,0.00}{#1}}
\newcommand{\AttributeTok}[1]{\textcolor[rgb]{0.49,0.56,0.16}{#1}}
\newcommand{\RegionMarkerTok}[1]{#1}
\newcommand{\InformationTok}[1]{\textcolor[rgb]{0.38,0.63,0.69}{\textbf{\textit{#1}}}}
\newcommand{\WarningTok}[1]{\textcolor[rgb]{0.38,0.63,0.69}{\textbf{\textit{#1}}}}
\newcommand{\AlertTok}[1]{\textcolor[rgb]{1.00,0.00,0.00}{\textbf{#1}}}
\newcommand{\ErrorTok}[1]{\textcolor[rgb]{1.00,0.00,0.00}{\textbf{#1}}}
\newcommand{\NormalTok}[1]{#1}
\IfFileExists{parskip.sty}{%
\usepackage{parskip}
}{% else
\setlength{\parindent}{0pt}
\setlength{\parskip}{6pt plus 2pt minus 1pt}
}
\setlength{\emergencystretch}{3em}  % prevent overfull lines
\providecommand{\tightlist}{%
  \setlength{\itemsep}{0pt}\setlength{\parskip}{0pt}}
\setcounter{secnumdepth}{0}
% Redefines (sub)paragraphs to behave more like sections
\ifx\paragraph\undefined\else
\let\oldparagraph\paragraph
\renewcommand{\paragraph}[1]{\oldparagraph{#1}\mbox{}}
\fi
\ifx\subparagraph\undefined\else
\let\oldsubparagraph\subparagraph
\renewcommand{\subparagraph}[1]{\oldsubparagraph{#1}\mbox{}}
\fi

% set default figure placement to htbp
\makeatletter
\def\fps@figure{htbp}
\makeatother


\date{}

\begin{document}

\subsection{CSCI 150 HW: class design
practice}\label{csci-150-hw-class-design-practice}

\emph{Due: Wednesday, November 7}

To receive full credit, for each exercise you should do the following:

\begin{enumerate}
\def\labelenumi{\arabic{enumi}.}
\item
  \textbf{Design}: First, design a Python class as requested in the
  exercise. Type in your class definition.
\item
  \textbf{Check}: Run the provided test code. Does your actual output
  agree with the given correct output?
\item
  \textbf{Evaluate}: If the actual output does not match the expected
  output, keep experimenting, consult the textbook or Python
  documentation, ask a friend or TA or professor, \emph{etc.} until you
  can fix your class definition and explain what your
  misunderstanding(s) were. (You do not need to do anything for step 3
  if the outputs already agree exactly.)
\end{enumerate}

You should consider the code in each exercise separately from the other
exercises.

\begin{enumerate}
\def\labelenumi{\arabic{enumi}.}
\item
  Write a Python class \texttt{BouncyBall}, which represents a bouncy
  ball containing a certain amount of air.

  \begin{itemize}
  \tightlist
  \item
    When a \texttt{BouncyBall} object is first created, it should have
    10 units of air.
  \item
    There should be a method \texttt{bounce()} which normally prints the
    word \texttt{Bounce!} and decreasese the amount of air in the ball
    by two units. However, if the amount of air is less than or equal to
    three, then \texttt{bounce()} does not decrease the amount of air
    and prints \texttt{Thupp.} instead of \texttt{Bounce!}.
  \item
    There should be a method \texttt{inflate()} which increases the
    amount of air by three units. If the amount of air ever becomes
    greater than 12, then the ball explodes by printing
    \texttt{BANG!!!}.
  \item
    You cannot bounce or inflate an exploded ball. After a ball
    explodes, calling \texttt{bounce()} or \texttt{inflate()} should
    just cause a message to be printed such as
    \texttt{Sorry,\ you\ cannot\ bounce\ this\ ball!\ \ It\ has\ exploded.}
  \end{itemize}

  To test your class, you can type in and run the following code:

\begin{Shaded}
\begin{Highlighting}[]
\KeywordTok{def}\NormalTok{ main():}
\NormalTok{    b }\OperatorTok{=}\NormalTok{ BouncyBall()}

    \ControlFlowTok{for}\NormalTok{ i }\KeywordTok{in} \BuiltInTok{range}\NormalTok{(}\DecValTok{6}\NormalTok{):}
\NormalTok{        b.bounce()}

\NormalTok{    b.inflate()}
\NormalTok{    b.bounce()}
\NormalTok{    b.bounce()}

    \ControlFlowTok{for}\NormalTok{ i }\KeywordTok{in} \BuiltInTok{range}\NormalTok{(}\DecValTok{5}\NormalTok{):}
\NormalTok{        b.inflate()}

\NormalTok{    b.bounce()}

\NormalTok{main()}
\end{Highlighting}
\end{Shaded}

  If your definition of \texttt{BouncyBall} is correct, \texttt{main()}
  should produce the following output:

\begin{verbatim}
Bounce!
Bounce!
Bounce!
Bounce!
Thupp.
Thupp.
Bounce!
Thupp.
BANG!!!
Sorry, you cannot inflate this ball!  It has exploded.
Sorry, you cannot bounce this ball!  It has exploded.
\end{verbatim}
\item
  Write a Python class \texttt{Gradebook} which works as follows:

  \begin{itemize}
  \tightlist
  \item
    When a new \texttt{Gradebook} object is first created, it should
    start out with an empty list of grades, and zero points of extra
    credit.
  \item
    There should be a method \texttt{add\_grade(g:\ int)} which adds the
    grade \texttt{g} to the end of the list.
  \item
    There should be a method \texttt{add\_ec(ec:\ int)} which adds
    \texttt{ec} points of extra credit to the current amount of extra
    credit.
  \item
    There should be a method \texttt{average()} which computes and
    returns the average of all the grades so far (the sum of all the
    grades, plus the extra credit score, divided by the number of
    grades).
  \end{itemize}

  You can test your implementation of \texttt{Gradebook} by running the
  code below:

\begin{Shaded}
\begin{Highlighting}[]
\KeywordTok{def}\NormalTok{ main():}
\NormalTok{gb }\OperatorTok{=}\NormalTok{ Gradebook()}
\NormalTok{gb.add_grade(}\DecValTok{90}\NormalTok{)}
\NormalTok{gb.add_grade(}\DecValTok{83}\NormalTok{)}
\NormalTok{gb.add_grade(}\DecValTok{97}\NormalTok{)}
\NormalTok{gb.add_ec(}\DecValTok{10}\NormalTok{)}

\BuiltInTok{print}\NormalTok{(gb.average())}
\end{Highlighting}
\end{Shaded}

  If your definition of \texttt{Gradebook} is correct, this should print
  \texttt{93.33333333333333}.
\end{enumerate}

\end{document}
