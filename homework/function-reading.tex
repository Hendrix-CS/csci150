\PassOptionsToPackage{unicode=true}{hyperref} % options for packages loaded elsewhere
\PassOptionsToPackage{hyphens}{url}
%
\documentclass[]{article}
\usepackage{lmodern}
\usepackage{amssymb,amsmath}
\usepackage{ifxetex,ifluatex}
\usepackage{fixltx2e} % provides \textsubscript
\ifnum 0\ifxetex 1\fi\ifluatex 1\fi=0 % if pdftex
  \usepackage[T1]{fontenc}
  \usepackage[utf8]{inputenc}
  \usepackage{textcomp} % provides euro and other symbols
\else % if luatex or xelatex
  \usepackage{unicode-math}
  \defaultfontfeatures{Ligatures=TeX,Scale=MatchLowercase}
\fi
% use upquote if available, for straight quotes in verbatim environments
\IfFileExists{upquote.sty}{\usepackage{upquote}}{}
% use microtype if available
\IfFileExists{microtype.sty}{%
\usepackage[]{microtype}
\UseMicrotypeSet[protrusion]{basicmath} % disable protrusion for tt fonts
}{}
\IfFileExists{parskip.sty}{%
\usepackage{parskip}
}{% else
\setlength{\parindent}{0pt}
\setlength{\parskip}{6pt plus 2pt minus 1pt}
}
\usepackage{hyperref}
\hypersetup{
            pdfborder={0 0 0},
            breaklinks=true}
\urlstyle{same}  % don't use monospace font for urls
\usepackage{color}
\usepackage{fancyvrb}
\newcommand{\VerbBar}{|}
\newcommand{\VERB}{\Verb[commandchars=\\\{\}]}
\DefineVerbatimEnvironment{Highlighting}{Verbatim}{commandchars=\\\{\}}
% Add ',fontsize=\small' for more characters per line
\newenvironment{Shaded}{}{}
\newcommand{\AlertTok}[1]{\textcolor[rgb]{1.00,0.00,0.00}{\textbf{#1}}}
\newcommand{\AnnotationTok}[1]{\textcolor[rgb]{0.38,0.63,0.69}{\textbf{\textit{#1}}}}
\newcommand{\AttributeTok}[1]{\textcolor[rgb]{0.49,0.56,0.16}{#1}}
\newcommand{\BaseNTok}[1]{\textcolor[rgb]{0.25,0.63,0.44}{#1}}
\newcommand{\BuiltInTok}[1]{#1}
\newcommand{\CharTok}[1]{\textcolor[rgb]{0.25,0.44,0.63}{#1}}
\newcommand{\CommentTok}[1]{\textcolor[rgb]{0.38,0.63,0.69}{\textit{#1}}}
\newcommand{\CommentVarTok}[1]{\textcolor[rgb]{0.38,0.63,0.69}{\textbf{\textit{#1}}}}
\newcommand{\ConstantTok}[1]{\textcolor[rgb]{0.53,0.00,0.00}{#1}}
\newcommand{\ControlFlowTok}[1]{\textcolor[rgb]{0.00,0.44,0.13}{\textbf{#1}}}
\newcommand{\DataTypeTok}[1]{\textcolor[rgb]{0.56,0.13,0.00}{#1}}
\newcommand{\DecValTok}[1]{\textcolor[rgb]{0.25,0.63,0.44}{#1}}
\newcommand{\DocumentationTok}[1]{\textcolor[rgb]{0.73,0.13,0.13}{\textit{#1}}}
\newcommand{\ErrorTok}[1]{\textcolor[rgb]{1.00,0.00,0.00}{\textbf{#1}}}
\newcommand{\ExtensionTok}[1]{#1}
\newcommand{\FloatTok}[1]{\textcolor[rgb]{0.25,0.63,0.44}{#1}}
\newcommand{\FunctionTok}[1]{\textcolor[rgb]{0.02,0.16,0.49}{#1}}
\newcommand{\ImportTok}[1]{#1}
\newcommand{\InformationTok}[1]{\textcolor[rgb]{0.38,0.63,0.69}{\textbf{\textit{#1}}}}
\newcommand{\KeywordTok}[1]{\textcolor[rgb]{0.00,0.44,0.13}{\textbf{#1}}}
\newcommand{\NormalTok}[1]{#1}
\newcommand{\OperatorTok}[1]{\textcolor[rgb]{0.40,0.40,0.40}{#1}}
\newcommand{\OtherTok}[1]{\textcolor[rgb]{0.00,0.44,0.13}{#1}}
\newcommand{\PreprocessorTok}[1]{\textcolor[rgb]{0.74,0.48,0.00}{#1}}
\newcommand{\RegionMarkerTok}[1]{#1}
\newcommand{\SpecialCharTok}[1]{\textcolor[rgb]{0.25,0.44,0.63}{#1}}
\newcommand{\SpecialStringTok}[1]{\textcolor[rgb]{0.73,0.40,0.53}{#1}}
\newcommand{\StringTok}[1]{\textcolor[rgb]{0.25,0.44,0.63}{#1}}
\newcommand{\VariableTok}[1]{\textcolor[rgb]{0.10,0.09,0.49}{#1}}
\newcommand{\VerbatimStringTok}[1]{\textcolor[rgb]{0.25,0.44,0.63}{#1}}
\newcommand{\WarningTok}[1]{\textcolor[rgb]{0.38,0.63,0.69}{\textbf{\textit{#1}}}}
\setlength{\emergencystretch}{3em}  % prevent overfull lines
\providecommand{\tightlist}{%
  \setlength{\itemsep}{0pt}\setlength{\parskip}{0pt}}
\setcounter{secnumdepth}{0}
% Redefines (sub)paragraphs to behave more like sections
\ifx\paragraph\undefined\else
\let\oldparagraph\paragraph
\renewcommand{\paragraph}[1]{\oldparagraph{#1}\mbox{}}
\fi
\ifx\subparagraph\undefined\else
\let\oldsubparagraph\subparagraph
\renewcommand{\subparagraph}[1]{\oldsubparagraph{#1}\mbox{}}
\fi

% set default figure placement to htbp
\makeatletter
\def\fps@figure{htbp}
\makeatother


\date{}

\begin{document}

\hypertarget{csci-150-hw-function-and-loop-reading-practice}{%
\subsection{CSCI 150 HW: function and loop reading
practice}\label{csci-150-hw-function-and-loop-reading-practice}}

\emph{Due: Wednesday, October 2}

To receive full credit, for each exercise you should do the following:

\begin{enumerate}
\def\labelenumi{\arabic{enumi}.}
\item
  \textbf{Predict}: First, complete the exercise \emph{without} using
  the Python interpreter. (You are welcome to refer to your notes or
  textbook, read Python documentation, look at examples from class,
  \emph{etc.}; just don't actually run any code.) Trace the execution of
  the code in the exercise.
\item
  \textbf{Check}: Run the code. Does the actual output agree with what
  you wrote down in step 1?
\item
  \textbf{Evaluate}: If your answer to part 1 was different than the
  actual output, keep experimenting with it, consult the textbook or
  Python documentation, ask a friend or TA or professor, \emph{etc.}
  until you can explain why the code works the way it does \emph{and}
  what your misunderstanding(s) were in part 1. (You do not need to do
  anything for step 3 if the output agrees exactly with what you wrote
  in step 1.)
\end{enumerate}

You will not be graded on how correct your answer is in part 1. However,
you \emph{will} be graded on the accuracy of your evaluation in step 3.
Obviously, I will not be able to tell the difference if you simply run
the code and paste the output for step 1; please do not do that! You
will only be depriving yourself of a learning opportunity (not to
mention that it is a violation of the academic integrity policy).

You should consider the code in each exercise separately from the other
exercises.

\begin{enumerate}
\def\labelenumi{\arabic{enumi}.}
\item
  Consider the functions defined below. Trace the execution when
  \texttt{main1()} is called.

\begin{Shaded}
\begin{Highlighting}[]
\KeywordTok{def}\NormalTok{ foo(a: }\BuiltInTok{int}\NormalTok{) }\OperatorTok{->} \BuiltInTok{int}\NormalTok{:}
\NormalTok{    b }\OperatorTok{=} \DecValTok{3}\OperatorTok{*}\NormalTok{a }\OperatorTok{+} \DecValTok{2}
    \ControlFlowTok{return}\NormalTok{ b}
    \BuiltInTok{print}\NormalTok{(}\StringTok{"In foo"}\NormalTok{)}

\KeywordTok{def}\NormalTok{ bar(x: }\BuiltInTok{int}\NormalTok{, y: }\BuiltInTok{int}\NormalTok{) }\OperatorTok{->} \BuiltInTok{int}\NormalTok{:}
    \ControlFlowTok{return}\NormalTok{ foo(x) }\OperatorTok{+}\NormalTok{ foo(y)}

\KeywordTok{def}\NormalTok{ main1():}
    \BuiltInTok{print}\NormalTok{(}\StringTok{"The value is "} \OperatorTok{+} \BuiltInTok{str}\NormalTok{(bar(}\DecValTok{2}\NormalTok{,}\DecValTok{3}\NormalTok{)))}

\NormalTok{main1()}
\end{Highlighting}
\end{Shaded}
\item
  Consider the functions defined below. Trace the execution when
  \texttt{main2()} is called.

\begin{Shaded}
\begin{Highlighting}[]
\KeywordTok{def}\NormalTok{ f1():}
    \BuiltInTok{print}\NormalTok{(}\StringTok{"mushroom"}\NormalTok{)}

\KeywordTok{def}\NormalTok{ f2():}
\NormalTok{    f1()}
    \BuiltInTok{print}\NormalTok{(}\StringTok{"badger"}\NormalTok{)}
\NormalTok{    f1()}

\KeywordTok{def}\NormalTok{ f3(n: }\BuiltInTok{int}\NormalTok{):}
\NormalTok{    f2()}
    \ControlFlowTok{if}\NormalTok{ n }\OperatorTok{>} \DecValTok{5}\NormalTok{:}
        \BuiltInTok{print}\NormalTok{(}\StringTok{"snake"}\NormalTok{)}
\NormalTok{        f1()}
    \ControlFlowTok{else}\NormalTok{:}
        \BuiltInTok{print}\NormalTok{(}\StringTok{"snaaaaake"}\NormalTok{)}

\KeywordTok{def}\NormalTok{ main2():}
\NormalTok{    f3(}\DecValTok{2}\NormalTok{)}
\NormalTok{    f3(}\DecValTok{6}\NormalTok{)}

\NormalTok{main2()}
\end{Highlighting}
\end{Shaded}
\item
  Trace the execution when \texttt{main3} is called.

\begin{Shaded}
\begin{Highlighting}[]
\KeywordTok{def}\NormalTok{ main3():}
\NormalTok{    s: }\BuiltInTok{int} \OperatorTok{=} \DecValTok{0}
\NormalTok{    i: }\BuiltInTok{int} \OperatorTok{=} \DecValTok{0}
    \ControlFlowTok{while}\NormalTok{ i }\OperatorTok{<} \DecValTok{5}\NormalTok{:}
\NormalTok{        j: }\BuiltInTok{int} \OperatorTok{=} \DecValTok{0}
        \ControlFlowTok{while}\NormalTok{ j }\OperatorTok{<}\NormalTok{ i:}
\NormalTok{            s }\OperatorTok{+=}\NormalTok{ j}
\NormalTok{            j }\OperatorTok{+=} \DecValTok{1}
\NormalTok{        i }\OperatorTok{+=} \DecValTok{1}
    \BuiltInTok{print}\NormalTok{(s)}

\NormalTok{main3()}
\end{Highlighting}
\end{Shaded}
\end{enumerate}

\end{document}
