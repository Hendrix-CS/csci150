\documentclass[]{article}
\usepackage{lmodern}
\usepackage{amssymb,amsmath}
\usepackage{ifxetex,ifluatex}
\usepackage{fixltx2e} % provides \textsubscript
\ifnum 0\ifxetex 1\fi\ifluatex 1\fi=0 % if pdftex
  \usepackage[T1]{fontenc}
  \usepackage[utf8]{inputenc}
\else % if luatex or xelatex
  \ifxetex
    \usepackage{mathspec}
  \else
    \usepackage{fontspec}
  \fi
  \defaultfontfeatures{Ligatures=TeX,Scale=MatchLowercase}
\fi
% use upquote if available, for straight quotes in verbatim environments
\IfFileExists{upquote.sty}{\usepackage{upquote}}{}
% use microtype if available
\IfFileExists{microtype.sty}{%
\usepackage[]{microtype}
\UseMicrotypeSet[protrusion]{basicmath} % disable protrusion for tt fonts
}{}
\PassOptionsToPackage{hyphens}{url} % url is loaded by hyperref
\usepackage[unicode=true]{hyperref}
\hypersetup{
            pdfborder={0 0 0},
            breaklinks=true}
\urlstyle{same}  % don't use monospace font for urls
\usepackage{color}
\usepackage{fancyvrb}
\newcommand{\VerbBar}{|}
\newcommand{\VERB}{\Verb[commandchars=\\\{\}]}
\DefineVerbatimEnvironment{Highlighting}{Verbatim}{commandchars=\\\{\}}
% Add ',fontsize=\small' for more characters per line
\newenvironment{Shaded}{}{}
\newcommand{\KeywordTok}[1]{\textcolor[rgb]{0.00,0.44,0.13}{\textbf{#1}}}
\newcommand{\DataTypeTok}[1]{\textcolor[rgb]{0.56,0.13,0.00}{#1}}
\newcommand{\DecValTok}[1]{\textcolor[rgb]{0.25,0.63,0.44}{#1}}
\newcommand{\BaseNTok}[1]{\textcolor[rgb]{0.25,0.63,0.44}{#1}}
\newcommand{\FloatTok}[1]{\textcolor[rgb]{0.25,0.63,0.44}{#1}}
\newcommand{\ConstantTok}[1]{\textcolor[rgb]{0.53,0.00,0.00}{#1}}
\newcommand{\CharTok}[1]{\textcolor[rgb]{0.25,0.44,0.63}{#1}}
\newcommand{\SpecialCharTok}[1]{\textcolor[rgb]{0.25,0.44,0.63}{#1}}
\newcommand{\StringTok}[1]{\textcolor[rgb]{0.25,0.44,0.63}{#1}}
\newcommand{\VerbatimStringTok}[1]{\textcolor[rgb]{0.25,0.44,0.63}{#1}}
\newcommand{\SpecialStringTok}[1]{\textcolor[rgb]{0.73,0.40,0.53}{#1}}
\newcommand{\ImportTok}[1]{#1}
\newcommand{\CommentTok}[1]{\textcolor[rgb]{0.38,0.63,0.69}{\textit{#1}}}
\newcommand{\DocumentationTok}[1]{\textcolor[rgb]{0.73,0.13,0.13}{\textit{#1}}}
\newcommand{\AnnotationTok}[1]{\textcolor[rgb]{0.38,0.63,0.69}{\textbf{\textit{#1}}}}
\newcommand{\CommentVarTok}[1]{\textcolor[rgb]{0.38,0.63,0.69}{\textbf{\textit{#1}}}}
\newcommand{\OtherTok}[1]{\textcolor[rgb]{0.00,0.44,0.13}{#1}}
\newcommand{\FunctionTok}[1]{\textcolor[rgb]{0.02,0.16,0.49}{#1}}
\newcommand{\VariableTok}[1]{\textcolor[rgb]{0.10,0.09,0.49}{#1}}
\newcommand{\ControlFlowTok}[1]{\textcolor[rgb]{0.00,0.44,0.13}{\textbf{#1}}}
\newcommand{\OperatorTok}[1]{\textcolor[rgb]{0.40,0.40,0.40}{#1}}
\newcommand{\BuiltInTok}[1]{#1}
\newcommand{\ExtensionTok}[1]{#1}
\newcommand{\PreprocessorTok}[1]{\textcolor[rgb]{0.74,0.48,0.00}{#1}}
\newcommand{\AttributeTok}[1]{\textcolor[rgb]{0.49,0.56,0.16}{#1}}
\newcommand{\RegionMarkerTok}[1]{#1}
\newcommand{\InformationTok}[1]{\textcolor[rgb]{0.38,0.63,0.69}{\textbf{\textit{#1}}}}
\newcommand{\WarningTok}[1]{\textcolor[rgb]{0.38,0.63,0.69}{\textbf{\textit{#1}}}}
\newcommand{\AlertTok}[1]{\textcolor[rgb]{1.00,0.00,0.00}{\textbf{#1}}}
\newcommand{\ErrorTok}[1]{\textcolor[rgb]{1.00,0.00,0.00}{\textbf{#1}}}
\newcommand{\NormalTok}[1]{#1}
\IfFileExists{parskip.sty}{%
\usepackage{parskip}
}{% else
\setlength{\parindent}{0pt}
\setlength{\parskip}{6pt plus 2pt minus 1pt}
}
\setlength{\emergencystretch}{3em}  % prevent overfull lines
\providecommand{\tightlist}{%
  \setlength{\itemsep}{0pt}\setlength{\parskip}{0pt}}
\setcounter{secnumdepth}{0}
% Redefines (sub)paragraphs to behave more like sections
\ifx\paragraph\undefined\else
\let\oldparagraph\paragraph
\renewcommand{\paragraph}[1]{\oldparagraph{#1}\mbox{}}
\fi
\ifx\subparagraph\undefined\else
\let\oldsubparagraph\subparagraph
\renewcommand{\subparagraph}[1]{\oldsubparagraph{#1}\mbox{}}
\fi

% set default figure placement to htbp
\makeatletter
\def\fps@figure{htbp}
\makeatother


\date{}

\begin{document}

\subsection{CSCI 150 HW: function reading
practice}\label{csci-150-hw-function-reading-practice}

\emph{Due: Wednesday, February 21}

To receive full credit, for each exercise you should do the following:

\begin{enumerate}
\def\labelenumi{\arabic{enumi}.}
\item
  \textbf{Predict}: First, complete the exercise \emph{without} using
  the Python interpreter. (You are welcome to refer to your notes or
  textbook, read Python documentation, look at examples from class,
  \emph{etc.}; just don't actually run any code.) Trace the execution of
  the code in the exercise, and write down the final output.
\item
  \textbf{Check}: Run the code. Does the actual output agree with what
  you wrote down in step 1?
\item
  \textbf{Evaluate}: If your answer to part 1 was different than the
  actual output, keep experimenting with it, consult the textbook or
  Python documentation, ask a friend or TA or professor, \emph{etc.}
  until you can explain why the code works the way it does \emph{and}
  what your misunderstanding(s) were in part 1. (You do not need to do
  anything for step 3 if the output agrees exactly with what you wrote
  in step 1.)
\end{enumerate}

You will not be graded on how correct your answer is in part 1. However,
you \emph{will} be graded on the accuracy of your evaluation in step 3.
Obviously, I will not be able to tell the difference if you simply run
the code and paste the output for step 1; please do not do that! You
will only be depriving yourself of a learning opportunity (not to
mention that it is a violation of the academic integrity policy).

Turn in your answers and evaluations either electronically
\href{https://goo.gl/forms/XsJVafSZLdedQY1M2}{via the usual form}, or on
paper.

You should consider the code in each exercise separately from the other
exercises.

You should complete the following exercises \emph{without} using a
computer, though you may consult your textbook. You should write your
answers by hand and turn them in on paper.

\begin{enumerate}
\def\labelenumi{\arabic{enumi}.}
\item
  Consider the functions defined below. What does \texttt{main()} print?

\begin{Shaded}
\begin{Highlighting}[]
\KeywordTok{def}\NormalTok{ foo(a):}
\NormalTok{    b }\OperatorTok{=} \DecValTok{3}\OperatorTok{*}\NormalTok{a }\OperatorTok{+} \DecValTok{2}
    \ControlFlowTok{return}\NormalTok{ b}

\KeywordTok{def}\NormalTok{ bar(x,y):}
    \ControlFlowTok{return}\NormalTok{ foo(x) }\OperatorTok{+}\NormalTok{ foo(y)}

\KeywordTok{def}\NormalTok{ main():}
    \BuiltInTok{print} \StringTok{"The value is "} \OperatorTok{+} \BuiltInTok{str}\NormalTok{(bar(}\DecValTok{2}\NormalTok{,}\DecValTok{3}\NormalTok{))}
\end{Highlighting}
\end{Shaded}
\item
  Consider the functions defined below. What is printed by
  \texttt{main2()}?

\begin{Shaded}
\begin{Highlighting}[]
\KeywordTok{def}\NormalTok{ f1():}
    \BuiltInTok{print} \StringTok{"mushroom"}

\KeywordTok{def}\NormalTok{ f2():}
\NormalTok{    f1()}
    \BuiltInTok{print} \StringTok{"badger"}
\NormalTok{    f1()}

\KeywordTok{def}\NormalTok{ f3(n):}
\NormalTok{    f2()}
    \ControlFlowTok{if}\NormalTok{ n }\OperatorTok{>} \DecValTok{5}\NormalTok{:}
        \BuiltInTok{print} \StringTok{"snake"}
\NormalTok{        f1()}
    \ControlFlowTok{else}\NormalTok{:}
        \BuiltInTok{print} \StringTok{"snaaaaake"}

\KeywordTok{def}\NormalTok{ main2():}
\NormalTok{    f3(}\DecValTok{2}\NormalTok{)}
\NormalTok{    f3(}\DecValTok{6}\NormalTok{)}
\end{Highlighting}
\end{Shaded}
\end{enumerate}

\end{document}
