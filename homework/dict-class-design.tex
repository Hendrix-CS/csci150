\PassOptionsToPackage{unicode=true}{hyperref} % options for packages loaded elsewhere
\PassOptionsToPackage{hyphens}{url}
%
\documentclass[]{article}
\usepackage{lmodern}
\usepackage{amssymb,amsmath}
\usepackage{ifxetex,ifluatex}
\usepackage{fixltx2e} % provides \textsubscript
\ifnum 0\ifxetex 1\fi\ifluatex 1\fi=0 % if pdftex
  \usepackage[T1]{fontenc}
  \usepackage[utf8]{inputenc}
  \usepackage{textcomp} % provides euro and other symbols
\else % if luatex or xelatex
  \usepackage{unicode-math}
  \defaultfontfeatures{Ligatures=TeX,Scale=MatchLowercase}
\fi
% use upquote if available, for straight quotes in verbatim environments
\IfFileExists{upquote.sty}{\usepackage{upquote}}{}
% use microtype if available
\IfFileExists{microtype.sty}{%
\usepackage[]{microtype}
\UseMicrotypeSet[protrusion]{basicmath} % disable protrusion for tt fonts
}{}
\IfFileExists{parskip.sty}{%
\usepackage{parskip}
}{% else
\setlength{\parindent}{0pt}
\setlength{\parskip}{6pt plus 2pt minus 1pt}
}
\usepackage{hyperref}
\hypersetup{
            pdfborder={0 0 0},
            breaklinks=true}
\urlstyle{same}  % don't use monospace font for urls
\usepackage{color}
\usepackage{fancyvrb}
\newcommand{\VerbBar}{|}
\newcommand{\VERB}{\Verb[commandchars=\\\{\}]}
\DefineVerbatimEnvironment{Highlighting}{Verbatim}{commandchars=\\\{\}}
% Add ',fontsize=\small' for more characters per line
\newenvironment{Shaded}{}{}
\newcommand{\AlertTok}[1]{\textcolor[rgb]{1.00,0.00,0.00}{\textbf{#1}}}
\newcommand{\AnnotationTok}[1]{\textcolor[rgb]{0.38,0.63,0.69}{\textbf{\textit{#1}}}}
\newcommand{\AttributeTok}[1]{\textcolor[rgb]{0.49,0.56,0.16}{#1}}
\newcommand{\BaseNTok}[1]{\textcolor[rgb]{0.25,0.63,0.44}{#1}}
\newcommand{\BuiltInTok}[1]{#1}
\newcommand{\CharTok}[1]{\textcolor[rgb]{0.25,0.44,0.63}{#1}}
\newcommand{\CommentTok}[1]{\textcolor[rgb]{0.38,0.63,0.69}{\textit{#1}}}
\newcommand{\CommentVarTok}[1]{\textcolor[rgb]{0.38,0.63,0.69}{\textbf{\textit{#1}}}}
\newcommand{\ConstantTok}[1]{\textcolor[rgb]{0.53,0.00,0.00}{#1}}
\newcommand{\ControlFlowTok}[1]{\textcolor[rgb]{0.00,0.44,0.13}{\textbf{#1}}}
\newcommand{\DataTypeTok}[1]{\textcolor[rgb]{0.56,0.13,0.00}{#1}}
\newcommand{\DecValTok}[1]{\textcolor[rgb]{0.25,0.63,0.44}{#1}}
\newcommand{\DocumentationTok}[1]{\textcolor[rgb]{0.73,0.13,0.13}{\textit{#1}}}
\newcommand{\ErrorTok}[1]{\textcolor[rgb]{1.00,0.00,0.00}{\textbf{#1}}}
\newcommand{\ExtensionTok}[1]{#1}
\newcommand{\FloatTok}[1]{\textcolor[rgb]{0.25,0.63,0.44}{#1}}
\newcommand{\FunctionTok}[1]{\textcolor[rgb]{0.02,0.16,0.49}{#1}}
\newcommand{\ImportTok}[1]{#1}
\newcommand{\InformationTok}[1]{\textcolor[rgb]{0.38,0.63,0.69}{\textbf{\textit{#1}}}}
\newcommand{\KeywordTok}[1]{\textcolor[rgb]{0.00,0.44,0.13}{\textbf{#1}}}
\newcommand{\NormalTok}[1]{#1}
\newcommand{\OperatorTok}[1]{\textcolor[rgb]{0.40,0.40,0.40}{#1}}
\newcommand{\OtherTok}[1]{\textcolor[rgb]{0.00,0.44,0.13}{#1}}
\newcommand{\PreprocessorTok}[1]{\textcolor[rgb]{0.74,0.48,0.00}{#1}}
\newcommand{\RegionMarkerTok}[1]{#1}
\newcommand{\SpecialCharTok}[1]{\textcolor[rgb]{0.25,0.44,0.63}{#1}}
\newcommand{\SpecialStringTok}[1]{\textcolor[rgb]{0.73,0.40,0.53}{#1}}
\newcommand{\StringTok}[1]{\textcolor[rgb]{0.25,0.44,0.63}{#1}}
\newcommand{\VariableTok}[1]{\textcolor[rgb]{0.10,0.09,0.49}{#1}}
\newcommand{\VerbatimStringTok}[1]{\textcolor[rgb]{0.25,0.44,0.63}{#1}}
\newcommand{\WarningTok}[1]{\textcolor[rgb]{0.38,0.63,0.69}{\textbf{\textit{#1}}}}
\setlength{\emergencystretch}{3em}  % prevent overfull lines
\providecommand{\tightlist}{%
  \setlength{\itemsep}{0pt}\setlength{\parskip}{0pt}}
\setcounter{secnumdepth}{0}
% Redefines (sub)paragraphs to behave more like sections
\ifx\paragraph\undefined\else
\let\oldparagraph\paragraph
\renewcommand{\paragraph}[1]{\oldparagraph{#1}\mbox{}}
\fi
\ifx\subparagraph\undefined\else
\let\oldsubparagraph\subparagraph
\renewcommand{\subparagraph}[1]{\oldsubparagraph{#1}\mbox{}}
\fi

% set default figure placement to htbp
\makeatletter
\def\fps@figure{htbp}
\makeatother


\date{}

\begin{document}

\oddsidemargin =-0.5in
\evensidemargin = -0.5in
\rightmargin = +0.5in
\textwidth = 8in

\hypertarget{csci-150-hw-class-design-practice}{%
\subsection{CSCI 150 HW: Dictionary and Class Homework
practice}\label{csci-150-hw-class-design-practice}}

\emph{Due: Wednesday,  April 13}

To receive full credit, for each exercise you should do the following:

\begin{enumerate}
\def\labelenumi{\arabic{enumi}.}
\item
  \textbf{Design}: First, design a Python class as requested in the
  exercise. Type in your class definition.
\item
  \textbf{Check}: Run the provided test code. Does your actual output
  agree with the given correct output?
\item
  \textbf{Evaluate}: If the actual output does not match the expected
  output, keep experimenting, consult the textbook or Python
  documentation, ask a friend or TA or professor, \emph{etc.} until you
  can fix your class definition and explain what your
  misunderstanding(s) were. (You do not need to do anything for step 3
  if the outputs already agree exactly.)
\end{enumerate}

You should consider the code in each exercise separately from the other
exercises.

\vspace{0.2in}

Upload your code to the class Teams page.

\begin{enumerate}
\item Write a function \verb|vowel_count| which takes in a list of strings (all lower case) and returns a dictionary, keyed on the five vowels a, e, i, o, and u. The value for each should be the total number of entries in the list which contain at least a single copy of that vowel. For example,
    
    \begin{verbatim}
      
      vowel_count(['time', 'is', 'the', 'end']) returns
      {'a': 0, 'e': 3, 'i': 2, 'o': 0, 'u': 0}
      
      
      vowel_count(['aardvark', 'facetious', 'too', 'to', 'two']) returns
      {'a': 2, 'e': 1, 'i': 1, 'o': 4', 'u': 1}
      
      (note that we are not counting the total number of a's, or e's, 
      but how many individual entries in the list contain at least one a)
    \end{verbatim}


\item  Write a function \texttt{char\_remaining} which takes a string as input and returns a dictionary whose keys are the characters in the string and values are the number of characters in the string remaining after the \emph{first} occurance of the keyed-character.  For example:

    \begin{verbatim}
      char_remaining('Hello') returns
       {'H': 4, 'e': 3, 'l': 2, 'o': 0}


      char_remaininf('CSCI 150 class') returns
       {'C': 13, 'S': 12, 'I': 10, ' ': 9, '1': 8, '5': 7, '0': 6, 'c': 4, 'l': 3, 'a': 2, 's':1}
    \end{verbatim}
    
\item  Write a function \verb|dict_count| which takes in two parameters: \verb|d|, which is a dictionary keyed on strings with integer values, and \verb|s| a single character string. The function should return an integer which is the sum of all values which correspond to keys which contain \verb|s|.  For example:
    
    \begin{verbatim}
      
      if d = {'hi': 3, 'there': 2, 'bye': 6} and s = 'h', then 
      dict(count(d,s)) returns 5
      
      if d = {'hi': 3, 'there': 2, 'bye': 6} and s = 'a', then
      dict(count(d,s)) returns 0
      
      if d = {'hi': 3, 'there': 2, 'bye': 6} and s = 'e', then
      dict(count(d,s)) returns 7
      
    \end{verbatim}



\def\labelenumi{\arabic{enumi}.}
\item
  Write a Python class \texttt{BouncyBall}, which represents a bouncy
  ball containing a certain amount of air.

  \begin{itemize}
  \tightlist
  \item
    When a \texttt{BouncyBall} object is first created, it should have
    10 units of air.
  \item
    There should be a method \texttt{bounce()} which normally prints the
    word \texttt{Bounce!} and decreases the amount of air in the ball by
    two units. However, if the amount of air is less than or equal to
    three, then \texttt{bounce()} does not decrease the amount of air
    and prints \texttt{Thupp.} instead of \texttt{Bounce!}.
  \item
    There should be a method \texttt{inflate()} which increases the
    amount of air by three units. If the amount of air ever becomes
    greater than 12, then the ball explodes by printing
    \texttt{BANG!!!}.
  \item
    You cannot bounce or inflate an exploded ball. After a ball
    explodes, calling \texttt{bounce()} or \texttt{inflate()} should
    just cause a message to be printed such as
    \texttt{Sorry,\ you\ cannot\ bounce\ this\ \ ball!\ \ It\ has\ exploded.}
  \end{itemize}

  To test your class, you can type in and run the following code:

\begin{Shaded}
\begin{Highlighting}[]
\KeywordTok{def}\NormalTok{ main():}
\NormalTok{    b }\OperatorTok{=}\NormalTok{ BouncyBall()}

    \ControlFlowTok{for}\NormalTok{ i }\KeywordTok{in} \BuiltInTok{range}\NormalTok{(}\DecValTok{6}\NormalTok{):}
\NormalTok{        b.bounce()}

\NormalTok{    b.inflate()}
\NormalTok{    b.bounce()}
\NormalTok{    b.bounce()}

    \ControlFlowTok{for}\NormalTok{ i }\KeywordTok{in} \BuiltInTok{range}\NormalTok{(}\DecValTok{5}\NormalTok{):}
\NormalTok{        b.inflate()}

\NormalTok{    b.bounce()}

\NormalTok{main()}
\end{Highlighting}
\end{Shaded}

  If your definition of \texttt{BouncyBall} is correct, \texttt{main()}
  should produce the following output:

\begin{verbatim}
Bounce!
Bounce!
Bounce!
Bounce!
Thupp.
Thupp.
Bounce!
Thupp.
BANG!!!
Sorry, you cannot inflate this ball!  It has exploded.
Sorry, you cannot bounce this ball!  It has exploded.
\end{verbatim}
\item
  Write a Python class \texttt{Gradebook} which works as follows:

  \begin{itemize}
  \tightlist
  \item
    When a new \texttt{Gradebook} object is first created, it should
    start out with an empty list of grades, and zero points of extra
    credit.
  \item
    There should be a method \texttt{add\_grade(g:\ int)} which adds the
    grade \texttt{g} to the end of the list.
  \item
    There should be a method \texttt{add\_ec(ec:\ int)} which adds
    \texttt{ec} points of extra credit to the current amount of extra
    credit.
  \item
    There should be a method \texttt{average()} which computes and
    returns the average of all the grades so far (the sum of all the
    grades, plus the extra credit score, divided by the number of
    grades).
  \end{itemize}

  You can test your implementation of \texttt{Gradebook} by running the
  code below:

\begin{Shaded}
\begin{Highlighting}[]
\KeywordTok{def}\NormalTok{ main():}
\NormalTok{    gb }\OperatorTok{=}\NormalTok{ Gradebook()}
\NormalTok{    gb.add_grade(}\DecValTok{90}\NormalTok{)}
\NormalTok{    gb.add_grade(}\DecValTok{83}\NormalTok{)}
\NormalTok{    gb.add_grade(}\DecValTok{97}\NormalTok{)}
\NormalTok{    gb.add_ec(}\DecValTok{10}\NormalTok{)}

    \BuiltInTok{print}\NormalTok{(gb.average())}

\NormalTok{main()}
\end{Highlighting}
\end{Shaded}

  If your definition of \texttt{Gradebook} is correct, this should print
  \texttt{93.33333333333333}.
\end{enumerate}

\vspace{0.2in}

\textbf{Specifications:}

To earn a \textbf{Complete} on this assignment, you need to:
\begin{itemize}
  \item All five problems are attempted
  \item Four of the five are completely correct; the fifth has a minor, non-obvious error not easily detectible through simple testing.
  \item  There are no \verb|return| vs \verb|print| issues.
\end{itemize}

To earn a \textbf{Partially Complete}, you need to:
\begin{itemize}
  \item  At least four of five are attempted
  \item  At least two are completely correct
  \item  At least one class problem has a correct \verb|__init__| method
  \item  There are no \verb|return| vs \verb|print| issues. 
\end{itemize}

\end{document} 