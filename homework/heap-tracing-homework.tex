% -*- mode: LaTeX; compile-command: "pdflatex exam1.tex" -*-
\documentclass{article}
\usepackage{amssymb}
\usepackage{amsmath}
\usepackage{multicol}
\usepackage{graphicx}
\usepackage{float}

\usepackage{amsmath, url, array}

\oddsidemargin -.5in
\evensidemargin -.5in
\textwidth 7in
\topmargin -.75in
\textheight 9in
\pagestyle{empty}
\setlength{\parindent}{0in}
\setlength{\unitlength}{0.85in}

\begin{document}

\textbf{CSCI 150 }

\vspace{0.1in}

\begin{center}
  \textbf{Tracing with the Heap}
\end{center}

\vspace{0.2in}
\textbf{Due Wednesday, April 6}

\vspace{0.2in}

\textbf{Grading Specifications:}

You will earn a \textbf{Complete} provide that:
\begin{itemize}
  \item all five problems are attempted,
  \item at least three a completely correct, showing all changes in the stack and heap as appropriate,
  \item a fourth has minor computational/careless error, but shows correct interactions with the heap, and
  \item on all five, there is no confusion about print vs return values.
\end{itemize}

You will earn a \textbf{Partially Complete} provided that:
\begin{itemize}
  \item  at least four are attempted,
  \item  on at least three, interactions with the heap are correct, except for minor computational errors, and
  \item  there is no confusion on at least two about print vs return values
\end{itemize}

You are \textbf{STRONGLY} encouraged to copy the code shown into a .py file or Kaggle notebook and run them. See if you printed output matches. Add additional print statements along the way to see if you are updating the heap/stack variables correctly. If you are confused, seek help from classmates, the CSCI tutors, or your instructor.

\vspace{0.2in}
For all, you might find it useful to include
\vspace{0.1in}

\verb|from typing import *|

\vspace{0.1in}

as the first line of your .py file or first cell in your Kaggle notebook



\begin{enumerate}

\item  \

\begin{verbatim}
def main1():
    a_list = [1, 2, 3]
    b_list = a_list
    temp_list = []
    for item in a_list:
        temp_list.append(item * 2)

    b_list.append(47)
    a_list[1] = -7

    print(a_list)
    print(b_list)
    print(temp_list)

main1()
\end{verbatim}
\newpage

\item  \

\begin{verbatim}
def main2():
    a_dict = {1: 'cat', 2: 'dog', 34: 'fish'}
    b_dict = a_dict
    temp_dict = {}
    for item in a_dict:
        temp_dict[item] = a_dict[item] + '!!'

    b_dict[100] = 'pig'
    a_dict[2] = 'snail'

    print(a_dict)
    print(b_dict)
    print(temp_dict)

main2()
\end{verbatim}
\vfill

\item \
\begin{verbatim}
def main3():
    a_str = 'bye'
    b_str = a_str
    temp_str = ''
    for item in a_str:
        temp_str += item + '!'

    b_str += 'z'


    print(a_str)
    print(b_str)
    print(temp_str)

main3()
\end{verbatim}
\vfill
\vfill
\newpage

\item \

\begin{verbatim}
def f1(a: Dict[str, int]) -> int:
    sum1 = 0
    for key in a:
        if a[key] >= 0:
            sum1 += a[key]
        else:
            a[key] = 0

    return sum1

def main4():
    b = {'Seme' : 23, 'Ferrer' : 12, 'Wilson' : -7}
    print(f1(b))
    print(b)

main4()
\end{verbatim}
\vfill

\item \
\begin{verbatim}
def g1(s: str) -> int:
    if 'a' in s:
        s = 'boo'

    return len(s)

def g2(lst: List[int]):
    s = 'exam'

    if len(lst) < len(s):
        print('Too short')
    else:
        i = 0
        for char in s:
            lst[i] = g1(char)
            i += 1


def main5():
    a_list = [6,2,9,8]
    g2(a_list)
    print(a_list)

main5()
\end{verbatim}

\vfill
\vfill


\end{enumerate}

\end{document} 