\documentclass[]{article}
\usepackage{lmodern}
\usepackage{amssymb,amsmath}
\usepackage{ifxetex,ifluatex}
\usepackage{fixltx2e} % provides \textsubscript
\ifnum 0\ifxetex 1\fi\ifluatex 1\fi=0 % if pdftex
  \usepackage[T1]{fontenc}
  \usepackage[utf8]{inputenc}
\else % if luatex or xelatex
  \ifxetex
    \usepackage{mathspec}
  \else
    \usepackage{fontspec}
  \fi
  \defaultfontfeatures{Ligatures=TeX,Scale=MatchLowercase}
\fi
% use upquote if available, for straight quotes in verbatim environments
\IfFileExists{upquote.sty}{\usepackage{upquote}}{}
% use microtype if available
\IfFileExists{microtype.sty}{%
\usepackage{microtype}
\UseMicrotypeSet[protrusion]{basicmath} % disable protrusion for tt fonts
}{}
\usepackage[unicode=true]{hyperref}
\hypersetup{
            pdfborder={0 0 0},
            breaklinks=true}
\urlstyle{same}  % don't use monospace font for urls
\usepackage{color}
\usepackage{fancyvrb}
\newcommand{\VerbBar}{|}
\newcommand{\VERB}{\Verb[commandchars=\\\{\}]}
\DefineVerbatimEnvironment{Highlighting}{Verbatim}{commandchars=\\\{\}}
% Add ',fontsize=\small' for more characters per line
\newenvironment{Shaded}{}{}
\newcommand{\KeywordTok}[1]{\textcolor[rgb]{0.00,0.44,0.13}{\textbf{{#1}}}}
\newcommand{\DataTypeTok}[1]{\textcolor[rgb]{0.56,0.13,0.00}{{#1}}}
\newcommand{\DecValTok}[1]{\textcolor[rgb]{0.25,0.63,0.44}{{#1}}}
\newcommand{\BaseNTok}[1]{\textcolor[rgb]{0.25,0.63,0.44}{{#1}}}
\newcommand{\FloatTok}[1]{\textcolor[rgb]{0.25,0.63,0.44}{{#1}}}
\newcommand{\ConstantTok}[1]{\textcolor[rgb]{0.53,0.00,0.00}{{#1}}}
\newcommand{\CharTok}[1]{\textcolor[rgb]{0.25,0.44,0.63}{{#1}}}
\newcommand{\SpecialCharTok}[1]{\textcolor[rgb]{0.25,0.44,0.63}{{#1}}}
\newcommand{\StringTok}[1]{\textcolor[rgb]{0.25,0.44,0.63}{{#1}}}
\newcommand{\VerbatimStringTok}[1]{\textcolor[rgb]{0.25,0.44,0.63}{{#1}}}
\newcommand{\SpecialStringTok}[1]{\textcolor[rgb]{0.73,0.40,0.53}{{#1}}}
\newcommand{\ImportTok}[1]{{#1}}
\newcommand{\CommentTok}[1]{\textcolor[rgb]{0.38,0.63,0.69}{\textit{{#1}}}}
\newcommand{\DocumentationTok}[1]{\textcolor[rgb]{0.73,0.13,0.13}{\textit{{#1}}}}
\newcommand{\AnnotationTok}[1]{\textcolor[rgb]{0.38,0.63,0.69}{\textbf{\textit{{#1}}}}}
\newcommand{\CommentVarTok}[1]{\textcolor[rgb]{0.38,0.63,0.69}{\textbf{\textit{{#1}}}}}
\newcommand{\OtherTok}[1]{\textcolor[rgb]{0.00,0.44,0.13}{{#1}}}
\newcommand{\FunctionTok}[1]{\textcolor[rgb]{0.02,0.16,0.49}{{#1}}}
\newcommand{\VariableTok}[1]{\textcolor[rgb]{0.10,0.09,0.49}{{#1}}}
\newcommand{\ControlFlowTok}[1]{\textcolor[rgb]{0.00,0.44,0.13}{\textbf{{#1}}}}
\newcommand{\OperatorTok}[1]{\textcolor[rgb]{0.40,0.40,0.40}{{#1}}}
\newcommand{\BuiltInTok}[1]{{#1}}
\newcommand{\ExtensionTok}[1]{{#1}}
\newcommand{\PreprocessorTok}[1]{\textcolor[rgb]{0.74,0.48,0.00}{{#1}}}
\newcommand{\AttributeTok}[1]{\textcolor[rgb]{0.49,0.56,0.16}{{#1}}}
\newcommand{\RegionMarkerTok}[1]{{#1}}
\newcommand{\InformationTok}[1]{\textcolor[rgb]{0.38,0.63,0.69}{\textbf{\textit{{#1}}}}}
\newcommand{\WarningTok}[1]{\textcolor[rgb]{0.38,0.63,0.69}{\textbf{\textit{{#1}}}}}
\newcommand{\AlertTok}[1]{\textcolor[rgb]{1.00,0.00,0.00}{\textbf{{#1}}}}
\newcommand{\ErrorTok}[1]{\textcolor[rgb]{1.00,0.00,0.00}{\textbf{{#1}}}}
\newcommand{\NormalTok}[1]{{#1}}
\IfFileExists{parskip.sty}{%
\usepackage{parskip}
}{% else
\setlength{\parindent}{0pt}
\setlength{\parskip}{6pt plus 2pt minus 1pt}
}
\setlength{\emergencystretch}{3em}  % prevent overfull lines
\providecommand{\tightlist}{%
  \setlength{\itemsep}{0pt}\setlength{\parskip}{0pt}}
\setcounter{secnumdepth}{0}
% Redefines (sub)paragraphs to behave more like sections
\ifx\paragraph\undefined\else
\let\oldparagraph\paragraph
\renewcommand{\paragraph}[1]{\oldparagraph{#1}\mbox{}}
\fi
\ifx\subparagraph\undefined\else
\let\oldsubparagraph\subparagraph
\renewcommand{\subparagraph}[1]{\oldsubparagraph{#1}\mbox{}}
\fi

\date{}

\begin{document}

\subsection{CSCI 150 HW: function, loop, and string reading
practice}\label{csci-150-hw-function-loop-and-string-reading-practice}

\emph{Due: Monday, February 26}

To receive full credit, for each exercise you should do the following:

\begin{enumerate}
\def\labelenumi{\arabic{enumi}.}
\item
  \textbf{Predict}: First, complete the exercise \emph{without} using
  the Python interpreter. (You are welcome to refer to your notes or
  textbook, read Python documentation, look at examples from class,
  \emph{etc.}; just don't actually run any code.) \emph{Trace the
  execution of the code, keeping track of the function stack, all
  variables, and any output produced.}
\item
  \textbf{Check}: Run the code. Does the actual output agree with what
  you wrote down in step 1?
\item
  \textbf{Evaluate}: If your answer to part 1 was different than the
  actual output, keep experimenting with it, consult the textbook or
  Python documentation, ask a friend or TA or professor, \emph{etc.}
  until you can explain why the code works the way it does \emph{and}
  what your misunderstanding(s) were in part 1. (You do not need to do
  anything for step 3 if the output agrees exactly with what you wrote
  in step 1.)
\end{enumerate}

You should consider the code in each exercise separately from the other
exercises.

\begin{enumerate}
\def\labelenumi{\arabic{enumi}.}
\item
  Trace the execution of the following code.

\begin{Shaded}
\begin{Highlighting}[]
\KeywordTok{def} \NormalTok{f(n: }\BuiltInTok{int}\NormalTok{) }\OperatorTok{->} \BuiltInTok{str}\NormalTok{:}
    \NormalTok{n }\OperatorTok{=} \DecValTok{2} \OperatorTok{*} \NormalTok{n }\OperatorTok{+} \DecValTok{1}
    \ControlFlowTok{return} \BuiltInTok{str}\NormalTok{(n)}

\KeywordTok{def} \NormalTok{g(n: }\BuiltInTok{int}\NormalTok{):}
    \NormalTok{s }\OperatorTok{=} \NormalTok{f(n) }\OperatorTok{+} \NormalTok{f(n}\DecValTok{+2}\NormalTok{)}
    \BuiltInTok{print}\NormalTok{(s)}
    \BuiltInTok{print}\NormalTok{(}\StringTok{"n is "} \OperatorTok{+} \BuiltInTok{str}\NormalTok{(n))}

\KeywordTok{def} \NormalTok{main():}
    \NormalTok{g(}\DecValTok{7}\NormalTok{)}
    \NormalTok{g(}\DecValTok{2}\NormalTok{)}

\NormalTok{main()}
\end{Highlighting}
\end{Shaded}

  The above code contains one trap for the unwary; what is it?
\item
  Trace the execution of the following code.

\begin{Shaded}
\begin{Highlighting}[]
\KeywordTok{def} \NormalTok{q(n: }\BuiltInTok{int}\NormalTok{) }\OperatorTok{->} \BuiltInTok{str}\NormalTok{:}
    \NormalTok{s }\OperatorTok{=} \StringTok{'TIPNR'}
    \ControlFlowTok{return} \NormalTok{s[n }\OperatorTok{%} \DecValTok{5}\NormalTok{]}

\KeywordTok{def} \NormalTok{m():}
    \NormalTok{i }\OperatorTok{=} \DecValTok{2}
    \NormalTok{count }\OperatorTok{=} \DecValTok{0}
    \NormalTok{s }\OperatorTok{=} \StringTok{''}
    \ControlFlowTok{while} \NormalTok{count }\OperatorTok{<} \DecValTok{5}\NormalTok{:}
        \NormalTok{s }\OperatorTok{+=} \NormalTok{q(i)}
        \NormalTok{i }\OperatorTok{+=} \DecValTok{2}
        \NormalTok{count }\OperatorTok{+=} \DecValTok{1}
    \BuiltInTok{print}\NormalTok{(s)}

\NormalTok{m()}
\end{Highlighting}
\end{Shaded}
\item
  What is printed by the following code? Hint: read about the
  \texttt{find} function here:
  \href{https://docs.python.org/3/library/stdtypes.html?highlight=find\#str.find}{\texttt{https://docs.python.org/3/library/stdtypes.html?highlight=find\#str.find}},
  and try some examples to make sure you understand what it does.

\begin{Shaded}
\begin{Highlighting}[]
\NormalTok{s }\OperatorTok{=} \StringTok{'thesethickthornythistlethingsthrivethroughoutthethicket'}

\NormalTok{pos }\OperatorTok{=} \DecValTok{0}
\ControlFlowTok{while} \NormalTok{pos }\OperatorTok{!=} \OperatorTok{-}\DecValTok{1}\NormalTok{:}
    \BuiltInTok{next} \OperatorTok{=} \NormalTok{s.find(}\StringTok{'th'}\NormalTok{,pos}\DecValTok{+1}\NormalTok{)}
    \ControlFlowTok{if} \BuiltInTok{next} \OperatorTok{!=} \OperatorTok{-}\DecValTok{1}\NormalTok{:}
        \BuiltInTok{print}\NormalTok{(s[pos:}\BuiltInTok{next}\NormalTok{])}
    \ControlFlowTok{else}\NormalTok{:}
        \BuiltInTok{print}\NormalTok{(s[pos:])}
    \NormalTok{pos }\OperatorTok{=} \BuiltInTok{next}
\end{Highlighting}
\end{Shaded}
\end{enumerate}

\end{document}
