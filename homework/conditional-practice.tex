\documentclass[]{article}
\usepackage{lmodern}
\usepackage{amssymb,amsmath}
\usepackage{ifxetex,ifluatex}
\usepackage{fixltx2e} % provides \textsubscript
\ifnum 0\ifxetex 1\fi\ifluatex 1\fi=0 % if pdftex
  \usepackage[T1]{fontenc}
  \usepackage[utf8]{inputenc}
\else % if luatex or xelatex
  \ifxetex
    \usepackage{mathspec}
  \else
    \usepackage{fontspec}
  \fi
  \defaultfontfeatures{Ligatures=TeX,Scale=MatchLowercase}
\fi
% use upquote if available, for straight quotes in verbatim environments
\IfFileExists{upquote.sty}{\usepackage{upquote}}{}
% use microtype if available
\IfFileExists{microtype.sty}{%
\usepackage{microtype}
\UseMicrotypeSet[protrusion]{basicmath} % disable protrusion for tt fonts
}{}
\usepackage[unicode=true]{hyperref}
\hypersetup{
            pdfborder={0 0 0},
            breaklinks=true}
\urlstyle{same}  % don't use monospace font for urls
\IfFileExists{parskip.sty}{%
\usepackage{parskip}
}{% else
\setlength{\parindent}{0pt}
\setlength{\parskip}{6pt plus 2pt minus 1pt}
}
\setlength{\emergencystretch}{3em}  % prevent overfull lines
\providecommand{\tightlist}{%
  \setlength{\itemsep}{0pt}\setlength{\parskip}{0pt}}
\setcounter{secnumdepth}{0}
% Redefines (sub)paragraphs to behave more like sections
\ifx\paragraph\undefined\else
\let\oldparagraph\paragraph
\renewcommand{\paragraph}[1]{\oldparagraph{#1}\mbox{}}
\fi
\ifx\subparagraph\undefined\else
\let\oldsubparagraph\subparagraph
\renewcommand{\subparagraph}[1]{\oldsubparagraph{#1}\mbox{}}
\fi

\date{}

\begin{document}

\subsection{CSCI 150 HW: conditional
practice}\label{csci-150-hw-conditional-practice}

\emph{Due: Monday, September 10}

\begin{enumerate}
\def\labelenumi{\arabic{enumi}.}
\item
  Find at least 5 errors in the following code. For each error, explain
  what is wrong and how you would fix it.

\begin{verbatim}
fruit = print("What is your favorite fruit?)
number = int(input("Pick a number between 1 and 10")
if fruit == "banana":
   print("Yes, we have no bananas.")
elif fruit == "apple" and number > 4 or < 20
   print("Soo many apppplles.")
if fruit == "pear":
   elif number > 6:
   print("We need to pear that down a bit.")
   else:
      print("Pearfect!")
else fruit == "blackberry":
   print("My favorite!")
\end{verbatim}
\end{enumerate}

For each of the following scenarios, write some Python code to generate
the intended output.

\begin{enumerate}
\def\labelenumi{\arabic{enumi}.}
\setcounter{enumi}{1}
\item
  Assume there is a variable \texttt{s} which contains a string. If the
  string comes before \texttt{f} in the dictionary, print \texttt{Fizz}.
  If the string is after \texttt{b} in the dictionary, print
  \texttt{Buzz}. If both the \texttt{f} and \texttt{b} conditions are
  true, print \texttt{FizzBuzz}. In all other cases, print the string
  \texttt{s} unchanged.
\item
  We are having a party with amounts of tea and candy; assume there are
  variables named \texttt{tea} and \texttt{candy} which contain
  integers. Print the outcome of the party: either \texttt{bad},
  \texttt{good}, or \texttt{great}. A party is good if both tea and
  candy are at least 5. However, if either tea or candy is at least
  double the amount of the other one, the party is not just good but
  great. In all cases, if either tea or candy is less than 5, the party
  is always bad.

  For example,

  \begin{itemize}
  \tightlist
  \item
    If \texttt{tea\ ==\ 3} and \texttt{candy\ ==\ 7}, you should print
    \texttt{bad}
  \item
    If \texttt{tea\ ==\ 6} and \texttt{candy\ ==\ 5}, you should print
    \texttt{good}
  \item
    If \texttt{tea\ ==\ 6} and \texttt{candy\ ==\ 12}, you should print
    \texttt{great}
  \end{itemize}
\end{enumerate}

\end{document}
