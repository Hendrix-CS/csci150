\PassOptionsToPackage{unicode=true}{hyperref} % options for packages loaded elsewhere
\PassOptionsToPackage{hyphens}{url}
%
\documentclass[]{article}
\usepackage{lmodern}
\usepackage{amssymb,amsmath}
\usepackage{ifxetex,ifluatex}
\usepackage{fixltx2e} % provides \textsubscript
\ifnum 0\ifxetex 1\fi\ifluatex 1\fi=0 % if pdftex
  \usepackage[T1]{fontenc}
  \usepackage[utf8]{inputenc}
  \usepackage{textcomp} % provides euro and other symbols
\else % if luatex or xelatex
  \usepackage{unicode-math}
  \defaultfontfeatures{Ligatures=TeX,Scale=MatchLowercase}
\fi
% use upquote if available, for straight quotes in verbatim environments
\IfFileExists{upquote.sty}{\usepackage{upquote}}{}
% use microtype if available
\IfFileExists{microtype.sty}{%
\usepackage[]{microtype}
\UseMicrotypeSet[protrusion]{basicmath} % disable protrusion for tt fonts
}{}
\IfFileExists{parskip.sty}{%
\usepackage{parskip}
}{% else
\setlength{\parindent}{0pt}
\setlength{\parskip}{6pt plus 2pt minus 1pt}
}
\usepackage{hyperref}
\hypersetup{
            pdfborder={0 0 0},
            breaklinks=true}
\urlstyle{same}  % don't use monospace font for urls
\usepackage{color}
\usepackage{fancyvrb}
\newcommand{\VerbBar}{|}
\newcommand{\VERB}{\Verb[commandchars=\\\{\}]}
\DefineVerbatimEnvironment{Highlighting}{Verbatim}{commandchars=\\\{\}}
% Add ',fontsize=\small' for more characters per line
\newenvironment{Shaded}{}{}
\newcommand{\AlertTok}[1]{\textcolor[rgb]{1.00,0.00,0.00}{\textbf{#1}}}
\newcommand{\AnnotationTok}[1]{\textcolor[rgb]{0.38,0.63,0.69}{\textbf{\textit{#1}}}}
\newcommand{\AttributeTok}[1]{\textcolor[rgb]{0.49,0.56,0.16}{#1}}
\newcommand{\BaseNTok}[1]{\textcolor[rgb]{0.25,0.63,0.44}{#1}}
\newcommand{\BuiltInTok}[1]{#1}
\newcommand{\CharTok}[1]{\textcolor[rgb]{0.25,0.44,0.63}{#1}}
\newcommand{\CommentTok}[1]{\textcolor[rgb]{0.38,0.63,0.69}{\textit{#1}}}
\newcommand{\CommentVarTok}[1]{\textcolor[rgb]{0.38,0.63,0.69}{\textbf{\textit{#1}}}}
\newcommand{\ConstantTok}[1]{\textcolor[rgb]{0.53,0.00,0.00}{#1}}
\newcommand{\ControlFlowTok}[1]{\textcolor[rgb]{0.00,0.44,0.13}{\textbf{#1}}}
\newcommand{\DataTypeTok}[1]{\textcolor[rgb]{0.56,0.13,0.00}{#1}}
\newcommand{\DecValTok}[1]{\textcolor[rgb]{0.25,0.63,0.44}{#1}}
\newcommand{\DocumentationTok}[1]{\textcolor[rgb]{0.73,0.13,0.13}{\textit{#1}}}
\newcommand{\ErrorTok}[1]{\textcolor[rgb]{1.00,0.00,0.00}{\textbf{#1}}}
\newcommand{\ExtensionTok}[1]{#1}
\newcommand{\FloatTok}[1]{\textcolor[rgb]{0.25,0.63,0.44}{#1}}
\newcommand{\FunctionTok}[1]{\textcolor[rgb]{0.02,0.16,0.49}{#1}}
\newcommand{\ImportTok}[1]{#1}
\newcommand{\InformationTok}[1]{\textcolor[rgb]{0.38,0.63,0.69}{\textbf{\textit{#1}}}}
\newcommand{\KeywordTok}[1]{\textcolor[rgb]{0.00,0.44,0.13}{\textbf{#1}}}
\newcommand{\NormalTok}[1]{#1}
\newcommand{\OperatorTok}[1]{\textcolor[rgb]{0.40,0.40,0.40}{#1}}
\newcommand{\OtherTok}[1]{\textcolor[rgb]{0.00,0.44,0.13}{#1}}
\newcommand{\PreprocessorTok}[1]{\textcolor[rgb]{0.74,0.48,0.00}{#1}}
\newcommand{\RegionMarkerTok}[1]{#1}
\newcommand{\SpecialCharTok}[1]{\textcolor[rgb]{0.25,0.44,0.63}{#1}}
\newcommand{\SpecialStringTok}[1]{\textcolor[rgb]{0.73,0.40,0.53}{#1}}
\newcommand{\StringTok}[1]{\textcolor[rgb]{0.25,0.44,0.63}{#1}}
\newcommand{\VariableTok}[1]{\textcolor[rgb]{0.10,0.09,0.49}{#1}}
\newcommand{\VerbatimStringTok}[1]{\textcolor[rgb]{0.25,0.44,0.63}{#1}}
\newcommand{\WarningTok}[1]{\textcolor[rgb]{0.38,0.63,0.69}{\textbf{\textit{#1}}}}
\setlength{\emergencystretch}{3em}  % prevent overfull lines
\providecommand{\tightlist}{%
  \setlength{\itemsep}{0pt}\setlength{\parskip}{0pt}}
\setcounter{secnumdepth}{0}
% Redefines (sub)paragraphs to behave more like sections
\ifx\paragraph\undefined\else
\let\oldparagraph\paragraph
\renewcommand{\paragraph}[1]{\oldparagraph{#1}\mbox{}}
\fi
\ifx\subparagraph\undefined\else
\let\oldsubparagraph\subparagraph
\renewcommand{\subparagraph}[1]{\oldsubparagraph{#1}\mbox{}}
\fi

% set default figure placement to htbp
\makeatletter
\def\fps@figure{htbp}
\makeatother


\date{}

\begin{document}

\hypertarget{csci-150-hw-dictionary-heap-reading-practice}{%
\subsection{CSCI 150 HW: dictionary + heap reading
practice}\label{csci-150-hw-dictionary-heap-reading-practice}}

\emph{Due: Wednesday, November 6}

To receive full credit, for each exercise you should do the following:

\begin{enumerate}
\def\labelenumi{\arabic{enumi}.}
\item
  \textbf{Predict}: First, complete the exercise \emph{without} using
  the Python interpreter. \emph{Trace the execution of the code, keeping
  track of the function stack, the heap, all variables, and any output
  produced.}

  If you like, you may print out copies of the tracing template found at
  \href{http://ozark.hendrix.edu/~yorgey/150/static/heap-tracing-template.pdf}{\texttt{http://ozark.hendrix.edu/\textasciitilde{}yorgey/150/static/heap-tracing-template.pdf}}
  (if you have this PDF open on your computer you may simply click the
  link above).
\item
  \textbf{Check}: Run the code. Does the actual output agree with what
  you wrote down in step 1?
\item
  \textbf{Evaluate}: If your answer to part 1 was different than the
  actual output, keep experimenting with it, consult the textbook or
  Python documentation, ask a friend or TA or professor, \emph{etc.}
  until you can explain why the code works the way it does \emph{and}
  what your misunderstanding(s) were in part 1.
\end{enumerate}

You should consider the code in each exercise separately from the other
exercises.

This line is intentionally left blank.

\begin{enumerate}
\def\labelenumi{\arabic{enumi}.}
\item
  Trace the execution of the following code.

\begin{Shaded}
\begin{Highlighting}[]
\KeywordTok{def}\NormalTok{ alice(lst : List[}\BuiltInTok{int}\NormalTok{], n : }\BuiltInTok{int}\NormalTok{):}

\NormalTok{    len_lst : }\BuiltInTok{int} \OperatorTok{=} \BuiltInTok{len}\NormalTok{(lst)}
    \ControlFlowTok{for}\NormalTok{ i }\KeywordTok{in} \BuiltInTok{range}\NormalTok{(len_lst):}
\NormalTok{        lst[i] }\OperatorTok{=}\NormalTok{ lst[i] }\OperatorTok{%}\NormalTok{ n}

\KeywordTok{def}\NormalTok{ bob(lst: List[}\BuiltInTok{int}\NormalTok{]) }\OperatorTok{->} \BuiltInTok{str}\NormalTok{:}
\NormalTok{    s }\OperatorTok{=} \StringTok{''}
\NormalTok{    letters: List[}\BuiltInTok{str}\NormalTok{] }\OperatorTok{=}\NormalTok{ [}\StringTok{'t'}\NormalTok{,}\StringTok{'e'}\NormalTok{,}\StringTok{'a'}\NormalTok{,}\StringTok{'h'}\NormalTok{,}\StringTok{'p'}\NormalTok{,}\StringTok{'!'}\NormalTok{,}\StringTok{'s'}\NormalTok{,}\StringTok{'y'}\NormalTok{]}
    \ControlFlowTok{for}\NormalTok{ num }\KeywordTok{in}\NormalTok{ lst:}
\NormalTok{        s }\OperatorTok{+=}\NormalTok{ letters[num]}

    \ControlFlowTok{return}\NormalTok{ s}

\KeywordTok{def}\NormalTok{ main():}

\NormalTok{    mylist }\OperatorTok{=}\NormalTok{ [}\DecValTok{3}\NormalTok{,}\DecValTok{25}\NormalTok{,}\DecValTok{14}\NormalTok{,}\OperatorTok{-}\DecValTok{2}\NormalTok{,}\DecValTok{5}\NormalTok{]}
\NormalTok{    alice(mylist,}\DecValTok{6}\NormalTok{)}
    \BuiltInTok{print}\NormalTok{(bob(mylist))}

\NormalTok{main()}
\end{Highlighting}
\end{Shaded}
\item
  Trace the execution of the following code.

\begin{Shaded}
\begin{Highlighting}[]
\KeywordTok{def}\NormalTok{ harry(s: }\BuiltInTok{str}\NormalTok{)}\OperatorTok{->}\NormalTok{ Dict[}\BuiltInTok{str}\NormalTok{,}\BuiltInTok{int}\NormalTok{]:}
\NormalTok{    return_dict }\OperatorTok{=}\NormalTok{ \{\}}
    \ControlFlowTok{for}\NormalTok{ char }\KeywordTok{in}\NormalTok{ s:}
\NormalTok{        return_dict[char] }\OperatorTok{=}\NormalTok{ s.find(char)}

    \ControlFlowTok{return}\NormalTok{ return_dict}

\KeywordTok{def}\NormalTok{ hermione(s: }\BuiltInTok{str}\NormalTok{, in_dict: Dict[}\BuiltInTok{str}\NormalTok{,}\BuiltInTok{int}\NormalTok{]):}
    \ControlFlowTok{for}\NormalTok{ char }\KeywordTok{in}\NormalTok{ s:}
        \ControlFlowTok{if}\NormalTok{ char }\KeywordTok{in}\NormalTok{ in_dict:}
\NormalTok{            in_dict[char] }\OperatorTok{+=}\NormalTok{ s.find(char)}
        \ControlFlowTok{else}\NormalTok{:}
\NormalTok{            in_dict[char] }\OperatorTok{=} \DecValTok{-5}

\KeywordTok{def}\NormalTok{ main1():}
\NormalTok{    s }\OperatorTok{=} \StringTok{'4 privet drive'}

\NormalTok{    main_dict }\OperatorTok{=}\NormalTok{ harry(s)}
\NormalTok{    hermione(}\StringTok{'hendrix'}\NormalTok{,main_dict)}
    \ControlFlowTok{for}\NormalTok{ key }\KeywordTok{in}\NormalTok{ main_dict:}
        \BuiltInTok{print}\NormalTok{(key }\OperatorTok{+} \StringTok{" "} \OperatorTok{+} \BuiltInTok{str}\NormalTok{(main_dict[key]))}

\NormalTok{main1()}
\end{Highlighting}
\end{Shaded}
\end{enumerate}

\end{document}
