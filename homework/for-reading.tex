\documentclass[]{article}
\usepackage{lmodern}
\usepackage{amssymb,amsmath}
\usepackage{ifxetex,ifluatex}
\usepackage{fixltx2e} % provides \textsubscript
\ifnum 0\ifxetex 1\fi\ifluatex 1\fi=0 % if pdftex
  \usepackage[T1]{fontenc}
  \usepackage[utf8]{inputenc}
\else % if luatex or xelatex
  \ifxetex
    \usepackage{mathspec}
  \else
    \usepackage{fontspec}
  \fi
  \defaultfontfeatures{Ligatures=TeX,Scale=MatchLowercase}
\fi
% use upquote if available, for straight quotes in verbatim environments
\IfFileExists{upquote.sty}{\usepackage{upquote}}{}
% use microtype if available
\IfFileExists{microtype.sty}{%
\usepackage[]{microtype}
\UseMicrotypeSet[protrusion]{basicmath} % disable protrusion for tt fonts
}{}
\PassOptionsToPackage{hyphens}{url} % url is loaded by hyperref
\usepackage[unicode=true]{hyperref}
\hypersetup{
            pdfborder={0 0 0},
            breaklinks=true}
\urlstyle{same}  % don't use monospace font for urls
\usepackage{color}
\usepackage{fancyvrb}
\newcommand{\VerbBar}{|}
\newcommand{\VERB}{\Verb[commandchars=\\\{\}]}
\DefineVerbatimEnvironment{Highlighting}{Verbatim}{commandchars=\\\{\}}
% Add ',fontsize=\small' for more characters per line
\newenvironment{Shaded}{}{}
\newcommand{\KeywordTok}[1]{\textcolor[rgb]{0.00,0.44,0.13}{\textbf{#1}}}
\newcommand{\DataTypeTok}[1]{\textcolor[rgb]{0.56,0.13,0.00}{#1}}
\newcommand{\DecValTok}[1]{\textcolor[rgb]{0.25,0.63,0.44}{#1}}
\newcommand{\BaseNTok}[1]{\textcolor[rgb]{0.25,0.63,0.44}{#1}}
\newcommand{\FloatTok}[1]{\textcolor[rgb]{0.25,0.63,0.44}{#1}}
\newcommand{\ConstantTok}[1]{\textcolor[rgb]{0.53,0.00,0.00}{#1}}
\newcommand{\CharTok}[1]{\textcolor[rgb]{0.25,0.44,0.63}{#1}}
\newcommand{\SpecialCharTok}[1]{\textcolor[rgb]{0.25,0.44,0.63}{#1}}
\newcommand{\StringTok}[1]{\textcolor[rgb]{0.25,0.44,0.63}{#1}}
\newcommand{\VerbatimStringTok}[1]{\textcolor[rgb]{0.25,0.44,0.63}{#1}}
\newcommand{\SpecialStringTok}[1]{\textcolor[rgb]{0.73,0.40,0.53}{#1}}
\newcommand{\ImportTok}[1]{#1}
\newcommand{\CommentTok}[1]{\textcolor[rgb]{0.38,0.63,0.69}{\textit{#1}}}
\newcommand{\DocumentationTok}[1]{\textcolor[rgb]{0.73,0.13,0.13}{\textit{#1}}}
\newcommand{\AnnotationTok}[1]{\textcolor[rgb]{0.38,0.63,0.69}{\textbf{\textit{#1}}}}
\newcommand{\CommentVarTok}[1]{\textcolor[rgb]{0.38,0.63,0.69}{\textbf{\textit{#1}}}}
\newcommand{\OtherTok}[1]{\textcolor[rgb]{0.00,0.44,0.13}{#1}}
\newcommand{\FunctionTok}[1]{\textcolor[rgb]{0.02,0.16,0.49}{#1}}
\newcommand{\VariableTok}[1]{\textcolor[rgb]{0.10,0.09,0.49}{#1}}
\newcommand{\ControlFlowTok}[1]{\textcolor[rgb]{0.00,0.44,0.13}{\textbf{#1}}}
\newcommand{\OperatorTok}[1]{\textcolor[rgb]{0.40,0.40,0.40}{#1}}
\newcommand{\BuiltInTok}[1]{#1}
\newcommand{\ExtensionTok}[1]{#1}
\newcommand{\PreprocessorTok}[1]{\textcolor[rgb]{0.74,0.48,0.00}{#1}}
\newcommand{\AttributeTok}[1]{\textcolor[rgb]{0.49,0.56,0.16}{#1}}
\newcommand{\RegionMarkerTok}[1]{#1}
\newcommand{\InformationTok}[1]{\textcolor[rgb]{0.38,0.63,0.69}{\textbf{\textit{#1}}}}
\newcommand{\WarningTok}[1]{\textcolor[rgb]{0.38,0.63,0.69}{\textbf{\textit{#1}}}}
\newcommand{\AlertTok}[1]{\textcolor[rgb]{1.00,0.00,0.00}{\textbf{#1}}}
\newcommand{\ErrorTok}[1]{\textcolor[rgb]{1.00,0.00,0.00}{\textbf{#1}}}
\newcommand{\NormalTok}[1]{#1}
\IfFileExists{parskip.sty}{%
\usepackage{parskip}
}{% else
\setlength{\parindent}{0pt}
\setlength{\parskip}{6pt plus 2pt minus 1pt}
}
\setlength{\emergencystretch}{3em}  % prevent overfull lines
\providecommand{\tightlist}{%
  \setlength{\itemsep}{0pt}\setlength{\parskip}{0pt}}
\setcounter{secnumdepth}{0}
% Redefines (sub)paragraphs to behave more like sections
\ifx\paragraph\undefined\else
\let\oldparagraph\paragraph
\renewcommand{\paragraph}[1]{\oldparagraph{#1}\mbox{}}
\fi
\ifx\subparagraph\undefined\else
\let\oldsubparagraph\subparagraph
\renewcommand{\subparagraph}[1]{\oldsubparagraph{#1}\mbox{}}
\fi

% set default figure placement to htbp
\makeatletter
\def\fps@figure{htbp}
\makeatother


\date{}

\begin{document}

\subsection{CSCI 150 HW: for loop reading
practice}\label{csci-150-hw-for-loop-reading-practice}

\emph{Due: Wednesday, April 3}

To receive full credit, for each exercise you should do the following:

\begin{enumerate}
\def\labelenumi{\arabic{enumi}.}
\item
  \textbf{Predict}: First, complete the exercise \emph{without} using
  the Python interpreter. \emph{Trace the execution of the code, keeping
  track of the function stack, all variables, and any output produced.}
\item
  \textbf{Check}: Run the code. Does the actual output agree with what
  you wrote down in step 1?
\item
  \textbf{Evaluate}: If your answer to part 1 was different than the
  actual output, keep experimenting with it, consult the textbook or
  Python documentation, ask a friend or TA or professor, \emph{etc.}
  until you can explain why the code works the way it does \emph{and}
  what your misunderstanding(s) were in part 1.
\end{enumerate}

You should consider the code in each exercise separately from the other
exercises.

\begin{enumerate}
\def\labelenumi{\arabic{enumi}.}
\item
  Trace the execution of the following code.

\begin{Shaded}
\begin{Highlighting}[]
\KeywordTok{def}\NormalTok{ aaa(lst: List[}\BuiltInTok{int}\NormalTok{]) }\OperatorTok{->}\NormalTok{ List[}\BuiltInTok{int}\NormalTok{]:}
\NormalTok{    bbb: List[}\BuiltInTok{int}\NormalTok{] }\OperatorTok{=}\NormalTok{ []}
    \ControlFlowTok{for}\NormalTok{ i }\KeywordTok{in} \BuiltInTok{range}\NormalTok{(}\BuiltInTok{len}\NormalTok{(lst)):}
\NormalTok{        bbb.append(lst[}\BuiltInTok{len}\NormalTok{(lst) }\OperatorTok{-}\NormalTok{ i }\OperatorTok{-} \DecValTok{1}\NormalTok{])}

    \ControlFlowTok{return}\NormalTok{ bbb}

\KeywordTok{def}\NormalTok{ main():}
\NormalTok{    mynums: List[}\BuiltInTok{int}\NormalTok{] }\OperatorTok{=}\NormalTok{ [}\DecValTok{4}\NormalTok{,}\DecValTok{6}\NormalTok{,}\DecValTok{2}\NormalTok{,}\DecValTok{9}\NormalTok{]}
    \BuiltInTok{print}\NormalTok{(aaa(mynums))}

\NormalTok{main()}
\end{Highlighting}
\end{Shaded}
\item
  Trace the execution of the following code.

\begin{Shaded}
\begin{Highlighting}[]
\NormalTok{xs: List[}\BuiltInTok{int}\NormalTok{] }\OperatorTok{=}\NormalTok{ [}\DecValTok{0}\NormalTok{,}\DecValTok{1}\NormalTok{,}\DecValTok{2}\NormalTok{,}\DecValTok{3}\NormalTok{,}\DecValTok{4}\NormalTok{]}
\ControlFlowTok{for}\NormalTok{ i }\KeywordTok{in} \BuiltInTok{range}\NormalTok{(}\BuiltInTok{len}\NormalTok{(xs)):}
\NormalTok{    xs[i] }\OperatorTok{=} \DecValTok{2}\OperatorTok{*}\NormalTok{xs[i] }\OperatorTok{+} \DecValTok{1}

\NormalTok{s: }\BuiltInTok{int} \OperatorTok{=} \DecValTok{0}
\NormalTok{p: }\BuiltInTok{int} \OperatorTok{=} \DecValTok{1}
\ControlFlowTok{for}\NormalTok{ x }\KeywordTok{in}\NormalTok{ xs:}
\NormalTok{    s }\OperatorTok{+=}\NormalTok{ x}
\NormalTok{    p }\OperatorTok{*=}\NormalTok{ x}

\BuiltInTok{print}\NormalTok{(s)}
\BuiltInTok{print}\NormalTok{(p)}
\end{Highlighting}
\end{Shaded}
\item
  Trace the execution of the following code.

\begin{Shaded}
\begin{Highlighting}[]
\KeywordTok{def}\NormalTok{ q(n: }\BuiltInTok{int}\NormalTok{) }\OperatorTok{->} \BuiltInTok{str}\NormalTok{:}
\NormalTok{    s: }\BuiltInTok{str} \OperatorTok{=} \StringTok{'TIPNR'}
    \ControlFlowTok{return}\NormalTok{ s[n }\OperatorTok{%} \DecValTok{5}\NormalTok{]}

\KeywordTok{def}\NormalTok{ m():}
\NormalTok{    s: }\BuiltInTok{str} \OperatorTok{=} \StringTok{''}
    \ControlFlowTok{for}\NormalTok{ count }\KeywordTok{in} \BuiltInTok{range}\NormalTok{(}\DecValTok{1}\NormalTok{,}\DecValTok{6}\NormalTok{):}
\NormalTok{        s }\OperatorTok{+=}\NormalTok{ q(}\DecValTok{2}\OperatorTok{*}\NormalTok{count)}
    \BuiltInTok{print}\NormalTok{(s)}

\NormalTok{m()}
\end{Highlighting}
\end{Shaded}
\end{enumerate}

\end{document}
