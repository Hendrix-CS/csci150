% -*- mode: LaTeX; compile-command: "pdflatex exam3_inclass_formA.tex" -*-
\documentclass{article}

\usepackage{amsmath, url, graphicx, array}

\begin{document}

\title{CSCI 150: Exam 3 -- In Class}
\author{}
\date{Monday, April 17, 2023}

\maketitle

\thispagestyle{empty}

You will have 50 minutes to complete this exam. \textbf{You are not allowed
  to use your notes, textbook, phone or computer.  Partial credit for
  incorrect answers can be given only if you show your work.}
\bigskip

  \vspace{0.3in}

Print your name : \underline{\phantom{XXXXXXXXXXXXXXXXXXXXXXXXX}}
\vspace{1in}



\newpage

\textbf{For each piece of code shown, trace, showing the function
  stack and heap.}

\begin{enumerate}

\item \
\begin{verbatim}
def main1():
    a = [8, 9, 7]
    b = [8, 9, 7]

    c = {'a': 7, 'b': 9, 'c': 2}
    d = c
    i = 0
    for key in c:
        if c[key] in a:
            c[key] = i * 3
            b[i] = i * 3
        i += 1

    print(a)
    print(b)

    print(c)
    print(d)

    print(i)

main1()
\end{verbatim}




\newpage

\item \
\begin{verbatim}
def f1(d: Dict[str, int], s: str) -> int:
    n = 0
    for key in d:
        if s in key:
            n += d[key]

    return n


def main2():
    a = ['x', 'y', 'z']

    c = {'yoyo': 7, 'syzygy': 12}
    d = {}

    for char in a:
        d[char] = f1(c, char)

    print(a)

    print(c)
    print(d)

main2()

\end{verbatim}

\newpage
\item \

\begin{verbatim}
def aa(n: int) -> int:
    if n > 5:
        return n
    else:
        return n + 3


def main3():
    lst = [8, 2, 4, 9]
    mst = lst
    pst = []

    for i in range(len(mst)):
        pst.append(mst[i])
        mst[i] = aa(mst[i])

    print(lst)
    print(mst)
    print(pst)

main3()
\end{verbatim}

\eject
\item Write a function \verb|abc_ya| which takes in a lower case string (which might have spaces or punctuation) \verb|s|. It should return a dictionary keyed on the three letters \verb|'a', 'b', 'c'| and return for each a count of the number of occurrences of each in \verb|s|.

    For example:
    \begin{itemize}
      \item \verb|abc_ya('a boy ran away')| should return\\  \verb|{'a': 4, 'b': 1, 'c': 0}|
      \item \verb|abc_ya('xyz')| should return \\ \verb|{'a': 0, 'b': 0, 'c': 0}|
      \item \verb|abc_ya('abcdeabcdeabcde')| should return \\ \verb|{'a': 3, 'b': 3, 'c': 3}|
    \end{itemize}

    \vspace{0.3in}

\begin{verbatim}
def abc_ya(s: str) -> Dict[str, int]:
    # write your code here

\end{verbatim}
\newpage
\item Write a class \verb|Pool|, whose objects represent swimming pools. Each pool is built with a parameter \verb|capacity|, which can be any positive integer (you do not need to check that the given value is positive) and initialized with zero water. A pool should have the following attributes:
    \begin{itemize}
      \item \verb|capacity| -- set as a parameter when the pool is created
      \item \verb|amount| -- the current amount of water in the pool, which is initially set to 0.
    \end{itemize}

    Pools should have the following methods:
    \begin{itemize}
      \item \verb|fill(n: int)| which adds $n$ units of water to the pool. If more is added than the pool can hold, the extra overflows, and the value of \verb|amount| should be set to the maximum capacity. (You can assume that $n$ will always be positive here.)
      \item \verb|drain()| which removes all water from the pool.
      \item \verb|swim()| which prints %
        \verb|'It is a beautiful day for a swim!'| \emph{unless} the
        pool is currently filled with less than 10 units of water. In
        that case, it should print the statement
        \begin{quote}
        \texttt{The pool needs more water for you to swim. It currently contains xx},
        \end{quote}
        where \verb|xx| is the current amount of water in the pool.
    \end{itemize}

    The following is an example of the code running in the console:

\begin{verbatim}
>>> a = Pool(50)
>>> a.swim()
The pool needs more water for you to swim. It currently contains 0.
>>> a.fill(35)
>>> a.swim()
It is a beautiful day for a swim!
>>> a.drain()
>>> a.swim()
The pool needs more water for you to swim. It currently contains 0.
>>> a.fill(6)
>>> a.swim()
The pool needs more water for you to swim. It currently contains 6.
>>> a.fill(4)
>>> a.swim()
It is a beautiful day for a swim!
\end{verbatim}

\vfill
See Next Page!!!!!

\newpage
\begin{verbatim}
class Pool:

    def __init__(self, capacity: int):
        # write your code here








    def fill(self, n: int):
        # write your code here








    def drain(self):
        # write your code here










    def swim(self):
        # write your code here



\end{verbatim}


\end{enumerate}

\end{document} 