\documentclass{article}[10 pt]
\usepackage{amssymb}
\usepackage{amsmath}
\usepackage{multicol}
\usepackage{graphicx}

\oddsidemargin -.5in \evensidemargin -.5in \textwidth 7in \topmargin
-.75in \textheight 9in \pagestyle{empty}
\setlength{\unitlength}{1in}

\newcommand{\ds}{\displaystyle}
\newcommand{\vs}{\vspace{0.1in}}

\begin{document}



\thispagestyle{empty}

\noindent{\bf  CSCI 150 - Spring 2023} \hspace{1in}

\vspace{.15in}

{\centerline{\bf Exam I Redo Problems  }}

\vspace{.2in}


\noindent{\bf Directions:} If you earned less than a \textbf{Complete} on Exam \#1, your individual rubric sheet will list which problem(s) you need to do, and the due date (likely 2/24/23). Below are the problems you will choose from. \textbf{Please turn in your original in-class Exam and your rubric sheet} when you turn in the redo.



\vs

\vs



\begin{enumerate}

\item \textbf{Booleans} If you need to rework one or two, please make sure to include a brief explanation, and show your work to receive credit.

\begin{enumerate}
\item \verb|True or ( (5 > 2) and (6 != 5))|

\item \verb|('test' < 'text') and (False or (True and False))|


\end{enumerate}

\item \textbf{Tracing} Trace the execution of the following Python program.
\begin{enumerate}

\item \

\begin{verbatim}
x = 7
y = 12
z = y - x + 1
z = z + 2
if z == z + 2:
    x = 9
elif x == 9:
    y = 5
else:
    x = 2
    y = 8

z = x + y
\end{verbatim}


\item \

\begin{verbatim}
a = 4
b = 5
s = 'sad'
t = 'happy'

if s < t or a > b:
    a = a + b

    if a == a + b:
        w = 'now'
    else:
        w = 'then'

else:
    if b != a:
        w = 'later'
        b = b - a
    else:
        w = 'gator'
        a = a + b
\end{verbatim}

\end{enumerate}


\item \textbf{Coding of Functions}

\begin{enumerate}


  \item Write a function \texttt{pairs} which, given three integers $a$, $b$, and $c$, returns \texttt{True} if exactly two of them are equal and \texttt{False} otherwise.

  For example:

  \texttt{pairs(5, 2, 5)} returns \texttt{True}

  \texttt{pairs(1, 2, 3)} returns \texttt{False}

  \texttt{pairs(4, 4, -4)} returns \texttt{True}

  \texttt{pairs(7, 7, 7)} returns \texttt{False}


\item  Usain Bolt wants to write a function \texttt{run\_today} to decide whether the weather is good enough to go for a run. The weather will always be a combination of a temperature (an integer, which could be positive, negative, or zero) and a \texttt{str} condition, which has multiple possible values.

    If the temperature is less than 10 or greater than 100, he will not run, and so it should return \texttt{False}, except if the condition is \texttt{`rain'} in which case he will not run unless the temperature is between 40 and 80 (inclusive). Finally, if the condition is \texttt{`cloudy'} he will run no matter the temperature, even if the temperature is above 100 or below 10.

    The function should return \texttt{True} if he will run and \texttt{False} if he will not.

    For example:

    \texttt{run\_today(75, 'sunny')} will return \texttt{True}

     \texttt{run\_today(5, 'sunny')} will return \texttt{False}

      \texttt{run\_today(50, 'rain')} will return \texttt{True}

       \texttt{run\_today(95, 'rain')} will return \texttt{False}

        \texttt{run\_today(120, 'cloudy')} will return \texttt{True}



\end{enumerate}

\end{enumerate}

\end{document} 