% -*- mode: LaTeX; compile-command: "pdflatex exam2-inclass-fall-2022.tex" -*-
\documentclass{article}

\usepackage{amsmath, url, graphicx, array}

\begin{document}

\title{CSCI 150: Exam 2 -- In Class}
\author{}
\date{Monday, March 13, 2023}

\maketitle

\thispagestyle{empty}

 You  have 50
minutes to complete this exam. \textbf{You are not allowed to use your
  notes, textbook, phone or computer.  Partial credit for incorrect answers can be given only
  if you show your work.}  \bigskip

  \vspace{0.3in}

Print your name : \underline{\phantom{XXXXXXXXXXXXXXXXXXXXXXXXX}}
\vspace{1in}



\newpage

\textbf{For each of questions \ref{traceAAA}--\ref{traceWhile}, trace
  the execution of the given code, showing the function stack and any
  printed output.}

\begin{enumerate}

\item \mbox{} \label{traceAAA}
\begin{verbatim}
def bbb(x : int) -> int:
    y = 3
    if x > y:
        return x - y
    else:
        return x + y

def aaa(x: int):
    print(f'The number is {bbb(x + 2)}.')
    return x + 9
    print('Done!')

def main1():
    print(aaa(4))
    print(aaa(-1))

main1()
\end{verbatim}




\newpage

\item \mbox{} \label{tracef1}
\begin{verbatim}
def f1():
    print('Test')

def f2(n: int):
    f1()
    if n < 2:
        print(n)
    else:
        f1()
        print(n)
        f1()
        print(n+1)
        f1()

def main2():
    f2(1)
    f2(2)

main2()
\end{verbatim}

\newpage
\item \mbox{} \label{traceWhile}

\begin{verbatim}
def main3():
    s = 'raining'
    t = ''
    i = 0
    j = 0

    while i < len(s):
        if s[i] > 'i':
            t += s[i]
        else:
            j += i
        i += 1

    print(t)
    print(j)

main3()
\end{verbatim}

\eject
\item Write a function \verb|count_before| which takes two parameters, a string \verb|s|, which might be empty or contain a number of characters (but will always be a lower case alphabetic), and a string \verb|char|, which will always be a single lower case alphabetic. Your function should return the number of characters in the string \verb|s| which come before \verb|char| in the alphabet.

    For example:
    \begin{itemize}
      \item \verb|count_before('hendrix', 'p')| returns \verb|5| (since each of 'h', 'e', 'n', 'd', and 'i' come before 'p')
      \item \verb|count_before('pepper', 'p')| returns \verb|2| (since only the two 'e's come before 'p')
      \item \verb|count_before('xray', 'e')| returns \verb|1|
      \item \verb|count_before('', 'a')| returns \verb|0|
    \end{itemize}

    \vspace{0.2in}

    \verb|def counter_before(s: str, char: str) -> int:|
    \eject

  \item Write a function \verb|div_two| which takes in a positive integer \verb|n| -- you can assume it will always be positive, and returns the number of times necessary for \verb|n| to be divided by \verb|2| to get less than or equal to \verb|1|. You will want to use regular division here: \verb|/|  (i.e. single slash)

For example:
    \begin{itemize}
      \item \verb|div_two(5)| returns \verb|3|, since we get the sequence 5 (step 0), 2.5 (step 1), 1.125 (step 2), and finally 0.625 (step 3)
      \item \verb|div_two(4)| returns \verb|2|, since we get the sequence 4 (step 0), 2.0 (step 1), 1.0 (step 2)
      \item \verb|div_two(8)| returns \verb|3|, since we get 8, 4.0, 2.0, 1.0
      \item \verb|div_two(11)| returns \verb|4|, since we get 11, 5.5, 2.75, 1.375, 0.6875
      \item \verb|div_two(1)| returns \verb|0|, since we do not need to take any steps!
    \end{itemize}

    \vspace{0.2in}


    \verb|def div_two(n: int) -> int:|

\end{enumerate}

\end{document} 