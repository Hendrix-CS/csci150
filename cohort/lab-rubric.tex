% -*- compile-command: "pdflatex lab-rubric.tex" -*-

\documentclass{article}

\usepackage{pdflscape, booktabs, longtable}
\usepackage[noheadfoot,margin=1in]{geometry}

\begin{document}

\thispagestyle{empty}

\begin{landscape}
  {\Large CSCI 150, Foundations of Computer Science: Lab and Project
    Rubric} \bigskip

  \begin{longtable}{p{1in}p{2.2in}p{2.2in}p{2.2in}p{2.2in}}
    \toprule
    \textbf{Criterion} & Excellent & Satisfactory & Needs improvement \\
    \midrule
    \textbf{Completeness}
    & All parts of the assigned lab are complete.
    & All but the last section are complete, or all are complete with
    some minor holes throughout.
    & Significant portions of the assignment are incomplete.
    \\
    \midrule
    \textbf{Correctness}
    & The submitted program correctly carries out the required
    behavior.  The program never crashes, even when given invalid input.
    & The program largely carries out the required behavior, with some
    small exceptions.  The program does not usually crash, but might
    in a few cases.
    & The program fails to implement the required behavior, or does so
    with some major problems; or the program is sometimes correct but
    often crashes.
    \\
    \midrule
    \textbf{Documentation}
    & The program is well-documented, with comments explaining the
    behavior, parameters, and return value of each function, and
    comments explaining any interesting or nonobvious code.  If there
    is a README file, it thoroughly explains the program, its purpose
    and features, and how to use it.
    & The program is mostly well-documented, but there may be some
    missing comments, or some existing comments may be too terse,
    vague, or confusing.  A README file may be lacking some detail.
    & Many parts of the program are uncommented, or comments may be
    unhelpful.
    \\
    \midrule
    \textbf{Style}
    & The program uses good style throughout and is a pleasure to
    read: it uses consistent indentation and spacing, uses informative
    variable names, and avoids redundancy. Lines are not too
    long. Comments contain well-written, complete sentences.
    & The program mostly uses good style, with a few lapses or issues
    that need improvement that do not impair the overall readability.
    & Stylistic issues make the program difficult to read and understand.
    \\
    \midrule \textbf{Reflection}
    & The submitted reflection document meets the stated requirements and
    contains a thoughtful discussion showing clear
    evidence of learning.
    & The reflection document mostly meets the stated requirements and
    shows evidence of learning.
    & The reflection document is missing, or contains a reflection
    that does not go beyond a superficial level.
    \\
  \end{longtable} \vspace{0.5in}

\end{landscape}


\end{document}
