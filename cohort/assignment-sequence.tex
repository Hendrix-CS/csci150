% -*- compile-command: "pdflatex assignment-sequence.tex" -*-
\documentclass{article}

\usepackage{url}

\begin{document}

\title{Assignment Sequence: CSCI 150}
\author{Brent Yorgey}

\maketitle

For my assignment sequence for CSCI 150, Foundations of Computer
Science, I have chosen to focus on the sequence of weekly lab
assignments.  Students attend a three-hour lab session each week where
they work on these assignments under the supervision of an instructor
and TAs; if any portion of the assignment remains unfinished at the
end of the three hours (which is expected; after the first few weeks
the labs are intentionally designed to take longer than three hours)
students are expected to complete the assignment and submit it by the
the lab session the following week.

These lab assignments are the most central, ``big'' assignments in the
course, and are one of the central tools we employ to give students
concrete experience with the concepts being learned in class, and to
help cement the concepts in their minds.  The material is very
cumulative, in the sense that most of the labs require students to
make use of skills developed on previous labs, in addition to whatever
skill or concept is being developed by the lab itself.  From a
pedagogical point of view, it would seem optimal to create assignments
that are not quite so relentlessly cumulative, to allow students some
breathing room and allow students who need more practice with a
concept some time to catch up, but given the nature of my discipline I
don't see any good way to do this.

\begin{itemize}
\item Lab 1: Minecraft Hour of Code

  This lab typically comes after only a single class meeting, so it is
  rather lighthearted and fun and requires almost no background
  knowledge to complete successfully.  Students complete an activity
  on a website designed to introduce students to basic programming
  (\url{http://code.org/}), where they write simple programs to help a
  character on the screen complete various tasks. Despite the lab's
  simplicity, some of the concepts the students encounter in this lab
  (sequencing, conditional statements, repetition, logic errors) are
  foundational for the rest of the course.

  \textbf{Purpose}: Overall the lab is designed to give them a
  successful taste of computer programming while introducing them to
  many central concepts that will reoccur throughout the course.

\item Lab 2: Kepler and Newton

  This is the first lab that students complete using the Python
  programming language, which is used throughout the remainder of the
  course.  In it, they create programs to get input from the user,
  perform calculations based on given equations, and display the
  results.

  \textbf{Purpose}: The lab is designed to give students practice with
  the basic syntax and operations of Python, and to help them see some
  of the connections between science, mathematics, and computation.

\item Lab 3: Diagnosing Heart Disease

  In this lab students use conditional statements (first seen in Lab
  1, but now transposed into the context of the Python language) and
  logic to create simple prediction models to determine whether
  patients are likely to have heart disease, based on a real-world
  data set.

  \textbf{Purpose}: The lab is designed to give students experience
  with the mechanics of writing conditional statements in Python, and
  more importantly, with thinking about a problem rigorously and
  logically using formal boolean logic.

  It is interesting to note (I had never explicitly considered it!)
  that Labs 2 and 3 are in and of themselves interchangeable, that is,
  successful completion of Lab 3 does not require mastery of the
  concepts from Lab 2.  However, in terms of sequencing of topics in
  the course, it would be very difficult to come up with good in-class
  examples illustrating the ideas behind Lab 3 without first
  introducing the material practiced in Lab 2.

\item Lab 4: This Day in History

  This lab serves as a comprehensive review prior to the first exam.
  It does not introduce any new concepts, but uses all the skills and
  concepts developed in the previous labs and pushes them a bit
  further in terms of the required sophistication of their
  application.

\item Lab 5: Guess My Number

  In this lab students implement a classic number-guessing game, where
  one player chooses a secret number and the other player makes
  repeated guesses, with the chooser telling them whether each guess
  is too low or too high.  Students implement two computer-vs-human
  versions of the game, with the computer playing the roles of both
  chooser and guesser.

  \textbf{Purpose}: The primary purpose of this lab is to give
  students practice writing ``while loops'' (one of the most primitive
  forms of repetition).  It requires (and assumes!) that students have
  mastered the skills from the first four labs.

\item Lab 6: Mutation is the Word

  In this lab, students implement a classic word game, where the
  player tries to change a starting word into an ending word by
  changing one letter at a time, such that every intermediate step is
  also a valid English word.

  \textbf{Purpose}: 

\item Lab 7: List Editor

  In this lab, students implement a ``line-oriented text editor'' of
  the type that some computers had prior to the ubiquity of screen
  interfaces.

  \textbf{Purpose}: The primary purpose is to give students experience
  building and manipulating lists.  XXX

\item Lab 8: Fractal Recursion

  In this lab, students implement 

  \textbf{Purpose}: Nominally, the purpose of this lab is to give
  students experience thinking recursively and implementing recursive
  functions.  To be honest, it mostly fails at this purpose, but
  that's OK since although this is a topic to which we want to expose
  students, they won't need it from a practical point of view unless
  they go on to take more CSCI courses, in which case they will have
  plenty more opportunities to practice it.  The real, practical
  purpose of the lab is to serve as an enjoyable project (they get to
  make pretty pictures) which the students can complete within the 3
  hour lab period before leaving for spring break.

\item Lab 9: Caesar's Secrets

  In this lab, students
\item Lab 10: Sentiment Analysis (dictionaries)
\item Lab 11: Water Jugs (classes \& objects)
\item Lab 12: Processing
\item Lab 13: On Stuckness \& Debugging
\end{itemize}

\end{document}
