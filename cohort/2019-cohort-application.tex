% -*- compile-command: "rubber -d 2019-cohort-application.tex" -*-

\documentclass{article}

\begin{document}
\noindent
\textbf{2019 Teaching Cohort application} \\
\textbf{Brent Yorgey, Math \& Computer Science} \bigskip

I would like to participate in the 2019 Teaching Cohort to help
enhance \textbf{CSCI 150, Foundations of Computer Science}, which I
will teach in the Fall of 2019.  This course fits the goals of the
cohort perfectly: it is a foundational course with some big challenges
in regards to diversity and inclusion.

CSCI 150 is required for the major in Computer Science (CS), and is
taken as a first course by almost all CS majors (except for a very
small number who receive AP credit for the course).  It gives students
a thorough grounding in the skill of computer programming and
introduces many of the most foundational concepts in Computer Science,
such as conditionals, iteration, recursion, decomposition, and
encapsulation.  The course is also taken by a large number of students
who are not intending to pursue a CS major, whether because they see
programming as a useful skill to enhance their studies in another
field, because they think it will make them a more informed citizen in
today's world, or simply because they need an NS-L credit.  So the
course is not only foundational, but also impacts a relatively large
number of students---at the moment, typically around 75-80 students
take the course per year.

One strength of the current course is that it gives students ample
opportunities for hands-on, engaged learning---especially via the lab
sections where students work in pairs to implement complex programming
assignments, as well as projects where students work independently,
within some given parameters, to create programs of their own design.
Another strength of the course is the way that it breaks down complex
concepts into simple, manageable building blocks, making the course
content accessible (in theory) to a wide range of students without any
assumptions about prior experience with computers or programming.

Computer science, as a whole, has a big problem with diversity and
inclusion.  The field tends to be dominated by white males; women and
people of color are severely underrepresented.  To a large extent, we
see this mirrored in our courses; for example, in recent years only
about 15-25\% of CS majors have been women.  More troubling, however,
is the drop in diversity we typically see between CSCI 150 and other
CSCI courses.  For example, this semester 21 out of 50 students (40\%)
taking CSCI 150 are women, and the distribution of students' ethnic
backgrounds seems to match the overall distribution of Hendrix
students fairly closely.  In other courses beyond CSCI 150, however,
the percentage of women drops, and there are hardly any students of
color other than international students.

Without a doubt, there are many different factors which combine to
influence students in their choice of a major, many of which are out
of our control; it would be silly to blame these statistics entirely
on the teaching of CSCI 150.  However, the course surely has some
effect, and it would be equally irresponsible to simply throw up our
hands and say there's nothing we can do.

Although this has been on my mind for a while, intentions are not
enough; I simply do not know what kind of enhancements to the course
might make it more welcoming or accessible for students
underrepresented in CS.  This opportunity would therefore be a perfect
fit: I would be excited to spend time with other faculty reading about
some of the latest ideas in inclusive pedagogy and thinking deeply
about how to enhance our courses, something that I am unlikely to be
able to do well on my own.

% seniors: 3/12 women (25%)
% juniors: 2/9 (22%)
% sophomores: 2/15 (13%)
% freshman: 4/14 (29%)

\subsection*{Spring 2019 Schedule}

My spring teaching schedule is as follows:

\begin{itemize}
\item MWF 9-10 (CSCI 151)
\item MWF 11-12 (CSCI 150)
\item T 1-4 (CSCI 151 lab)
\end{itemize}

I do not know of any other standing commitments or time/day restrictions.

\end{document}
